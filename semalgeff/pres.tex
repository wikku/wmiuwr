\documentclass{beamer}
\usefonttheme[onlymath]{serif}
\usepackage[utf8]{inputenc}
\usepackage{silence}
\usepackage{amsmath,amsfonts,amssymb,amsthm}
\usepackage{mathtools}
\usepackage{newunicodechar}
\usepackage{stmaryrd}
\usepackage{mathpartir}

\title{Relational interpretation of algebraic effects}
\author{Wiktor Kuchta}
\date{January 17, 2024}

\newcommand{\N}{\mathbb{N}}
\newcommand{\V}{\mathcal{V}}
\newcommand{\G}{\mathcal{G}}
\newcommand{\E}{\mathcal{E}}
\renewcommand{\S}{\mathcal{S}}
\newcommand{\K}{\mathcal{K}}
\newcommand{\PS}{\mathcal{P}}
\newcommand{\ol}{\overline}
\newcommand{\+}{\enspace}
\newcommand{\keyword}[1]{\textsf{\textup{#1}}}
\newcommand{\KwHandle}{\keyword{handle}}
\newcommand{\Handle}{\KwHandle\;}
\newcommand{\KwLift}{\keyword{lift}}
\newcommand{\Lift}[1]{\KwLift#1}
\newcommand{\Free}{\textrm{-}\mathrm{free}}
\newcommand{\subst}[2]{[#1/#2]}
\newcommand{\Op}{\mathit{op}}
\newcommand{\Int}{\mathsf{int}}
\newcommand{\Ask}{\mathsf{ask}}

\newcommand{\kT}{\mathsf{T}}
\newcommand{\kE}{\mathsf{E}}
\newcommand{\kR}{\mathsf{R}}

\newunicodechar{│}{\mid} % Digr vv
\newunicodechar{╱}{\mathbin{/}} % Digr FD
\newunicodechar{∅}{\emptyset} % Digr /0
\newunicodechar{□}{\square} % Digr OS
\newunicodechar{∷}{::} % Digr ::
\newunicodechar{α}{\alpha}
\newunicodechar{ε}{\varepsilon}
\newunicodechar{γ}{\gamma} % Digr g*
\newunicodechar{κ}{\kappa}
\newunicodechar{λ}{\lambda}
\newunicodechar{μ}{\mu}
\newunicodechar{ρ}{\rho}
\newunicodechar{σ}{\sigma}
\newunicodechar{τ}{\tau}
\newunicodechar{η}{\eta} % Digr y*
\newunicodechar{Δ}{\Delta}
\newunicodechar{Γ}{\Gamma}
\newunicodechar{ℕ}{\N} % Digr NN 8469 nonstandard
\newunicodechar{∪}{\cup} % Digr )U
\newunicodechar{≈}{\approx} % Digr ?2
\newunicodechar{∈}{\in} % Digr (-
\newunicodechar{∃}{\exists} % Digr TE
\newunicodechar{∀}{\forall} % Digr FA
\newunicodechar{∧}{\wedge} % Digr AN
\newunicodechar{∨}{\vee} % Digr OR
\newunicodechar{⊥}{\bot} % Digr -T
\newunicodechar{⊢}{\vdash} % Digr \- 8866 nonstandard
\newunicodechar{⊨}{\models} % Digr \= 8872 nonstandard
\newunicodechar{⇒}{\Rightarrow} % Digr =>
\newunicodechar{⇔}{\iff} % Digr ==
\newunicodechar{↦}{\mapsto} % Digr T> 8614 nonstandard
\newunicodechar{⟦}{\llbracket} % Digr [[ 10214 nonstandard (needs pkg stmaryrd)
\newunicodechar{⟧}{\rrbracket} % Digr ]] 10215 nonstandard

% cursed
\WarningFilter{newunicodechar}{Redefining Unicode}
\newunicodechar{·}{\ifmmode\cdot\else\textperiodcentered\fi} % Digr .M
\newunicodechar{→}{\ifmmode\rightarrow\else\textrightarrow\fi} % Digr ->


\begin{document}

\begin{frame}
	\titlepage
\end{frame}

\section{Intro}

\begin{frame}
	\frametitle{Background: operational semantics}
	So far, we've formalized the semantics of effects and handlers as step relations between expressions:
	\begin{equation*}
		\begin{aligned}[c]
			&\Handle \\
			&\quad\Ask () + 5 \\
			&\{ \Ask\;()\;k → k\;10\}
		\end{aligned}
	\qquad→\qquad
		\begin{aligned}[c]
			&\Handle \\
			&\quad10 + 5 \\
			&\{ \Ask\;()\;k → k\;10\}
		\end{aligned}
	\end{equation*}
\end{frame}

\begin{frame}
	\frametitle{Background: type-and-effect systems}
	Expressions have not only a type, but also potential effects:
	$$\Ask() + 5 : \Int ╱ \Ask$$

	\pause

	Formally, we have %syntax-directed
	rules for constructing typing judgments, e.g.,
	\begin{mathpar}
	\inferrule
		{Γ, x:τ_1 ⊢ e : τ_2 ╱ ρ}
		{Γ ⊢ λx.\,e : τ_1 →_ρ τ_2 ╱ ·}
	\end{mathpar}
	% But this is not just a game of pushing Greek letters around.
	% There is a purpose behind it, a meaning.
	% We have an intuition for it.
\end{frame}

\begin{frame}
	\frametitle{But what does $e : τ ╱ ρ$ mean?}

	Intuitively,
		``$e$ may perform $ρ$ before evaluating to a $τ$''.
	% $e$ performing an effect or evaluating to something is a question for the *semantics*

	\pause
	So it refers to how the program behaves – to the \textit{semantics}.
	\pause

	Such program properties are undecidable.

	A type system can only \textit{approximate} the real notion.
\end{frame}

\begin{frame}
	\frametitle{Agenda}
	\begin{enumerate}
	\item
		Definition of semantic typing, using the step relation $→$.
		$$Γ ⊨ e : τ ╱ ρ$$
	\item
		Compatibility with (i.e., soundness of) syntactic typing.
		$$Γ ⊢ e : τ ╱ ρ$$
		(\textit{Corollary:} well-typed programs $⊢ e : τ ╱ ·$ indeed terminate.)
	\item
		Row polymorphism. Semantic equivalence.
		$$Δ; Γ ⊨ e_1 ≈ e_2 : τ ╱ ρ$$
	\end{enumerate}
\end{frame}



\section{A simple calculus with effect handlers}

\frame{\tableofcontents[currentsection]}

\begin{frame}
	\frametitle{Syntax}
	We extend the call-by-value $λ$-calculus.

	Assume $\Op$ ranges over a set of operation names, e.g. $\mathrm{ask}, \mathrm{pick}$...

\begin{align*}
	%\textrm{Var} ∋ f,r,x,y,…    &                    \\%&\textrm{(variables)}\\
	v,u          &::= x │ λx.\,e \\
	e            &::=
		%v │ e\;e │ \Do v │ \Handle e\,\{x,r.\,e;\,x.\,e\}
		v │ e\;e │ \Op\;v │ \Handle e\,\{\Op\;x\;k→\,e\} \\
\end{align*}

\end{frame}

\begin{frame}
	\frametitle{Reduction}
	$$
	K            ::=
		□ │ K\;e │ v\;K │ \Handle K\,\{\Op\;x\;k→e\}
	$$

\begin{mathpar}

	(λx.\,e)\;v ↦ e\subst{v}{x}

	%\Lift{v} ↦ v

	\inferrule
		{K \Op\Free \\ v_c = λz.\,\Handle K[z]\,\{\Op\;x\;k→e_h\}}
		{\Handle K[\Op\; v]\,\{\Op\;x\;k→e_h\} ↦ e_h\subst{v}{x}\subst{v_c}{k}}

	\Handle v\,\{\Op\;x\;k→e_h\} ↦  v
\end{mathpar}

\begin{mathpar}
	\inferrule
		{e_1 ↦ e_2}
		{K[e_1] → K[e_2]}
\end{mathpar}
\end{frame}


\section{Semantic typing.}
\frame{\tableofcontents[currentsection]}



\begin{frame}
	Consider closed terms first.% Ultimately, we want to define $\E⟦τ/ρ⟧$.
	\begin{align*}
		%\V⟦·⟧ &: \mathrm{Type} → \PS(\mathrm{Val})\\
		\V⟦\mathsf{bool}⟧ &= \{ \mathsf{true}, \mathsf{false} \} \\
		\V⟦\mathsf{σ →_ρ τ}⟧ &= \{ λx.\,e \mid ∀v∈\V⟦σ⟧.\,e\subst{v}{x} ∈ \E⟦τ/ρ⟧ \}
	\end{align*}
	\pause
	\begin{align*}
		\E⟦τ'╱\ol{\Op_i:{σ_i{⇒}τ_i}}⟧ &= \{ e │ ∃v∈\V⟦τ'⟧.\,e→^*v \}\\
						&\mspace{2mu}∪\mspace{2mu} \{ e \\
						&\quad\,│ ∃K, i, v.\, e →^* K[\Op_i\;v] \\
						&\quad\mspace{0.5mu}∧ K\,\Op_i\textrm{-free}\\
						&\quad\mspace{0.5mu}∧ v∈\V⟦σ_i⟧\\
						&\quad\mspace{0.5mu}∧ ∀u∈\V⟦τ_i⟧.\,K[u] ∈ \E⟦τ'╱\ol{\Op_i:{σ_i{⇒}τ_i}}⟧ \}
	\end{align*}
\end{frame}

\begin{frame}
	\frametitle{Inductive definition, formally in set theory}
	$\E⟦τ'╱\ol{\Op_i:{σ_i{⇒}τ_i}}⟧$ is the least fixed point of the following function.
	\begin{align*}
		\E'({\color<2->{orange}X})&= \{ e │ ∃v∈\V⟦τ'⟧.\,e→^*v \}\\
						&\mspace{2mu}∪\mspace{2mu} \{ e \\
						&\quad\,│ ∃K, i, v.\, e →^* K[\Op_i\;v] \\
						&\quad\mspace{0.5mu}∧ K\,\Op_i\textrm{-free}\\
						&\quad\mspace{0.5mu}∧ v∈\V⟦σ_i⟧\\
						&\quad\mspace{0.5mu}∧ ∀u∈\V⟦τ_i⟧.\,K[u] ∈ {\color<2->{orange}X} \} \\
	\end{align*}
	\pause
	The function $\E'$ is monotone, so it has a least fixed point by the Knaster-Tarski theorem.
	%\pause
	%More precisely, it's the intersection of its prefix points ($\E'$-closed sets):
	%$$\bigcap\{X │ \E'(X) ⊆ X \}.$$

\end{frame}

\begin{frame}
	\frametitle{Examples (blackboard)}

	$$\mathrm{ask}() \;\mathsf{xor}\; \mathrm{ask()} : \mathsf{bool} ╱ {\mathrm{ask}:\mathsf{unit}⇒\mathsf{bool}}$$
\end{frame}

\begin{frame}
	\frametitle{Closing}

	$$\G⟦Γ⟧ = \{ γ : \mathrm{Var} → \mathrm{Val} \mid ∀x:τ∈Γ.\,γ(x) ∈ \V⟦τ⟧ \}$$

	$$ Γ ⊨ e : τ ╱ ρ \iff ∀γ∈\G⟦Γ⟧.\,γ(e)∈\E⟦τ/ρ⟧ $$
\end{frame}

\section{Simple type system. Compatibility.}
\frame{\tableofcontents[currentsection]}
\begin{frame}
	\frametitle{Simple typing}
	\begin{mathpar}
	τ ::= b │ τ→_ρτ

	ε ::= \Op:τ ⇒ τ

	ρ ::= · │ ε · ρ


	\inferrule
		{x : τ ∈ Γ}
		{Γ ⊢ x : τ ╱ ·}

	\inferrule
		{Γ, x:τ_1 ⊢ e : τ_2 ╱ ρ}
		{Γ ⊢ λx.\,e : τ_1 →_ρ τ_2 ╱ ·}

	\inferrule
		{Γ ⊢ e_1 : τ_1 →_ρ τ_2 ╱ ρ \\ Γ ⊢ e_2 : τ_1 ╱ ρ}
		{Γ ⊢ e_1\,e_2 : τ_2 ╱ ρ}

	\inferrule
		{Γ ⊢ v : τ_1 ╱ ·}
		{Γ ⊢ \Op\;v : τ_2 ╱ \Op:τ_1 \Rightarrow τ_2}
	%\inferrule
	%	{Γ ⊢ e : τ ╱ τ_1{⇒}τ_2 · ρ \\
	%	 Γ,x:τ_1,r:τ_2→_ρ τ_r ⊢ e_h : τ_r ╱ ρ \\
	%	 Γ, x:τ ⊢ e_r : τ_r ╱ ρ}
	%	{Γ ⊢ \Handle e\;\{x,r.\,e_h;x.\,e_r\} : τ_r ╱ ρ}

	\inferrule
		{Γ ⊢ e : τ ╱ \Op:τ_1{⇒}τ_2 · ρ \\
		 Γ,x:τ_1,k:τ_2→_ρ τ ⊢ e_h : τ ╱ ρ}
		{Γ ⊢ \Handle e\;\{\Op\;x\;k→e_h\} : τ ╱ ρ}

	%\inferrule
	%	{Δ ⊢ τ_1 <: τ_2 \\ Δ ⊢ ρ_1 <: ρ_2 \\ Δ;Γ ⊢ e : τ_1 ╱ ρ_1}
	%	{Δ;Γ ⊢ e : τ_2 ╱ ρ_2}

\end{mathpar}
\end{frame}

\begin{frame}
	\frametitle{Compatibility}

	For a typing rule
	$$
	\inferrule
		{Γ, x:τ_1 ⊢ e : τ_2 ╱ ρ}
		{Γ ⊢ λx.\,e : τ_1 →_ρ τ_2 ╱ ·}
	$$
	the corresponding compatibility lemma is

	\begin{lemma}[Lambda compatibility]
	If $$Γ, x:τ_1 ⊨ e : τ_2 ╱ ρ$$ then $$Γ ⊨ λx.\,e : τ_1 →_ρ τ_2 ╱ · $$
	\end{lemma}

\end{frame}

\begin{frame}
	\frametitle{Tying everything together: fundamental theorem}
	By the compatibility lemmas for each typing rule, we have
	$$Γ ⊢ e : τ ╱ ρ \implies Γ ⊨ e : τ ╱ ρ.$$

	In particular, if $⊢ e : τ ╱ ·$, then $e ∈ \E⟦τ/·⟧$.
	For the empty row $·$, the only possibility is termination to a value.

\end{frame}




\section{Polymorphism. Semantic equivalence.}
\frame{\tableofcontents[currentsection]}
 \begin{frame}
	\frametitle{Polymorphic typing: kinding}
	\begin{mathpar}
	τ ::= α │ ∀α.\,τ │ b │ τ→_ρτ │ · │ (\Op:τ⇒τ) · ρ \\
		κ ::= \kT │ \kR \\

	\inferrule
		{Δ ⊢ τ_1 ∷ \kT \\ Δ ⊢ ρ ∷ \kR \\ Δ ⊢ τ_2 ∷ \kT}
		{Δ ⊢ τ_1 →_ρ τ_2 ∷ \kT}
		\\

	\inferrule
		{α ∷ κ ∈ Δ}
		{Δ ⊢ α ∷ κ}

	\inferrule
		{Δ, α :: κ ⊢ τ ∷ \kT}
		{Δ ⊢ ∀α::κ.\, τ ∷ \kT}

%	\inferrule
%		{Δ ⊢ τ_1 ∷ \kT \\ Δ ⊢ τ_2 ∷ \kT}
%		{Δ ⊢ τ_1 \Rightarrow τ_2 ∷ \kE}
		\\


	\inferrule
		{ }
		{Δ ⊢ · ∷ \kR}

	\inferrule
		{Δ ⊢ τ_1 ∷ \kT \\ Δ ⊢ τ_2 ∷ \kT \\ Δ ⊢ ρ ∷ \kR}
		{Δ ⊢ (\Op : τ_1 ⇒ τ_2) · ρ ∷ \kR}
\end{mathpar}

\end{frame}

\begin{frame}
	\frametitle{Polymorphic typing}
	$$
		τ ::= α │ ∀α.\,τ │ b │ τ→_ρτ │ · │ (\Op:τ⇒τ) · ρ
	$$

	Old rules get a type variable environment $Δ$, e.g.
$$
	\inferrule
		{Δ;Γ, x:τ_1 ⊢ e : τ_2 ╱ ρ}
		{Δ;Γ ⊢ λx.\,e : τ_1 →_ρ τ_2 ╱ ·}
$$

Rules pertaining to polymorphism actually use it:
$$
	\inferrule
	{Δ,α∷κ;Γ ⊢ e : τ ╱ ·}
	{Δ;Γ ⊢ e : ∀α∷κ.\,τ ╱ ·}
$$
$$
		\inferrule
		{Δ ⊢ σ ∷ κ \\ Δ;Γ ⊢ e : ∀α∷κ.\, τ ╱ ρ}
		{Δ;Γ ⊢ e : τ\subst{σ}{α} ╱ ρ}
$$
\end{frame}

\begin{frame}
	\frametitle{Lift}
	What if we have a row
	$$(\Op : {σ⇒τ}) · ρ$$ and then substitute something
	that also has $\Op$ for $ρ$?

	\vspace{1ex}
	To be able to use the second occurrence of $\Op$, we introduce \textit{lift},
	which makes operations inside skip the nearest handler:
	$${\color<3,4>{orange}\Handle} {\color<2,4>{blue}\Handle} ({\color<2,4>{blue}\Op\;1}) + (\Lift^\Op{({\color<3,4>{orange}\Op\;\mathtt{"text"}})})\, {\color<2,4>{blue}\{\Op ... \}}\, {\color<3,4>{orange}\{\Op ... \}}$$

\pause\pause\pause

%Formally, we use a \textit{freeness} judgment on contexts:
%\begin{mathpar}
%	\inferrule
%		{ }
%		{0\Free(□)}
%
%	\inferrule
%		{n\Free(K)}
%		{n\Free(K\;e)}
%
%	\inferrule
%		{n\Free(K)}
%		{n\Free(v\;K)}
%
%	\inferrule
%		{n\Free(K)}
%		{{n+1}\Free(\Lift{K})}
%
%	\inferrule
%		{{n+1}\Free(K)}
%		%{n\Free(\Handle K\,\{x,r.\,e_h;\,x.\,e_r\})}
%		{n\Free(\Handle K\,\{x,r.\,e_h\})}
%\end{mathpar}
\end{frame}

\begin{frame}
	\frametitle{Binary relations: interpretation of kinds}

	\begin{align*}
		⟦\kT⟧ &= \mathcal{P}(\mathrm{Val}^2) \\
		⟦\kR⟧ &= \mathcal{P}(\mathrm{Exp}^2\times(\mathrm{Op}→ℕ)^2\times⟦\kT⟧) \\
	\end{align*}

\end{frame}

\begin{frame}
	\frametitle{Binary relations: values and effects}
	\begin{align*}
		⟦\mathsf{bool}⟧_η &= \{ (\mathsf{true}, \mathsf{true}), (\mathsf{false}, \mathsf{false}) \} \\
		⟦\mathsf{σ →_ρ τ}⟧_η &= \{ (λx.\,e_1, λx.\,e_2)\\
							   &\quad\,\mid ∀(v_1,v_2)∈⟦σ⟧_η.\,(e_1\subst{v_1}{x}, e_2\subst{v_2}{x}) ∈ \E⟦τ/ρ⟧_η \}
		\end{align*}
	\begin{align*}
		⟦\Op:σ⇒τ⟧_η &= \{ (\Op\;v_1, \Op\;v_2, (\Op↦0), (\Op↦0), ⟦τ⟧_η)\\
					  &\quad\,│ (v_1, v_2) ∈ ⟦σ⟧_η \} \\
		⟦·⟧_η &= ∅ \\
		⟦(\Op:ε)·ρ⟧_η &= ⟦\Op:ε⟧_η ∪ (⟦ρ⟧_η\uparrow^\Op)
	\end{align*}
	\begin{align*}
	⟦α⟧_η &= η(α) \\
	⟦∀α∷κ.\,τ⟧_η &= \bigcap_{μ∈⟦κ⟧} ⟦τ⟧_{η[α↦μ]}
	\end{align*}
\end{frame}

\begin{frame}
	\frametitle{Binary relations: expressions}

	\begin{align*}
		\E⟦τ╱ρ⟧_η &= \{ (e_1,e_2) │ ∃(v_1,v_2)∈⟦τ⟧_η.\,e_i→^*v_i \}\\
						&\mspace{2mu}∪\mspace{2mu} \{ (e_1,e_2) \\
						&\quad\,│ ∃K_1, K_2, (e_1', e_2', f_1, f_2, μ)∈⟦ρ⟧_η.\, e_i →^* K_i[e_i'] \\
						&\quad\mspace{0.5mu}∧ K_i\,f_i\Free \\
						&\quad\mspace{0.5mu}∧ ∀(u_1,u_2)∈μ,(K_1[u_1], K_2[u_2]) ∈ \E⟦τ╱ρ⟧_η \} \\
	\end{align*}
\end{frame}

\begin{frame}
	\frametitle{Equivalence examples (blackboard)}
	\begin{align*}
	&f: ∀α.\,(1 →_α \mathsf{int})­→_α τ ⊨ \\
	&\Handle f\;\mathrm{ask}\;\{\mathrm{ask}\;\_\;k → k\;5 \}
	≈ f\; (λx.\,5) : τ ╱ ·
	\end{align*}
\end{frame}

\section{Homework}

\begin{frame}
	\frametitle{Problem 1: Finish the equivalence example}
	We have
	\begin{equation}
		R = \Big\{\big(\mathrm{ask}\;(), (λx.\,5)\;(), (\mathrm{ask}↦0), ∅, \{(5,5)\}\big)\Big\}
	\end{equation}
	\begin{equation}
		(f_1\;\mathrm{ask}, f_2\;(λx.\,5)) ∈ \E⟦τ ╱ α⟧_{[α↦R]}
	\end{equation}
	Show by induction on (2) that
	$$(\Handle f_1\;\mathrm{ask}\;\{\mathrm{ask}\;x\;k→k\;5\}, f_2\;(λx.\,5)) ∈ \E⟦τ ╱ ·⟧_∅$$

	You should only need to know this about freeness:
	$K\,(\mathrm{ask}↦0)\Free$ means $\Handle K[\mathrm{ask} ()] \{\mathrm{ask}...\}$ can be reduced.
\end{frame}

\begin{frame}
	\frametitle{Problem 2: Interpretation of polymorphic operations}
	Consider polymorphic operations, e.g.
	$$(\mathrm{id}\;(λx.\,x+2))\;(\mathrm{id}\;40) : \mathsf{int} ╱ \mathrm{id} : {α{∷}\kT}.\,{α⇒α}$$
	(This is a different feature than polymorphic effects that let you e.g. have a $\mathrm{state}\;\textsf{int}$ handler and $\mathrm{state}\;\textsf{string}$ handler.)

	Which semantic interpretation is correct? Why?

	\begin{equation}
		⟦\mathrm{op} : {α{∷}κ}.\,σ⇒τ⟧_η = \bigcap_{μ∈⟦κ⟧} ⟦\mathrm{op} : σ⇒τ⟧_{η[α↦μ]}
	\end{equation}
	\begin{equation}
		⟦\mathrm{op} : {α{∷}κ}.\,σ⇒τ⟧_η = \bigcup_{μ∈⟦κ⟧} ⟦\mathrm{op} : σ⇒τ⟧_{η[α↦μ]}
	\end{equation}


\end{frame}


\end{document}
