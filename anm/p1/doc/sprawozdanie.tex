\documentclass[a4paper, 12pt]{article}
\usepackage[utf8]{inputenc}
\usepackage{polski}
\usepackage{amsmath,amssymb,amsfonts,amsthm}
\usepackage{graphicx}
\usepackage{hyperref}

\DeclareMathOperator{\sgn}{sgn}
\newcommand{\macheps}{\textsf{\textbf{u}}}

\date{\today}
\title{
	\LARGE \textbf{Pracownia z analizy numerycznej} \\
	\Large Sprawozdanie do zadania P1.5 \\
	\large Prowadzący: Filip Chudy}
\author{Wiktor Kuchta}

\newtheorem{tw}{Twierdzenie}

\begin{document}

\maketitle

\section{Wstęp}
Funkcja eksponencjalna dzięki swoim własnościom jest jedną z najważniejszych
funkcji w matematyce czystej i stosowanej.
Niniejsze sprawozdanie sprawdza metodę obliczania $e^x$ polegającą na
skorzystaniu z szeregu Taylora po zastosowaniu wzorów redukujących problem
do $|x| < \frac{1}{2}\ln 2$.
Obliczenia numeryczne były wykonane w 64-bitowej arytmetyce zmiennoprzecinkowej.

\section{Algorytm obliczania $e^x$}
\subsection{Sprowadzenie do mniejszej dziedziny}
Przyjmijmy $z = x\log_2 e = x / \ln 2$, $m = \lfloor z + \frac{1}{2} \rfloor$.
Wtedy
$$e^x =
2^{x\log_2 e} = 2^z = 2^{m + (z-m)} = 2^m 2^{z-m} = 2^m e^{(z-m)\ln 2}.$$
Liczba $m$ to zaokrąglenie
$z$ do najbliższej liczby całkowitej\footnote{
We wzorze w treści zadania powinno być zaokrąglanie w stronę zera
(obcinanie cyfr po przecinku) zamiast podłogi
albo pominięte $\sigma = \sgn x$ tak jak tutaj
}, więc $|z-m| < \frac{1}{2}$.
Z powyższego wzoru wynika algorytm, który sprowadza problem do
$(-\frac{1}{2}\ln 2, \frac{1}{2}\ln 2)$
w kilku prostych operacjach zmiennoprzecinkowych:
dzielenie przez stałą $\ln 2$, zaokrąglanie, odejmowanie, mnożenie przez
$\ln 2$ (i potem dodanie liczby całkowitej $m$ do cechy).

Możemy także skorzystać ze wzoru $e^{-x} = 1/e^x$, aby zredukować problem do
liczb nieujemnych, ale na tym etapie korzyści z tego nie są jasne.

\subsection{Algorytm obliczania sumy częściowej}

Obliczanie
$$
\sum_{k=0}^n \frac{x^k}{k!} = 1 + x + \frac{x^2}{2} + \dots + \frac{x^n}{n!}
$$
wyraz po wyrazie
sprawdzając, dla jakiego $n$ mamy $|x^{n+1}/(n+1)!|$ mniejsze od ustalonego
$\varepsilon$ wiąże się z utratą wydajności lub
dokładności algorytmu w porównaniu z użyciem schematu Hornera dla
ustalonego $n$.

Jeśli $0 < x < \frac{1}{2}\ln 2$, to wyrazy sumy będą szybko malejące.
Aby zminimalizować błędy przy sumowaniu tych składników, powinniśmy
zawsze dodawać pary liczb zmiennoprzecinkowych o możliwie bliskich rzędach
wielkości, a więc sumować wyrazy od najmniejszych (od prawej).
Ale dodawanie od prawej wymagałoby najpierw znalezienia odpowiednio
małego wyrazu lub obliczanie i spamiętywanie ich od lewej, czyli
niepotrzebnych obliczeń lub też użycia pamięci.

Sytuacja jest bardziej skomplikowana dla argumentów $-\frac{1}{2}\ln 2 < x < 0$,
ale sumowanie od końca też powinno być lepsze niż od początku, bo wartości
bezwzględne wyrazów maleją szybko, więc nie dojdze do dużej utraty cyfr
znaczących z powodu dodawania składników o przeciwnych znakach.

Z drugiej strony obliczanie tej sumy dla ustalonego $n$ to obliczanie wartości
ustalonego wielomianu, na co optymalnym algorytmem jest schemat Hornera.
Nie wykonuje on porównań zmiennoprzecinkowych i wykonuje dwa razy mniej mnożeń,
jest prostszy i teoretycznie dokładniejszy niż wcześniej opisywany sposób.

Dla porównania zaimplementowałem trzy algorytmy:
\texttt{naiveexp} sumuje kolejno wyrazy dopóki są większe od parametru $\varepsilon$,
\texttt{rnaiveexp} tak samo ale spamiętuje wyrazy i sumuje je od końca,
\texttt{hornerexp} oblicza schematem Hornera sumę częściową stopnia $n$.

\section{Dobór parametrów}
W przedziale $(-\frac{\ln 2}{2}, \frac{\ln 2}{2})$ funkcja eksponencjalna
przyjmuje wartości $(\frac{1}{2}\sqrt{2}, \sqrt{2})$.
Dla wartości tak blisko $1$ błędy względne i bezwzględne będą zbliżone, więc
teraz będziemy się zajmować tylko błędami bezwzględnymi.

W naszej dziedzinie wyrazy szeregu Taylora szybko maleją, więc błąd bezwzględny
obliczonej sumy częściowej powinien być w przybliżeniu rzędu pierwszego
niedodanego wyrazu.
Naturalnie nie chcielibyśmy uwzględniać wyrazów mniejszych niż różnica
między wynikiem a sąsiednimi liczbami zmiennoprzecinkowymi, bo one niezależnie
od algorytmu będą miały znikomy wpływ na wynik.
W naszym wypadku oznacza to, że teoretycznie chcemy zakończyć sumę częściową na
wyrazach mniejszych niż $1.11\cdot10^{-16}$,\footnote{
	Przypadkowo precyzja arytmetyki \macheps,
	bo liczby w $(\frac{1}{2}\sqrt{2}, 1)$ mają cechę równą $-1$.
	W przedziale wartości $[1, \sqrt{2})$ ta różnica to oczywiście 2\macheps.}
czyli zakładając pesymistycznie $x=\frac{1}{2}\ln 2$, zakończyć na wyrazie z
potęgą $14$.

Z moich eksperymentów w IJulii wynika, że wyliczone powyżej parametry działają
bardzo dobrze (poza \texttt{naiveexp}, którym już nie będę się zajmował) i
często dają wynik dokładnie taki, jak wbudowana funkcja eksponencjalna.
Zmniejszenie $\varepsilon$ daje nieco dokładniejsze wyniki, więc od tej pory
przyjmuję w algorytmach $\varepsilon = 10^{-17}$ lub $n=14$.
Algorytm \texttt{hornerexp} działa szybko i $14$ to rozsądna stała liczba
iteracji, więc nie ma sensu np. wprowadzać heurystyk zmniejszających $n$ dla
małych argumentów.

Doświadczenia na $(-\frac{1}{2}\ln 2, \frac{1}{2} \ln 2)$ nie wykazały różnic
w zachowaniu algorytmów dla liczb dodatnich i ujemnych, więc nie będę korzystał
z $e^{-x} = 1/e^x$, aby sprowadzić problem do liczb jednego znaku.

\section{Działanie na całej osi liczb rzeczywistych}
Korzystając z algorytmu na redukcję dziedziny i obliczanie funkcji
eksponencjalnej na tej zawężonej dziedzinie możemy zaimplementować algorytm
obliczający ją dla dowolnej liczby zmiennoprzecinkowej.
Wiemy, że na zawężonej dziedzinie \texttt{hornerexp} i \texttt{rnaiveexp}
są praktycznie bezbłędne, więc tak naprawdę sprawdzimy efekty błędu
powstałego w algorytmie \texttt{reducerange}.

Na wykresach będę zaznaczał liczbę dokładnych cyfr dziesiętnych wyniku, gdzie
przy braku błędu przyjmuję $16$ dokładnych cyfr, tzn. jeśli $\tilde{x}$ to błąd
względny, to jako liczbę dokładnych cyfr przyjmuję
$$-\log_{10}(\max(\tilde{x}, 10^{-16})).$$
Na osi poziomej będzie zaznaczony zakres argumentów,
a oś pionowa będzie oznaczać liczbę dokładnych cyfr.

\begin{figure}
	\centering
	\resizebox{\linewidth}{!}{\input{p1.pdf_tex}}
	\caption{Wykres błędów na $[-0.3, 0.3]$}
	\label{p1}
\end{figure}

Wykres \ref{p1} przedziału $[-0.3, 0.3]$ pokazuje, że na tym przedziale
błędy są takie same jak dla algorytmu bez redukcji dziedziny (czyli znikome).
\begin{figure}
	\centering
	\resizebox{\linewidth}{!}{\input{p2.pdf_tex}}
	\caption{Wykres błędów na $[0.3, 3]$}
	\label{p2}
\end{figure}


Na wykresie \ref{p2} przedziału $[0.3, 3]$ widać,
że algorytm \texttt{reducerange} już coś robi.
Nadal nie tracimy ani jednej cyfry, ale widać pewien wzór --- błędy są
największe nieco ponad wielokrotnościami $\ln 2$.
Wtedy $u$ jest bardzo małe, czyli powstało z różnicy bliskich liczb.
\begin{figure}
	\centering
	\resizebox{\linewidth}{!}{\input{p3.pdf_tex}}
	\caption{Wykres błędów na $[3, 600]$}
	\label{p3}
\end{figure}


Na wykresie \ref{p3} przedziału $[3, 600]$ widać kształt odwróconego wykresu
funkcji logarytmicznej.
Dla największych argumentów możemy stracić już ponad 4 dokładne cyfry
dziesiętne.
Liczba $e^{710}$ jest już za duża i otrzymujemy $\mathtt{Inf}$.

Wykres wygląda symetrycznie względem prostej $x=0$.


\section{Podsumowanie}
Funkcję eksponencjalną można szybko i bardzo dokładnie przybliżyć np. schematem
Hornera dla $|x| < \frac{1}{2}\ln 2$ i dalej korzystając z tego przybliżyć do 13
dokładnych cyfr dziesiętnych dla pozostałych liczb zmiennoprzecinkowych.
Opisany sposób wyboru parametrów algorytmu można zastosować dla arytmetyk
różnych precyzji.



\end{document}
