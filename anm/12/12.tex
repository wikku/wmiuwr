\documentclass[a4paper, 12pt]{article}
\usepackage[utf8]{inputenc}
\usepackage{silence}
\usepackage{polski}
\usepackage{parskip}
\usepackage{amsmath,amsfonts,amssymb,amsthm}
\usepackage{mathtools}
\usepackage{enumitem}
%\usepackage{pgfplots}
%\pgfplotsset{compat=1.16}
\usepackage{newunicodechar}
\usepackage{etoolbox}
\usepackage{breqn}
\usepackage[margin=1.2in]{geometry}
\setcounter{secnumdepth}{0}

\title{}
\author{Wiktor Kuchta}
\date{\vspace{-4ex}}

\DeclareMathOperator{\rank}{rank}
\DeclareMathOperator{\Lin}{Lin}
\DeclareMathOperator{\sgn}{sgn}
\newcommand{\N}{\mathbb{N}}
\newcommand{\Z}{\mathbb{Z}}
\newcommand{\Q}{\mathbb{Q}}
\newcommand{\R}{\mathbb{R}}
\newcommand{\C}{\mathbb{C}}
\newcommand{\inner}[2]{( #1 \, | \, #2)}
\newcommand{\norm}[1]{\left\lVert #1 \right\rVert}
\newcommand{\modulus}[1]{\left| #1 \right|}
\newcommand{\abs}{\modulus}
\newtheorem{theorem}{Twierdzenie}
\newtheorem{lemat}{Lemat}
\newcommand{\ol}{\overline}
\DeclareMathOperator{\tr}{tr}
\DeclareMathOperator{\diag}{diag}
\newcommand{\+}{\enspace}
\newcommand{\sump}{\sideset{}{'}{∑}} % sum prime
\newcommand{\sumb}{\sideset{}{"}{∑}} % sum bis

\newunicodechar{∅}{\emptyset} % Digr /0
\newunicodechar{∞}{\infty} % Digr 00
\newunicodechar{∂}{\partial} % Digr dP
\newunicodechar{α}{\alpha}
\newunicodechar{β}{\beta}
\newunicodechar{ξ}{\xi} % Digr c*
\newunicodechar{δ}{\delta} % Digr d*
\newunicodechar{ε}{\varepsilon}
\newunicodechar{φ}{\varphi}
\newunicodechar{θ}{\theta} % Digr h*
\newunicodechar{λ}{\lambda}
\newunicodechar{μ}{\mu}
\newunicodechar{π}{\pi}
\newunicodechar{σ}{\sigma}
\newunicodechar{τ}{\tau}
\newunicodechar{ω}{\omega}
\newunicodechar{η}{\eta} % Digr y*
\newunicodechar{Δ}{\Delta}
\newunicodechar{Π}{\Pi}
\newunicodechar{ℕ}{\N} % Digr NN 8469 nonstandard
\newunicodechar{ℤ}{\Z} % Digr ZZ 8484 nonstandard
\newunicodechar{ℚ}{\Q} % Digr QQ 8474 nonstandard
\newunicodechar{ℝ}{\R} % Digr RR 8477 nonstandard
\newunicodechar{ℂ}{\C} % Digr CC 8450 nonstandard
\newunicodechar{∑}{\sum}
\newunicodechar{∏}{\prod}
\newunicodechar{∫}{\int}
\newunicodechar{∓}{\mp}
\newunicodechar{⌈}{\lceil} % Digr <7
\newunicodechar{⌉}{\rceil} % Digr >7
\newunicodechar{⌊}{\lfloor} % Digr 7<
\newunicodechar{⌋}{\rfloor} % Digr 7>
\newunicodechar{◁}{\triangleleft}
\newunicodechar{▷}{\triangleright}
\newunicodechar{≤}{\le}
\newunicodechar{≥}{\ge}
\newunicodechar{≠}{\ne}
\newunicodechar{⊆}{\subseteq} % Digr (_
\newunicodechar{⊇}{\supseteq} % Digr _)
\newunicodechar{⊂}{\subset} % Digr (C
\newunicodechar{⊃}{\supset} % Digr C)
\newunicodechar{∩}{\cap} % Digr (U
\newunicodechar{∪}{\cup} % Digr )U
\newunicodechar{∼}{\sim} % Digr ?1
\newunicodechar{∈}{\in} % Digr (-
\newunicodechar{∋}{\ni} % Digr -)
\newunicodechar{∇}{\nabla} % Digr NB
\newunicodechar{∃}{\exists} % Digr TE
\newunicodechar{∀}{\forall} % Digr FA
\newunicodechar{∧}{\wedge} % Digr AN
\newunicodechar{∨}{\vee} % Digr AN
\newunicodechar{⇒}{\implies} % Digr =>

% cursed
\WarningFilter{newunicodechar}{Redefining Unicode}
\newunicodechar{·}{\ifmmode\cdot\else\textperiodcentered\fi} % Digr .M
\newunicodechar{×}{\ifmmode\times\else\texttimes\fi} % Digr *X
\newunicodechar{→}{\ifmmode\to\else\textrightarrow\fi} % Digr ->
\newunicodechar{⟨}{\ifmmode\langle\else\textlangle\fi} % Digr LA 10216 nonstandard
\newunicodechar{⟩}{\ifmmode\rangle\else\textrangle\fi} % Digr RA 10217 nonstandard
\newunicodechar{…}{\ifmmode\dots\else\textellipsis\fi} % Digr .,
\newunicodechar{±}{\ifmmode\pm\else\textpm\fi} % Digr +-

% https://tex.stackexchange.com/a/438184
\ExplSyntaxOn
\AtBeginDocument
 {
  \char_set_active_eq:nN { `: } \colon
  \mathcode`:="8000
 }
\ExplSyntaxOff
% colon: for types; ratio∶ (digr :R) for relations (set builder)



\begin{document}

\maketitle

\section*{M12.2}
Mamy daną dziedzinę $[a,b]$, funkcję wagową $p$.
Chcemy znaleźć $A_0, x_0$ takie,
że dla dowolnego wielomianu $w$ stopnia $<2$ zachodzi
$$∫_a^b p(x)w(x) \, dx = A_0 w(x_0).$$

Wiemy, że kwadratura Gaussa $Q_0$ ma rząd $2$, a więc spełnia nasze wymagania.
Musimy rozważyć ciąg standardowych wielomianów ortogonalnych $P_i$
prostopadłych w sensie wagi $p$.
Mamy $P_0(x) = x, P_1(x) = \left(x - \frac{⟨x P_0, P_0⟩}{⟨P_0, P_0⟩}\right)P_0(x)$.
Zatem
$$x_0 = \frac{∫_a^b x p(x)\,dx}{∫_a^b p(x)\,dx},$$
$$A_0 = ∫_a^b p(x) λ_0(x) \, dx = ∫_a^b p(x) \, dx.$$

\section*{M12.3}
Kwadratura Gaussa $Q_1$ jest rzędu $4$ i żądanej postaci,
więc spełnia nasze wymagania.

Musimy znaleźć wielomiany ortogonalne dla wagi $(1+x^2)$:
\begin{align*}
	P_0(x) &= 1 \\
	P_1(x) &= x - \frac{⟨x P_0, P_0⟩}{⟨P_0, P_0⟩} = x - \frac{∫_0^1 x + x^3\,dx}{∫_0^1 1 + x^2 \, dx} =  x-\frac{9}{16} \\
	P_2(x) &= \left(x - \frac{⟨x P_1, P_1⟩}{⟨P_1, P_1⟩}\right)P_1(x) - \frac{⟨P_1,P_1⟩}{⟨P_0,P_0⟩} \\
	&= \left(x - \frac{∫_0^1 (x+x^3)\left(x- \frac{9}{16}\right)^2\,dx}{∫_0^1
	(1+x^2)\left(x-
	\frac{9}{16}\right)^2\,dx}\right)\left(x-\frac{9}{16}\right) - \frac{∫_0^1
	(1+x^2)\left(x- \frac{9}{16}\right)^2\,dx}{∫_0^1 1 + x^2 \, dx} \\
	&= \left(x - \frac{829}{1712}\right)\left(x-\frac{9}{16}\right) - \frac{107}{1280} \\
	&= x^2 - \frac{112 x}{107} + \frac{101}{535} \\
	&= \left(x - \frac{56}{107} + \frac{\sqrt{4873/5}}{107}\right)\left(x - \frac{56}{107} - \frac{\sqrt{4873/5}}{107}\right).
\end{align*}

\section*{M12.4}
Niech $⟨f, g⟩ = ∫_a^b p(x) f(x) g(x) \, dx$ i $P_i$ to standardowe wielomiany
ortogonalne w sensie normy średniokwadratowej z tą wagą.

Niech $w_n$ to wielomian moniczny stopnia $n$.
Możemy go zapisać w postaci $P_n + r$,
gdzie $r$ jest stopnia $<n$.
Wtedy
\begin{align*}
&∫_a^b p(x) (w_n(x))^2 \, dx \\
&=∫_a^b p(x) (P_n(x))^2 \, dx +
∫_a^b p(x) (r(x))^2 \, dx +
2∫_a^b p(x) r(x) P_n(x) \, dx,
\end{align*}
gdzie ostatni składnik to $2⟨r, P_n⟩ = 0$.
Zakładamy, że waga $p$ jest dodatnia,
więc pierwszy składnik jest stały,
a aby zminimalizować całość musimy przyjąć $r = 0$.

\section*{M12.7}
Dla równania postaci $y'(t) = f(t, y(t))$ jawną metodą Eulera jest
$$y_{n+1} = y_n + h f_n,$$
a niejawną metodą Eulera jest
$$y_{n+1} = y_n + h f_{n+1},$$
gdzie $f_k = f(t_k, y_k), t_k = t_0 + kh, h>0$.

Rozważmy problem
$$y'(t) = λy(t) \+ (t > 0), \quad y(0) = 1,$$
gdzie $λ < 0$, $t_0 = 0$.
Mamy $f(t,y(t)) = λy(t)$.

Jawna metoda Eulera daje nam
$$y_{n+1} = (1+hλ)y_n,$$
więc $y_n → 0$ jeśli $\frac{-2}{λ} > h > 0$.

Niejawna metoda Eulera daje nam
$$y_{n+1} = y_n + hλy_{n+1} = \frac{y_n}{1-hλ}.$$
Aby $y_n → 0$ musimy mieć $\abs{\frac{1}{1-hλ}} < 1$, a więc zawsze.

\section*{M12.8}
Rozważmy problem
$$y'(t) = λy(t) \+ (t > 0), \quad y(0) = 1,$$
gdzie $λ < 0$, $t_0 = 0$.
Mamy $f(t,y(t)) = λy(t)$.

Metoda Cranka-Nicolson ma wzór
$$y_{n+1} = y_n + \frac{h}{2}\left(f_n + f_{n+1}\right).$$
Stosując go do naszego problemu otrzymujemy
\begin{align*}
	y_{n+1}
	= y_n + \frac{h}{2}\left(λy_n + λy_{n+1}\right)
	= \frac{1 + \frac{hλ}{2}}{1 - \frac{hλ}{2}}y_n
	= \frac{2 + hλ}{2 - hλ}y_n.
\end{align*}

\section*{M12.9}
Rozważmy problem
$$y'(t) = λy(t) \+ (t > 0), \quad y(0) = 1,$$
gdzie $λ < 0$, $t_0 = 0$.
Mamy $f(t,y) = λy$.

Metoda Heuna ma wzór
$$y_{n+1} = y_n + \frac{h}{2}\Big(f_n + f(t_{n+1}, y_n + hf_n)\Big).$$
Stosując go do naszego problemu otrzymujemy
\begin{align*}
	y_{n+1}
	= y_n + \frac{h}{2}\left(λy_n + λy_n + λ^2hy_n\right)
	= \left(1 + hλ + \frac{λ^2 h^2}{2}\right)y_n.
\end{align*}


\end{document}
