%! TEX TS-program = xelatex
%! TEX program = xelatex

\documentclass[a4paper, 12pt]{article}
%\usepackage[utf8]{inputenc}
\usepackage{silence}
\usepackage{polski}
\usepackage{parskip}
\usepackage{amsmath,amsfonts,amssymb,amsthm}
\usepackage{mathtools}
\usepackage{enumitem}
%\usepackage{pgfplots}
%\pgfplotsset{compat=1.16}
\usepackage{newunicodechar}
\usepackage{etoolbox}
\usepackage{breqn}
\usepackage[margin=1.2in]{geometry}
\setcounter{secnumdepth}{0}

\title{}
\author{Wiktor Kuchta}
\date{\vspace{-4ex}}

\DeclareMathOperator{\rank}{rank}
\DeclareMathOperator{\Lin}{Lin}
\DeclareMathOperator{\sgn}{sgn}
\newcommand{\N}{\mathbb{N}}
\newcommand{\Z}{\mathbb{Z}}
\newcommand{\Q}{\mathbb{Q}}
\newcommand{\R}{\mathbb{R}}
\newcommand{\C}{\mathbb{C}}
\newcommand{\inner}[2]{( #1 \, | \, #2)}
\newcommand{\norm}[1]{\left\lVert #1 \right\rVert}
\newcommand{\modulus}[1]{\left| #1 \right|}
\newcommand{\abs}{\modulus}
\newtheorem{theorem}{Twierdzenie}
\newtheorem{lemat}{Lemat}
\newcommand{\ol}{\overline}
\DeclareMathOperator{\tr}{tr}
\DeclareMathOperator{\diag}{diag}
\newcommand{\+}{\enspace}
\newcommand{\sump}{\sideset{}{'}{\sum}} % sum prime
\newcommand{\sumb}{\sideset{}{"}{\sum}} % sum bis
%\newcommand{∑}{\sump}


\newunicodechar{∅}{\emptyset} % Digr /0
\newunicodechar{∞}{\infty} % Digr 00
\newunicodechar{∂}{\partial} % Digr dP
\newunicodechar{α}{\alpha}
\newunicodechar{β}{\beta}
\newunicodechar{ξ}{\xi} % Digr c*
\newunicodechar{δ}{\delta} % Digr d*
\newunicodechar{ε}{\varepsilon}
\newunicodechar{φ}{\varphi}
\newunicodechar{θ}{\theta} % Digr h*
\newunicodechar{μ}{\mu}
\newunicodechar{π}{\pi}
\newunicodechar{σ}{\sigma}
\newunicodechar{τ}{\tau}
\newunicodechar{ω}{\omega}
\newunicodechar{η}{\eta} % Digr y*
\newunicodechar{Δ}{\Delta}
\newunicodechar{Π}{\Pi}
\newunicodechar{ℕ}{\N} % Digr NN 8469 nonstandard
\newunicodechar{ℤ}{\Z} % Digr ZZ 8484 nonstandard
\newunicodechar{ℚ}{\Q} % Digr QQ 8474 nonstandard
\newunicodechar{ℝ}{\R} % Digr RR 8477 nonstandard
\newunicodechar{ℂ}{\C} % Digr CC 8450 nonstandard
\newunicodechar{∑}{\sum}
\newunicodechar{∏}{\prod}
\newunicodechar{∫}{\int}
\newunicodechar{∓}{\mp}
\newunicodechar{⌈}{\lceil} % Digr <7
\newunicodechar{⌉}{\rceil} % Digr >7
\newunicodechar{⌊}{\lfloor} % Digr 7<
\newunicodechar{⌋}{\rfloor} % Digr 7>
\newunicodechar{◁}{\triangleleft}
\newunicodechar{▷}{\triangleright}
\newunicodechar{≤}{\le}
\newunicodechar{≥}{\ge}
\newunicodechar{≠}{\ne}
\newunicodechar{⊆}{\subseteq} % Digr (_
\newunicodechar{⊇}{\supseteq} % Digr _)
\newunicodechar{⊂}{\subset} % Digr (C
\newunicodechar{⊃}{\supset} % Digr C)
\newunicodechar{∩}{\cap} % Digr (U
\newunicodechar{∪}{\cup} % Digr )U
\newunicodechar{∼}{\sim} % Digr ?1
\newunicodechar{∈}{\in} % Digr (-
\newunicodechar{∋}{\ni} % Digr -)
\newunicodechar{∇}{\nabla} % Digr NB
\newunicodechar{∃}{\exists} % Digr TE
\newunicodechar{∀}{\forall} % Digr FA
\newunicodechar{∧}{\wedge} % Digr AN
\newunicodechar{∨}{\vee} % Digr AN
\newunicodechar{⇒}{\implies} % Digr =>

% cursed
\WarningFilter{newunicodechar}{Redefining Unicode}
\newunicodechar{·}{\ifmmode\cdot\else\textperiodcentered\fi} % Digr .M
\newunicodechar{×}{\ifmmode\times\else\texttimes\fi} % Digr *X
\newunicodechar{→}{\ifmmode\to\else\textrightarrow\fi} % Digr ->
\newunicodechar{⟨}{\ifmmode\langle\else\textlangle\fi} % Digr LA 10216 nonstandard
\newunicodechar{⟩}{\ifmmode\rangle\else\textrangle\fi} % Digr RA 10217 nonstandard
\newunicodechar{…}{\ifmmode\dots\else\textellipsis\fi} % Digr .,
\newunicodechar{±}{\ifmmode\pm\else\textpm\fi} % Digr +-

% https://tex.stackexchange.com/a/438184
\ExplSyntaxOn
\AtBeginDocument
 {
  \char_set_active_eq:nN { `: } \colon
  \mathcode`:="8000
 }
\ExplSyntaxOff
% colon: for types; ratio∶ (digr :R) for relations (set builder)



\begin{document}


\maketitle

\section*{M11.1}

Wielomiany Czebyszewa spełniają równość $T_n(\cos θ) = \cos (nθ)$,
a ekstrema wielomianu $T_n$ są w punktach
$u_k = \cos \left(\frac{kπ}{n}\right)$.
Zatem

\begin{align*}
T_{n+j}(u_k)
&= T_{n+j} \left(\cos \left(\frac{kπ}{n}\right)\right) \\
&= \cos \left(\frac{(n+j)kπ}{n}\right) \\
&= \cos \left(kπ+\frac{jkπ}{n}\right) \\
&= (-1)^k · \cos \frac{jkπ}{n} \\
&= T_n(u_k) · T_j(u_k).
\end{align*}

\section*{M11.2}

Niech $ε > 0$.
Oznaczmy $x_{ni}$ zera $n+1$-szego wielomianu ortogonalnego z wagą $p$,
a $A_{ni}$ to odpowiednie współczynniki z kwadratury Gaussa.

Z tw. Weierstrassa istnieje $w$ taki, że $f-w < ε$ na $[a,b]$.  % p waga
Dla $n > \deg w$ wzór Gaussa jest dokładny, więc mamy
\begin{align*}
\abs{G_n(f) - I(f)}
&≤ \abs{∫_a^b p(x) f(x)\,dx - ∫_a^b p(x) w(x) \, dx} \\
&+ \abs{∑_{i=0}^n A_{ni}w(x_{ni}) - ∑_{i=0}^n A_{ni}f(x_{ni})} \\
&≤ ∫_a^b p(x)\abs{f(x) - w(x)}\,dx + ∑_{i=0}^n A_{ni}\abs{w(x_{ni})- f(x_{ni})} \\
&≤ ε ∫_a^b p(x)\,dx + ε∑_{i=0}^n A_{ni} = 2ε ∫_a^b w(x) \, dx.
\end{align*}


\section*{M11.3}
Błąd złożonej metody Simpsona dla $∫_0^{π} \sin x\,dx$ możemy oszacować z góry przez
$$\frac{h^4}{180}(π-0)\sin^{(4)}(ξ) ≤ \frac{π^5}{180 n^4}.$$
Zatem jeśli $n≥18$, to otrzymamy błąd $≤ 2·10^{-5}$.
Można eksperymentalnie sprawdzić, że wystarczy $n=16$.

Błąd złożonej metody trapezów dla $∫_0^{π} \sin x\,dx$ możemy oszacować z góry przez
$$\abs{-\frac{π^3}{12n^2}\sin^{(2)}(ξ)} ≤ \frac{π^3}{12n^2}.$$
Zatem jeśli $n≥360$, to otrzymamy błąd $≤ 2·10^{-5}$.
Można eksperymentalnie sprawdzić, że wystarczy $n=287$.

\section*{M11.4}
$$w_n(x) = \sump_{k=0}^n a_k T_k(x)$$

\begin{align*}
∫_{-1}^{1} w_n(x) \, dx
&= \sump_{i=0}^n a_i ∫_{-1}^1 T_i(x) \, dx \\
&= \sump_{i=0}^{⌊ n/2 ⌋} a_{2i} ∫_{-1}^1 T_{2i}(x) \, dx \\
&= 2 \sump_{i=0}^{⌊ n/2 ⌋} \frac{a_{2i}}{1-4i^2} & \text{M10.1}
\end{align*}


\section*{M11.5}

$$e^{e^{e^x}}$$

\section*{M11.7}
Niech $w_n(x) = ∑_{j=0}^n c_j \cos(jx)$.
Wtedy
$$v_n(t) = w_n(\arccos t) = ∑_{j=0}^n c_j \cos(j \arccos t) = ∑_{j=0}^n c_j T_j(t).$$

Wielomiany Czebyszewa $T_n$ tworzą ciąg ortogonalny dla iloczynu skalarnego
\begin{align*}
⟨f,g⟩
= ∫_{-1}^1 \frac{1}{\sqrt{1-x^2}} f(x) g(x) \, dx
= ∫_0^{π} f(\cos t) g(\cos t) \, dt,
\end{align*}

zatem zminimalizowanie całki
$$∫_0^{π} \left(π - x^2 - w_n(x)\right)^2\,dx
=∫_0^{π} \left(π - (\arccos (\cos x))^2 - v_n(\cos x)\right)^2\,dx$$
sprowadza się do znalezienia wielomianu optymalnego dla $f(t) = π - (\arccos t)^2$.
Jest to po prostu rzut ortogonalny $f$ na $Π_n$, więc
$$c_j = \frac{⟨f, T_j⟩}{⟨T_j, T_j⟩}.$$




\end{document}
