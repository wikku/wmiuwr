\documentclass[a4paper, 12pt]{article}
\usepackage[utf8]{inputenc}
\usepackage{silence}
\usepackage{polski}
\usepackage{parskip}
\usepackage{amsmath,amsfonts,amssymb,amsthm}
\usepackage{mathtools}
\usepackage{enumitem}
%\usepackage{pgfplots}
%\pgfplotsset{compat=1.16}
\usepackage{newunicodechar}
\usepackage{etoolbox}
\usepackage[margin=1.2in]{geometry}
\setcounter{secnumdepth}{0}

\title{}
\author{Wiktor Kuchta}
\date{\vspace{-4ex}}

\DeclareMathOperator{\rank}{rank}
\DeclareMathOperator{\Lin}{Lin}
\DeclareMathOperator{\sgn}{sgn}
\newcommand{\N}{\mathbb{N}}
\newcommand{\Z}{\mathbb{Z}}
\newcommand{\Q}{\mathbb{Q}}
\newcommand{\R}{\mathbb{R}}
\newcommand{\C}{\mathbb{C}}
\newcommand{\inner}[2]{( #1 \, | \, #2)}
\newcommand{\norm}[1]{\left\lVert #1 \right\rVert}
\newcommand{\modulus}[1]{\left| #1 \right|}
\newcommand{\abs}{\modulus}
\newtheorem{theorem}{Twierdzenie}
\newtheorem{lemat}{Lemat}
\newcommand{\ol}{\overline}
\DeclareMathOperator{\tr}{tr}
\DeclareMathOperator{\diag}{diag}
\newcommand{\+}{\enspace}
\newcommand{\sump}{\sideset{}{'}{∑}} % sum prime
\newcommand{\sumb}{\sideset{}{"}{∑}} % sum bis

\newunicodechar{∅}{\emptyset} % Digr /0
\newunicodechar{∞}{\infty} % Digr 00
\newunicodechar{∂}{\partial} % Digr dP
\newunicodechar{α}{\alpha}
\newunicodechar{β}{\beta}
\newunicodechar{ξ}{\xi} % Digr c*
\newunicodechar{δ}{\delta} % Digr d*
\newunicodechar{ε}{\varepsilon}
\newunicodechar{φ}{\varphi}
\newunicodechar{θ}{\theta} % Digr h*
\newunicodechar{λ}{\lambda}
\newunicodechar{μ}{\mu}
\newunicodechar{π}{\pi}
\newunicodechar{σ}{\sigma}
\newunicodechar{τ}{\tau}
\newunicodechar{ω}{\omega}
\newunicodechar{η}{\eta} % Digr y*
\newunicodechar{Δ}{\Delta}
\newunicodechar{Φ}{\Phi} % Digr F*
\newunicodechar{Π}{\Pi}
\newunicodechar{Ψ}{\Psi} % digr Q*
\newunicodechar{ℕ}{\N} % Digr NN 8469 nonstandard
\newunicodechar{ℤ}{\Z} % Digr ZZ 8484 nonstandard
\newunicodechar{ℚ}{\Q} % Digr QQ 8474 nonstandard
\newunicodechar{ℝ}{\R} % Digr RR 8477 nonstandard
\newunicodechar{ℂ}{\C} % Digr CC 8450 nonstandard
\newunicodechar{∑}{\sum}
\newunicodechar{∏}{\prod}
\newunicodechar{∫}{\int}
\newunicodechar{∓}{\mp}
\newunicodechar{⌈}{\lceil} % Digr <7
\newunicodechar{⌉}{\rceil} % Digr >7
\newunicodechar{⌊}{\lfloor} % Digr 7<
\newunicodechar{⌋}{\rfloor} % Digr 7>
\newunicodechar{◁}{\triangleleft} % Digr Tl
\newunicodechar{▷}{\triangleright} % Digr Tr
\newunicodechar{≤}{\le}
\newunicodechar{≥}{\ge}
\newunicodechar{≪}{\ll} % Digr <*
\newunicodechar{≫}{\gg} % Digr *>
\newunicodechar{≠}{\ne}
\newunicodechar{⊆}{\subseteq} % Digr (_
\newunicodechar{⊇}{\supseteq} % Digr _)
\newunicodechar{⊂}{\subset} % Digr (C
\newunicodechar{⊃}{\supset} % Digr C)
\newunicodechar{∩}{\cap} % Digr (U
\newunicodechar{∪}{\cup} % Digr )U
\newunicodechar{∼}{\sim} % Digr ?1
\newunicodechar{∈}{\in} % Digr (-
\newunicodechar{∋}{\ni} % Digr -)
\newunicodechar{∇}{\nabla} % Digr NB
\newunicodechar{∃}{\exists} % Digr TE
\newunicodechar{∀}{\forall} % Digr FA
\newunicodechar{∧}{\wedge} % Digr AN
\newunicodechar{∨}{\vee} % Digr OR
\newunicodechar{⊥}{\bot} % Digr -T
\newunicodechar{⊤}{\top} % Digr TO 8868 nonstandard
\newunicodechar{⇒}{\implies} % Digr =>

% cursed
\WarningFilter{newunicodechar}{Redefining Unicode}
\newunicodechar{·}{\ifmmode\cdot\else\textperiodcentered\fi} % Digr .M
\newunicodechar{×}{\ifmmode\times\else\texttimes\fi} % Digr *X
\newunicodechar{→}{\ifmmode\to\else\textrightarrow\fi} % Digr ->
\newunicodechar{⟨}{\ifmmode\langle\else\textlangle\fi} % Digr LA 10216 nonstandard
\newunicodechar{⟩}{\ifmmode\rangle\else\textrangle\fi} % Digr RA 10217 nonstandard
\newunicodechar{…}{\ifmmode\dots\else\textellipsis\fi} % Digr .,
\newunicodechar{±}{\ifmmode\pm\else\textpm\fi} % Digr +-

% https://tex.stackexchange.com/a/438184
% https://tex.stackexchange.com/q/528480
\newunicodechar{∶}{\mathbin{\text{:}}}
\def\newcolon{%
  \nobreak\mskip2mu\mathpunct{}\nonscript\mkern-\thinmuskip{\text{:}}%
  \mskip 6mu plus 1 mu \relax}
\mathcode`:="8000
{\catcode`:=\active \global\let:\newcolon}
% colon: for types; ratio∶ (digr :R) for relations (set builder)


\begin{document}

\maketitle

\section*{M14.1}
Z definicji normy indukowanej
$$
\norm{A}_p =
\sup_{\mathbf{x} ≠ \mathbf{0}}
    \frac{\norm{A\mathbf{x}}_p}{\norm{\mathbf{x}}_p}
$$
mamy wprost zgodność normy z normą indukowaną:
$$\norm{A\mathbf{x}}_p ≤ \norm{A}_p \norm{\mathbf{x}}_p.$$

Korzystając z definicji i łączności mnożenia macierzy otrzymujemy tezę:
$$
\norm{AB}_p
= \sup_{\mathbf{x} ≠ \mathbf{0}}
    \frac{\norm{AB\mathbf{x}}_p}{\norm{\mathbf{x}}_p}
≤ \sup_{\mathbf{x} ≠ \mathbf{0}}
\frac{\norm{A}_p\norm{B\mathbf{x}}_p}{\norm{\mathbf{x}}_p}
≤ \sup_{\mathbf{x} ≠ \mathbf{0}}
\frac{\norm{A}_p\norm{B}_p\norm{\mathbf{x}}_p}{\norm{\mathbf{x}}_p}
= \norm{A}_p\norm{B}_p.
$$

\section*{M14.2}
$$
B=
\begin{bmatrix}
	1 & -1 & -1 & … & -1 \\
	  &  1 & -1 & … & -1 \\
	  & & \ddots&&\vdots \\
	  &    &    &  1&  -1 \\
	  &    &    &   &  1
\end{bmatrix}
$$

Macierz $B$ jest trójkątna i ma jedynki na przekątnej,
więc $\det B = 1$.
Niech
$$c_j = [
	2^{j-2} ,
	2^{j-3} ,
	\dots  ,
	2       ,
	1       ,
	1       ,
	0       ,
	\dots  ,
	0
	]^⊤,
$$
tzn. ostatnia jedynka jest na $j$-tej pozycji.
Mamy $B c_j = e_j$,
więc macierz $C$ złożona z kolumn $c_j$ jest odwrotnością $B$.
Jest ona także górnotrójkątna i $\det C = 1$.

Norma $\norm{\;·\;}_∞$ macierzy to jej maksymalna suma modułów w wierszu,
zatem $\norm{B}_∞ = n$, a $\norm{C}_∞ = 2^{n-1}$.
Mamy $\mathrm{cond}_∞(B) = \norm{B}_∞\norm{B^{-1}}_∞ = n 2^{n-1} ≫ 1 = \det B$,
co oznacza,
że przy rozwiązywaniu układu równań tej macierzy może dojść do dużych błędów.

\section*{M14.4}
Mamy
$A\mathbf{x} = \mathbf{b}$
i
$A\tilde{\mathbf{x}} = \mathbf{b}-\mathbf{r}$, więc % r — residuum
$\mathbf{x}-\tilde{\mathbf{x}} = A^{-1}\mathbf{r}$.
Nakładając obustronnie normę i korzystając ze zgodności norm otrzymujemy
$$\norm{\mathbf{x}-\tilde{\mathbf{x}}}
= \norm{A^{-1}\mathbf{r}}
≤ \norm{A^{-1}}\norm{\mathbf{r}}
= \norm{A}\norm{A^{-1}}\frac{\norm{\mathbf{r}}}{\norm{A}}
= \mathrm{cond}(A)\frac{\norm{\mathbf{r}}}{\norm{A}}.$$
Dalej $\norm{\mathbf{b}} = \norm{A\mathbf{x}} ≤ \norm{A} \norm{\mathbf{x}}$,
więc dzieląc obustronnie przez $\norm{\mathbf{x}}$ otrzymujemy
$$\frac{\norm{\mathbf{x}-\tilde{\mathbf{x}}}}{\norm{\mathbf{x}}}
≤ \mathrm{cond}(A)\frac{\norm{\mathbf{r}}}{\norm{A}\norm{\mathbf{x}}}
≤ \mathrm{cond}(A)\frac{\norm{\mathbf{r}}}{\norm{\mathbf{b}}}.$$

\section*{M14.6}
Załóżmy, że wszystkie wartości własne $λ_i$ macierzy $A ∈ ℝ^{n × n}$ są
rzeczywiste i spełniają nierówności
$$0 < α ≤ λ_i ≤ β \quad (i=1,2,…,n).$$
Rozpatrzmy metodę iteracyjną Richardsona
$$\mathbf{x}^{(x+1)} = (I-τA)\mathbf{x}^{(k)} + τ\mathbf{b} \qquad (k ≥ 0)$$
zastosowaną do rozwiązywania układu $A\mathbf{x}=\mathbf{b}$,
gdzie $0 < τ < 2/β$.

Zauważmy, że macierz $I-τA$ ma wektory własne $A$,
ale z wartościami własnymi $1 - τλ_i$.
Zatem są one niewiększe niż $1 - τα$ i niemniejsze niż $1 - τβ$.
Mamy $1- τα < 1$ z dodatniości $τ$ i $α$.
Mamy $τβ < 2$, więc $1- τβ > -1$.
Zatem wartości własne $I-τA$ są co do modułu mniejsze niż $1$
i iteracja jest zbieżna.

\section*{M14.7}
Załóżmy, że $A$ jest macierzą ze ściśle dominującą przekątną wierszowo.
Metoda Jacobiego rozpatruje podział $A$ na jej przekątną i macierze pod
i nad przekątną:
$A = L + D + U$.
Następnie przeprowadzamy iterację o wzorze
$$ \mathbf{x}^{(x+1)} = - D^{-1}(L + U)\mathbf{x}^{(k)} - D^{-1} \mathbf{b}, $$
tzn. macierz przekształcenia tej metody to $B_J = -D^{-1}(L+U)$.

Macierz $D$ się składa z odwróconych wyrazów na przekątnej macierzy $A$.
Mnożąc ją z prawej przez $(L+U)$ otrzymujemy macierz $(L+U)$,
w której każdy wiersz został podzielony przez odpowienią wartość na przekątnej
$A$.
Skoro $(L+U)$ ma zera na przekątnej i $\abs{a_{ii}} > ∑_{j=1, \, j≠i}^n \abs{a_{ij}}$,
to $\norm{B_J}_∞ = \norm{L+U}_∞ < 1$.

Do zbieżności metody wystarczy mieć jakąkolwiek normę macierzy iteracji
mniejszą niż $1$, zatem w naszym przypadku metoda jest zbieżna.

\section*{M14.8}
Załóżmy, że $A$ jest macierzą ze ściśle dominującą przekątną wierszowo.
Metoda Gaussa-Seidela rozpatruje podział $A$ na macierz dolnotrójkątną i ściśle
górnotrójkątną tzn.  $A = L_* + U$.
Następnie przeprowadzamy iterację o wzorze
$$ \mathbf{x}^{(x+1)} = -L_*^{-1}U\mathbf{x}^{(n)} - L_*^{-1} \mathbf{b}.$$

Rozpatrzmy wartości własne macierzy $B_S$:
\begin{align*}
	-L_*^{-1}U \mathbf{x} &= λ \mathbf{x}, \\
	-U \mathbf{x} &= λ L_* \mathbf{x}.
\end{align*}
Załóżmy, że $\norm{\mathbf{x}}_∞ = 1$, $\abs{x_i} = 1$.
Skupiając się na $i$-tym wierszu otrzymujemy
\begin{align*}
	-∑_{j=i+1}^n a_{ij} x_j &= λ∑_{j=1}^i a_{ij}x_j \\
	λ a_{ii} x_i &= -∑_{j=i+1}^n a_{ij} x_j - λ∑_{j=1}^{i-1} a_{ij} x_j \\
	\abs{λ} \abs{a_{ii}} &≤ ∑_{j=i+1}^n \abs{a_{ij}} - \abs{λ}∑_{j=1}^{i-1} \abs{a_{ij}} \\
	\abs{λ} &≤
		\frac{∑_{j=i+1}^n \abs{a_{ij}}}
		     {\abs{a_{ii}} - ∑_{j=1}^{i-1} \abs{a_{ij}}}.
\end{align*}
Z dominacji przekątniowej mamy
\begin{align*}
	\abs{a_{ii}} &> ∑_{j=1}^{i-1} \abs{a_{ij}} + ∑_{j=i+1}^n \abs{a_{ij}} \\
	\abs{a_{ii}} -∑_{j=1}^{i-1} \abs{a_{ij}} &> ∑_{j=i+1}^n \abs{a_{ij}},
\end{align*}
a zatem $\abs{λ} < 1$.
Wszystkie wartości własne są co do modułu mniejsze niż $1$,
a zatem metoda jest zbieżna.

\section*{M14.9}
Załóżmy, że $A$ jest macierzą ściśle dominującą kolumnowo.
Metoda Jacobiego rozpatruje podział $A$ na jej przekątną i macierze pod
i nad przekątną:
$A = L + D + U$.
Następnie przeprowadzamy iterację o wzorze
$$ \mathbf{x}^{(x+1)} = - D^{-1}(L + U)\mathbf{x}^{(k)} - D^{-1} \mathbf{b}, $$
tzn. macierz przekształcenia tej metody to $B_J = -D^{-1}(L+U)$.

Rozpatrzmy normę $\norm{C}_N = \norm{D C D^{-1}}_1$.
Wtedy $\norm{B_J}_N = \norm{(L+U)D^{-1}}_1$.
Mnożenie z prawej przez macierz diagonalną mnoży kolumny macierzy po lewej przez
odpowiednie wyrazy na przekątnej.
To oznacza, że z dominacji przekątniowej dla $C = (L+U)D^{-1}$ mamy
$∑_{i=1}^n \abs{c_{ij}} < 1$.
Norma $\norm{\;·\;}_1$ to maksimum takich sum modułów dla kolumn, zatem
$\norm{B_J}_N < 1$ i iteracja jest zbieżna.








\end{document}
