\documentclass[a4paper, 12pt]{article}
\usepackage[utf8]{inputenc}
\usepackage{silence}
\usepackage{polski}
\usepackage{parskip}
\usepackage{amsfonts,amssymb,amsthm}
\usepackage{mathtools}
\usepackage{enumitem}
%\usepackage{pgfplots}
%\pgfplotsset{compat=1.16}
\usepackage{newunicodechar}
%\usepackage{etoolbox}
\usepackage[margin=1.2in]{geometry}
\setcounter{secnumdepth}{0}

\title{}
\author{Wiktor Kuchta}
\date{\vspace{-4ex}}

\DeclareMathOperator{\rank}{rank}
\DeclareMathOperator{\Lin}{Lin}
\DeclareMathOperator{\sgn}{sgn}
\newcommand{\N}{\mathbb{N}}
\newcommand{\Z}{\mathbb{Z}}
\newcommand{\Q}{\mathbb{Q}}
\newcommand{\R}{\mathbb{R}}
\newcommand{\C}{\mathbb{C}}
\newcommand{\inner}[2]{( #1 \, | \, #2)}
\newcommand{\norm}[1]{\left\lVert #1 \right\rVert}
\newcommand{\modulus}[1]{\left| #1 \right|}
\newcommand{\abs}{\modulus}
\newtheorem{theorem}{Twierdzenie}
\newtheorem{lemat}{Lemat}
\newcommand{\ol}{\overline}
\DeclareMathOperator{\tr}{tr}
\DeclareMathOperator{\diag}{diag}
\newcommand{\+}{\enspace}
\newcommand{\sump}{\sideset{}{'}{∑}} % sum prime
\newcommand{\sumb}{\sideset{}{"}{∑}} % sum bis

\newunicodechar{∅}{\emptyset} % Digr /0
\newunicodechar{∞}{\infty} % Digr 00
\newunicodechar{∂}{\partial} % Digr dP
\newunicodechar{α}{\alpha}
\newunicodechar{β}{\beta}
\newunicodechar{ξ}{\xi} % Digr c*
\newunicodechar{δ}{\delta} % Digr d*
\newunicodechar{ε}{\varepsilon}
\newunicodechar{φ}{\varphi}
\newunicodechar{θ}{\theta} % Digr h*
\newunicodechar{λ}{\lambda}
\newunicodechar{μ}{\mu}
\newunicodechar{π}{\pi}
\newunicodechar{σ}{\sigma}
\newunicodechar{τ}{\tau}
\newunicodechar{ω}{\omega}
\newunicodechar{η}{\eta} % Digr y*
\newunicodechar{Δ}{\Delta}
\newunicodechar{Φ}{\Phi} % Digr F*
\newunicodechar{Π}{\Pi}
\newunicodechar{Ψ}{\Psi} % digr Q*
\newunicodechar{ℕ}{\N} % Digr NN 8469 nonstandard
\newunicodechar{ℤ}{\Z} % Digr ZZ 8484 nonstandard
\newunicodechar{ℚ}{\Q} % Digr QQ 8474 nonstandard
\newunicodechar{ℝ}{\R} % Digr RR 8477 nonstandard
\newunicodechar{ℂ}{\C} % Digr CC 8450 nonstandard
\newunicodechar{∑}{\sum}
\newunicodechar{∏}{\prod}
\newunicodechar{∫}{\int}
\newunicodechar{∓}{\mp}
\newunicodechar{⌈}{\lceil} % Digr <7
\newunicodechar{⌉}{\rceil} % Digr >7
\newunicodechar{⌊}{\lfloor} % Digr 7<
\newunicodechar{⌋}{\rfloor} % Digr 7>
\newunicodechar{◁}{\triangleleft} % Digr Tl
\newunicodechar{▷}{\triangleright} % Digr Tr
\newunicodechar{≤}{\le}
\newunicodechar{≥}{\ge}
\newunicodechar{≠}{\ne}
\newunicodechar{⊆}{\subseteq} % Digr (_
\newunicodechar{⊇}{\supseteq} % Digr _)
\newunicodechar{⊂}{\subset} % Digr (C
\newunicodechar{⊃}{\supset} % Digr C)
\newunicodechar{∩}{\cap} % Digr (U
\newunicodechar{∪}{\cup} % Digr )U
\newunicodechar{∼}{\sim} % Digr ?1
\newunicodechar{∈}{\in} % Digr (-
\newunicodechar{∋}{\ni} % Digr -)
\newunicodechar{∇}{\nabla} % Digr NB
\newunicodechar{∃}{\exists} % Digr TE
\newunicodechar{∀}{\forall} % Digr FA
\newunicodechar{∧}{\wedge} % Digr AN
\newunicodechar{∨}{\vee} % Digr OR
\newunicodechar{⊥}{\bot} % Digr -T
\newunicodechar{⊤}{\top} % Digr TO 8868 nonstandard
\newunicodechar{⇒}{\implies} % Digr =>

% cursed
\WarningFilter{newunicodechar}{Redefining Unicode}
\newunicodechar{·}{\ifmmode\cdot\else\textperiodcentered\fi} % Digr .M
\newunicodechar{×}{\ifmmode\times\else\texttimes\fi} % Digr *X
\newunicodechar{→}{\ifmmode\to\else\textrightarrow\fi} % Digr ->
\newunicodechar{⟨}{\ifmmode\langle\else\textlangle\fi} % Digr LA 10216 nonstandard
\newunicodechar{⟩}{\ifmmode\rangle\else\textrangle\fi} % Digr RA 10217 nonstandard
\newunicodechar{…}{\ifmmode\dots\else\textellipsis\fi} % Digr .,
\newunicodechar{±}{\ifmmode\pm\else\textpm\fi} % Digr +-


% https://tex.stackexchange.com/a/438184
% https://tex.stackexchange.com/q/528480
\newunicodechar{∶}{\mathbin{\text{:}}}
\def\newcolon{%
  \nobreak\mskip2mu\mathpunct{}\nonscript\mkern-\thinmuskip{\text{:}}%
  \mskip 6mu plus 1 mu \relax}
\mathcode`:="8000
{\catcode`:=\active \global\let:\newcolon}
% colon: for types; ratio∶ (digr :R) for relations (set builder)

\begin{document}

%\maketitle
\section*{M13.1}
\subsection*{(a)}
$$\norm{\mathbf{x}}_1 = ∑_{k=1}^n \abs{x_k}$$
Dodatnia określoność i jednorodność są oczywiste.
Nierówność trójkąta dla norm wynika z nierówności trójkąta dla wartości
bezwzględnej:
\begin{align*}
	\norm{\mathbf{x} + \mathbf{y}}_1
	= ∑_{k=1}^n \abs{x_k + y_k}
	≤ ∑_{k=1}^n \abs{x_k} + ∑_{k=1}^n \abs{y_k}
	= \norm{\mathbf{x}}_1 + \norm{\mathbf{y}}_1.
\end{align*}

\subsection*{(b)}
$$\norm{\mathbf{x}}_∞ = \max_{1 ≤ k ≤ n} \abs{x_k}$$
Dodatnia określoność i jednorodność są oczywiste.
\begin{align*}
	\norm{\mathbf{x} + \mathbf{y}}_∞
	&= \max_{1 ≤ k ≤ n} \abs{x_k + y_k}
	≤ \max_{1 ≤ k ≤ n} \left(\abs{x_k} + \abs{y_k}\right) \\
	&≤ \max_{1 ≤ k ≤ n} \abs{x_k} + \max_{1 ≤ k ≤ n} \abs{y_k}
	= \norm{\mathbf{x}}_∞ + \norm{\mathbf{y}}_∞
\end{align*}

\section*{M13.2}
Mamy pokazać, że
$$\sup_{\mathbf{x} ∈ ℝ^n \setminus \{\mathbf{0}\}}
\frac{\norm{A\mathbf{x}}_2}{\norm{\mathbf{x}}_2}
= \sup_{\norm{\mathbf{x}}_2 = 1} \norm{A\mathbf{x}}_2$$
jest równe pierwiastkowi kwadratowemu największej wartości własnej $A^\top A$.

Macierz $A^\top A$ jest symetryczna i nieujemnie określona,
więc ma układ ortonormalny wektorów własnych $(v_i)$ z nieujemnymi wartościami
własnymi $λ_i$.

Zakładając, że $\mathbf{x} = ∑_{i=1}^n α_i \mathbf{v}_i$ ma normę $1$, mamy
\begin{align*}
\norm{A\mathbf{x}}_2^2
&= \mathbf{x}^⊤ A^⊤ A \mathbf{x}
= \mathbf{x}^⊤ A^⊤ A ∑_{i=1}^n α_i \mathbf{v}_i
= \mathbf{x}^⊤ ∑_{i=1}^n α_i A^⊤ A \mathbf{v}_i
= \left(∑_{i=1}^n α_i \mathbf{v}_i\right)^⊤ ∑_{i=1}^n α_i λ_i \mathbf{v}_i \\
&= ∑_{i=1}^n α_i \mathbf{v}_i^⊤ ∑_{i=j}^n α_j λ_j \mathbf{v}_j
= ∑_{i=1}^n ∑_{i=j}^n α_i a_j λ_j \mathbf{v}_i^⊤ \mathbf{v}_j
= ∑_{i=1}^n α_i^2 λ_i.
\end{align*}
Liczby $α_i^2$ tworzą podział jedynki,
zatem kwadrat normy możemy oszacować z góry przez
$\max_{1 ≤ k ≤ n} λ_k.$
To supremum jest osiągane dla odpowiedniego wektora własnego.


\section*{M13.3}
\subsection*{(a)}
\begin{align*}
\norm{\mathbf{x}}_∞
	&= \max_{1 ≤ k ≤ n} \abs{x_k}
	≤ ∑_{k=1}^n \abs{x_k}
	= \norm{\mathbf{x}}_1
	≤ n\max_{1 ≤ k ≤ n} \abs{x_k}
	≤ n\norm{\mathbf{x}}_∞
\end{align*}

\subsection*{(b)}
$$
	\norm{\mathbf{x}}_∞^2
	= \max_{1 ≤ k ≤ n} x_k^2
	≤ ∑_{k=0}^n x_k^2
	= \norm{\mathbf{x}}_2^2
	≤ n\max_{1 ≤ k ≤ n} x_k^2
	≤ n \norm{\mathbf{x}}_∞^2
$$
$$
\norm{\mathbf{x}}_∞ ≤ \norm{\mathbf{x}}_2 ≤ \sqrt{n}\norm{\mathbf{x}}_∞$$

\subsection*{(c)}
Z nierówności Cauchy'ego-Schwarza
$$
\left[\abs{x_1}, …, \abs{x_n}\right]^\top\left[1, …, 1\right]
≤ \norm{\left[\abs{x_1}, …, \abs{x_n}\right]^\top}_2 \norm{\left[1, …, 1\right]^\top}_2,
$$
czyli
$$
	\norm{\mathbf{x}}_1
	= ∑_{k=1}^n \abs{x_k}
	≤ \sqrt{n}\sqrt{∑_{k=1}^n x_k^2}
	= \sqrt{n}\norm{\mathbf{x}}_2.
$$

$$x_1^2 + … + x_n^2 ≤ \left(\abs{x_1} + … + \abs{x_k}\right)^2,$$
bo po prawej mamy składniki dodatnie i każdy składnik po lewej występuje po
prawej.
Zatem
$$\norm{\mathbf{x}}_2 ≤ \norm{\mathbf{x}}_1.$$

\section*{M13.4}
Mamy pokazać, że
$$\sup_{\mathbf{x} ∈ ℝ^n \setminus \{\mathbf{0}\}}
\frac{\norm{A\mathbf{x}}_∞}{\norm{\mathbf{x}}_∞}
= \sup_{\norm{\mathbf{x}}_∞ = 1} \norm{A\mathbf{x}}_∞
= \max_{1 ≤ i ≤ n} ∑_{j=1}^n \abs{a_{ij}}.$$

Dla $\norm{\mathbf{x}}_∞ = 1$ mamy
\begin{align*}
\norm{A\mathbf{x}}_∞
= \max_{1 ≤ i ≤ n} \abs{\left(A\mathbf{x}\right)_i}
= \max_{1 ≤ i ≤ n} \abs{∑_{j=1}^n a_{ij} x_j}
≤ \max_{1 ≤ i ≤ n} ∑_{j=1}^n \abs{a_{ij} x_j}
≤ \max_{1 ≤ i ≤ n} ∑_{j=1}^n \abs{a_{ij}}.
\end{align*}
To oszacowanie górne jest osiągane dla $x_j = \sgn a_{ij}$.

\section*{M13.5}
Wykazać, że wzór
$$\norm{A}_E = \sqrt{∑_{1≤i,\, j≤n} a_{ij}^2}$$
definiuje submultiplikatywną normę w $ℝ^{n^2}$ zgodną z normą wektorową
$\norm{\;·\;}_2$.

Jest to norma, bo jest ona równoważna normie $\norm{\;·\;}_2$ dla wektorów z
$ℝ^{n^2}$.

Oznaczmy $i$-ty wiersz macierzy $A$ przez $a_{i*}$,
a $j$-tą kolumnę przez $a_{*j}$.

Korzystając z nierówności Cauchy'ego-Schwarza możemy pokazać
zgodność normy $\norm{\;·\;}_E$ z $\norm{\;·\;}_2$:
\begin{align*}
\norm{A\mathbf{x}}_2^2
&= ∑_{i=1}^n \left(a_{i*}\mathbf{x}\right)^2
≤ ∑_{i=1}^n \norm{a_{i*}}_2^2 \norm{\mathbf{x}}_2^2
= ∑_{1≤i,j≤n}^n a_{ij}^2 \norm{\mathbf{x}}_2^2
= \norm{A}_E^2 \norm{\mathbf{x}}_2^2.
\end{align*}

Ze zgodności norm możemy wywnioskować submultiplikatywność:
Niech $AB = C$, wtedy
\begin{align*}
\norm{C}_E
&= \sqrt{∑_{1≤i,\, j≤n} c_{ij}^2}
= \sqrt{∑_{1≤j≤n} \norm{c_{*j}}_2^2}
= \sqrt{∑_{1≤j≤n} \norm{A b_{*j}}_2^2} \\
&≤ \sqrt{∑_{1≤j≤n} \norm{A}_E^2 \norm{b_{*j}}_2^2}
= \norm{A}_E \sqrt{∑_{1≤j≤n} \norm{b_{*j}}_2^2}
= \norm{A}_E \sqrt{∑_{1≤i,\, j≤n} b_{ij}^2} \\
&= \norm{A}_E \norm{B}_E.
\end{align*}

\section*{M13.7}
\subsection*{(a)}
Niech $L$ to macierz trójkątna dolna z jedynkami na głównej przekątnej.
Oznaczmy jej odwrotność przez $L'$

Skoro $L L' = I$, to
$$
L [l'_{*1} \+ l'_{*2} \+ … \+ l'_{*n}]
=[L l'_{*1} \+ L l'_{*2} \+ … \+ L l'_{*n}]
=[e_1 \+ e_2 \+ … \+ e_n].
$$
Skupmy się na $k$-tej kolumnie.

Dla kolejnych $1 ≤ i < k$ mamy
$L_{i*} l'_{*k} = 0$, czyli z zerowości pierwszych $i-1$ wierszy $l'_{*k}$
i zerowości kolumn dalszych niż $i$ wiersza $L_{i*}$ mamy
$L_{ii} l'_{ik} = 0$.
Skoro $L_{ii} = 1$, to $l'_{ik} = 0$.

Podobnie mamy $L_{k*} l'_{*k} = 1$, więc $L_{kk} l'_{kk} = 1$ i $l'_{kk} = 1$.

Zatem $L'$ jest dolnotrójkątna z jedynkami na przekątnej.

\subsection*{(b)}
Niech $L$ to macierz dolnotrójkątna z jedynkami na przekątnej.

Niech $L_j$ to macierz powstała poprzez wyzerowanie w $L$ wyrazów
pod przekątną, poza kolumną $j$.
Są to macierze operacji wierszowych --
dodawania wielokrotności wierszy mnożonej macierzy do wiersza $j$.
Odwrotność $L_j$ powstaje poprzez negację wyrazów poza przekątną.

Mamy $L = L_1 L_2 … L_n$ i $L^{-1} = L_n^{-1} L_{n-1}^{-1} … L_1^{-1}$,
ten iloczyn jest łatwy do obliczenia, bo to tylko operacje wierszowe.



\end{document}
