\documentclass[a4paper, 12pt]{article}
\usepackage[utf8]{inputenc}
\usepackage{silence}
\usepackage{polski}
\usepackage{parskip}
\usepackage{amsmath,amsfonts,amssymb,amsthm}
\usepackage{mathtools}
\usepackage{newunicodechar}
\usepackage[margin=1.2in]{geometry}

\title{}
\author{Wiktor Kuchta}
\date{\vspace{-4ex}}

\newcommand{\modulus}[1]{\left| #1 \right|}
\newcommand{\abs}{\modulus}
\newunicodechar{∞}{\infty} % Digr 00
\newunicodechar{∑}{\sum}
\newunicodechar{∏}{\prod}

\WarningFilter{newunicodechar}{Redefining Unicode}
\newunicodechar{·}{\ifmmode\cdot\else\textperiodcentered\fi} % Digr .M

\begin{document}

\maketitle

\iffalse
% ŹLE ZROZUMIANE — ostatnia kula nie musi być czarna
\section*{4/2}
W urnie mamy $b$ kul białych i $c$ czarnych.
Po wyciągnięciu kuli z urny wrzucamy ją z powrotem
i dokładamy $d$ kul tego samego koloru.
Jakie jest prawdopodobieństwo wyciągnięcia $k$ kul czarnych w $n$ losowaniach?

Weźmy przykładowo $k=3$, $n=5$ i spójrzmy na prawdopodobieństwo
wybrania czarnej kuli za pierwszym, trzecim i piątym razem:
$$
\frac{c}{c+b} ·
\frac{b}{c+b+d} ·
\frac{c+d}{c+b+2d} ·
\frac{b+d}{c+b+3d} ·
\frac{c+2d}{c+b+4d}.
$$
Liczniki to liczby kul czarnych bądź białych w urnie,
a mianowniki to liczby wszystkich kul w urnie.
Rozwiązaniem jest suma wszystkich prawdopodobieństw takiej postaci.

Mianowniki zawsze będą takie same
— po $l$ losowaniach w urnie jest $c + b + ld$ kul.
Liczniki się zawsze wymnożą do $∏_{l=0}^{k-1} (c+ld) ∏_{m=0}^{n-k} (b+md)$
— po $l$ wylosowaniach kul czarnych w puli jest $c+ld$ kul czarnych,
a po $m$ wylosowaniach kul białych $b+md$ kul białych.

Liczba możliwych kolejności losowania to $\binom{n-1}{k-1}$,
bo interesują nas sytuacje, gdzie ostatnie losowanie daje nam kulę czarną.

Zatem sumarycznie prawdopodobieństwo wynosi
$$\frac{∏_{l=0}^{k-1}(c+ld) ∏_{l=0}^{n-k}(b+ld)}{∏_{l=0}^{n-1}(c+b+ld)}\binom{n-1}{k-1}.$$

\fi

\section*{4/6}

Przesyłane siecią pliki mogą z prawdopodobieństem $p$ być poprawnie przesłane,
z prawdopodobieństwem $q$ być przesłane, ale z pewnymi uszkodzeniami
albo z prawdopodobieństwem $1-p-q$ w trakcie przesyłu sieć może się zawiesić.
Jakie jest prawdopodobieństwo, że w trakcie wielokrotnego (niezależnego)
przesyłania plików poprawne przesłanie nastąpi przed zawieszeniem sieci?

Inaczej mówiąc, próbujemy przesłać pliki do skutku.
Prawdopodobieństwo zdarzenia,
że zanim się udało, przesłaliśmy $k$ razy (z uszkodzeniami),
wynosi $q^k p$.

Rozwiązaniem jest suma takich prawdopodobieństw dla wszystkich możliwych $k$:
$$∑_{k=0}^∞ q^kp = p∑_{k=0}^∞ q^k = p \frac{1}{1-q} = \frac{p}{1-q}.$$


\end{document}
