\documentclass[a4paper, 12pt]{article}
\usepackage[utf8]{inputenc}
\usepackage{silence}
\usepackage{polski}
\usepackage{amsmath,amsfonts,amssymb,amsthm}
\usepackage{mathtools}
\usepackage{newunicodechar}
\usepackage[margin=1.2in]{geometry}

\title{}
\author{Wiktor Kuchta}
\date{\vspace{-4ex}}

\DeclareMathOperator{\EX}{E}
\DeclareMathOperator{\Var}{Var}
\newcommand{\modulus}[1]{\left| #1 \right|}
\newcommand{\abs}{\modulus}
\newunicodechar{Θ}{\Theta}
\newunicodechar{Φ}{\Phi} % Digr F*
\newunicodechar{≈}{\approx} % Digr ?2

% cursed
\WarningFilter{newunicodechar}{Redefining Unicode}
\newunicodechar{…}{\ifmmode\dots\else\textellipsis\fi} % Digr .,


\begin{document}

\maketitle

\section*{7/8}
Niech $X_0=0$, $X_{j+1} \sim \mathcal{U}[X_{j}, 1]$ i $Y_k=2^k(1-X_k)$.
Zauważmy, że $\EX[X_k] = 1 - 2^{-k}$,
więc $\EX[Y_k] = 2^k(1-(1-2^{-k}))=1$.

Ogólniej,
$$\EX(X_{k+1} \mid X_1=x_1, …, X_k=x_k) = \frac{x_k+1}{2},$$ tzn.
$$\EX(X_{k+1} \mid X_1, …, X_k) = \frac{X_k+1}{2}.$$

Z liniowości wartości oczekiwanej
\begin{align*}
	\EX(2^{k+1}(1-X_{k+1}) \mid X_1,…,X_k)
		&= 2^{k+1}\left(1-\frac{X_k+1}{2}\right) \\
		&= 2^{k+1} - 2^kX_k - 2^k \\
		&= 2^k(1-X_k),\\
	\EX(Y_k \mid X_1,…,X_k) &= Y_k.
\end{align*}
Ciąg $(Y_i)$ jest martyngałem względem $(X_i)$,
a więc też względem samego siebie.

\section*{7/13}
Liczbę zepsutych laptopów możemy modelować zmienną o rozkładzie Bernoulliego
$X \sim B(1000, p)$ o wartości oczekiwanej $\EX[X] = 1000p$ i wariancji $\Var(X) = p(1-p)$.
Skoro liczba prób jest duża, to $X$ możemy aproksymować rozkładem normalnym,
a $X/1000$ (szansę awarii) aproksymować zmienną $Y \sim N(p, \frac{p(1-p)}{1000})$.

$$\Pr(\frac{|Y-p|}{\sqrt{p(1-p)/1000}} < x) = 1 - 2Φ(-x) = 2Φ(x)-1.$$
Chcemy $2Φ(x)-1 = 0.9$, więc $x = Φ^{-1}(0.95) ≈ 1.645$.
$$\Pr(|Y-p| < x{\sqrt{p(1-p)/1000}}) ≈ 0.95.$$

\iffalse

% CHYBA NAMIESZANE

Wstawiając pomiar $p=0.016$ otrzymujemy
$x{\sqrt{p(1-p)/1000}} ≈ 0.00653$,
więc przedział ufności to $0.016 ± 0.00653$.

\fi

\end{document}
