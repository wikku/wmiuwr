\documentclass[a4paper, 12pt]{article}
\usepackage[utf8]{inputenc}
\usepackage{silence}
\usepackage{polski}
\usepackage{amsmath,amsfonts,amssymb,amsthm}
\usepackage{mathtools}
%\usepackage{pgfplots}
%\pgfplotsset{compat=1.16}
\usepackage{newunicodechar}
\usepackage{etoolbox}
\usepackage[margin=1.2in]{geometry}

\title{}
\author{Wiktor Kuchta}
\date{\vspace{-4ex}}
\DeclareMathOperator{\EX}{E}

\newunicodechar{∑}{\sum}

\begin{document}

\maketitle

\section*{5/11}
Niech $X \sim B(n,p)$ i $Y \sim B(m,p)$ będą niezależnymi zmiennymi losowymi.
\begin{align*}
	\shortintertext{Z definicji możemy obliczyć funkcję tworzącą momenty $X$:}
	M_X(t)
		&= \EX[e^{tX}] \\
		&= ∑_{i=0}^n e^{ti} \binom{n}{i}p^i(1-p)^{n-i} \\
		&= ∑_{i=0}^n \binom{n}{i} (e^t p)^i (1-p)^{n-i} \\
		&= (e^t p + 1 - p)^n. \\ \\
	\shortintertext{To daje nam ogólny wzór, więc dla zmiennej $Y$:}
	M_Y(t)
		&= (e^t p + 1 - p)^m.
		\\
		\\
	\shortintertext{Korzystając z niezależności obliczamy funkcję tworzącą momenty $X+Y$:}
	M_{X+Y}(t)
		&= \EX[e^{t(X+Y)}] \\
		&= \EX[e^{tX} e^{tY}] \\
		&= \EX[e^{tX}] \EX[e^{tY}] \\
		&= (e^t p + 1 - p)^n (e^t p + 1 - p)^m \\
		&= (e^t p + 1 - p)^{n+m}.
\end{align*}
Skoro funkcje tworzące momenty jednoznacznie wyznaczają rozkład,
to
$$X+Y \sim B(n+m, p).$$

\end{document}
