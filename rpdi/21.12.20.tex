\documentclass[a4paper, 12pt]{article}
\usepackage[utf8]{inputenc}
\usepackage{polski}
\usepackage{amsmath,amsfonts,amssymb,amsthm}
\usepackage{mathtools}
\usepackage{newunicodechar}
\usepackage[margin=1.2in]{geometry}
\usepackage{microtype}

\title{}
\author{Wiktor Kuchta}
\date{\vspace{-4ex}}

\newcommand{\modulus}[1]{\left| #1 \right|}
\newcommand{\abs}{\modulus}
\DeclareMathOperator{\EX}{E}

\newunicodechar{∞}{\infty} % Digr 00
\newunicodechar{∑}{\sum}

\begin{document}

\maketitle

\section*{4/3}
Mamy chód losowy na liczbach naturalnych,
gdzie z prawdopodobieństwem $p$ idziemy w lewo o $1$,
a z prawdopodobieństwem $1-p$ w prawo o $1$.
Chcemy się dowiedzieć, jakie jest prawdopodobieństwo,
że startując z $b$ kiedyś osiągniemy $0$.

Na początek rozpatrzmy przypadek $b=1$.
Zero możemy osiągnąć tylko w nieparzystą liczbę kroków.
Aby osiągnąć zero po raz pierwszy po dokładnie $2k+1$ krokach,
musimy wykonać $k+1$ kroków w lewo i $k$ w prawo.
Ostatni krok musi być krokiem w lewo, a w trakcie pierwszych $2k$
kroków zawsze musiało być wykonanych nie więcej kroków w lewo niż w prawo.
Liczba możliwych takich ścieżek to liczba Catalana $C_n$.

Prawdopodobieństwo osiągnięcia $0$ zaczynając z $b=1$ to będzie zatem
$$∑_{k=0}^∞ C_k p^{k+1} (1-p)^k.$$
Do obliczenia szeregu możemy skorzystać z funkcji tworzącej liczb Catalana
$$c(x) = ∑_{k=0}^∞ C_n x^n = \frac{1-\sqrt{1-4x}}{2x},$$
wtedy otrzymujemy
\begin{align*}
∑_{k=0}^∞ C_k p^{k+1} q^k
&= p∑_{k=0}^∞ C_k p^k q^k \\
&= p c(p(1-p)) \\
&= p\frac{1-\sqrt{1-4p(1-p)}}{2p(1-p)} \\
&= \frac{1-\sqrt{1-4p+4p^2}}{2(1-p)} \\
&= \frac{1-\abs{1-2p}}{2(1-p)} \\
&= \max \left\{\frac{p}{1-p}, 1\right\}.
\end{align*}

Zauważmy,
że jest to ogólniej prawdopodobieństwo osiągnięcia $n$ zaczynając z $n+1$,
zatem prawdopodobieństwo osiągnięcia $0$ zaczynając z $b$ to będzie jego $b$-ta
potęga:
$$\max \left\{1, \left(\frac{p}{1-p}\right)^b \right\}.$$

\section*{4/10}
Algorytm losowo produkuje litery z alfabetu 26-znakowego.
Jaka jest spodziewana ilość wystąpień słowa ,,robak'' w ciągu 1000000 znaków?

Niech $X_i$ to indykator zdarzenia,
że na pozycji $i$ w ciągu mamy słowo ,,robak''.
Wtedy $\EX[X_i] = P(X_i=1) = 26^{-5}.$

Zatem oczekiwana liczba wystąpień to
$$\EX\left[∑_{k=1}^{999996} X_i\right] = ∑_{k=1}^{999996} \EX[X_i]
= \frac{999996}{26^5}.$$

\end{document}
