\documentclass[a4paper, 12pt]{article}
\usepackage[utf8]{inputenc}
\usepackage{silence}
\usepackage{polski}
\usepackage{parskip}
\usepackage{amsmath,amsfonts,amssymb,amsthm}
\usepackage{mathtools}
\usepackage{enumerate}
\usepackage{newunicodechar}
\usepackage[margin=1.2in]{geometry}

\title{}
\author{Wiktor Kuchta}
\date{\vspace{-4ex}}

\newcommand{\modulus}[1]{\left| #1 \right|}
\newcommand{\abs}{\modulus}
\DeclareMathOperator{\EX}{E}
\DeclareMathOperator{\Var}{Var}

\newunicodechar{∞}{\infty} % Digr 00
\newunicodechar{δ}{\delta} % Digr d*
\newunicodechar{σ}{\sigma}
\newunicodechar{Φ}{\Phi} % Digr F*
\newunicodechar{∑}{\sum}
\newunicodechar{∏}{\prod}
\newunicodechar{≤}{\le}
\newunicodechar{≥}{\ge}
\newunicodechar{≠}{\ne}
\newunicodechar{≈}{\approx} % Digr ?2
\newunicodechar{∈}{\in} % Digr (-

\WarningFilter{newunicodechar}{Redefining Unicode}
\newunicodechar{·}{\ifmmode\cdot\else\textperiodcentered\fi} % Digr .M
\newunicodechar{→}{\ifmmode\rightarrow\else\textrightarrow\fi} % Digr ->
\newunicodechar{…}{\ifmmode\dots\else\textellipsis\fi} % Digr .,

\begin{document}

\maketitle

\section*{6/9}
Niech $nX \sim B(n,p)$, wtedy $\EX[X] = p$ i $\Var(X) = σ^2 = \frac{1}{n}p(1-p) ≤ \frac{1}{4n}$.

Z nierówności Czebyszewa możemy oszacować
\begin{align*}
	\Pr(|X-p|>δ) =
	\Pr(|X-p|>\frac{δ}{σ}σ) ≤ \frac{σ^2}{δ^2}.
\end{align*}

\begin{enumerate}[i)]
	\item
		Jeśli $p< \frac{3}{10} $; to $σ^2 ≤ \frac{1}{n} · \frac{3}{10} · \frac{7}{10} = \frac{21}{100n}.$
		Wstawiając do powyższego i przyjmując $δ= \frac{1}{100} $ otrzymujemy
		$$\Pr(|X-p|> \frac{1}{100} ) ≤ \frac{2100}{n},$$
		co jest mniejsze niż $ \frac{1}{20} $ dla $n≥42000$.
	\item Jeśli nic nie wiemy o $p$, to
		$$\Pr(|X-p|> \frac{1}{100} ) ≤ \frac{2500}{n},$$
		co jest mniejsze niż $ \frac{1}{20} $ dla $n≥50000$.
\end{enumerate}


\newpage

\section*{6/11}
Niech $X_i$ dla $i=0,1,…$ będzie ciągiem wzajemnie niezależnych zmiennych
losowych o rozkładzie $\Pr(X_i=e^i)=\Pr(X_i=e^{3i})= \frac{1}{2}$ i $N \sim \mathcal{N}(0,1)$.

Niech $Z_i ∈ \{-1,1\}$ tak, aby $X_i = e^{(2+ Z_i) i}$.
Zauważmy, że $\frac{1}{n}∑_{i=1}^n Z_i \rightsquigarrow \mathcal{N}( 0 , 1 )$.
\begin{enumerate}[a)]
	\item
		\begin{align*}
			X_i^{\frac{1}{in}} &= e^{\frac{(2+Z_i)i}{in}} = e^{\frac{2+Z_i}{n}}, \\
			∏_{i=1}^{n+2018} X_i^{\frac{1}{in}} &= e^{\frac{1}{n}∑_{i=1}^{n+2018} (2+Z_i)}
			→ e^{2+N}. \\
			\Pr\left(∏_{i=1}^{n+2018} X_i^{\frac{1}{in}} < x\right) &≈ \Pr(2+N < \ln x) = Φ\left( \ln x - 2\right).
		\end{align*}
	\iffalse
	% ŹLE
	\item
		\begin{align*}
			\frac{X_i^{\frac{1}{i\sqrt{n}}}}{e^{\frac{2}{\sqrt{n}}}} &=
				\frac{e^{\frac{(2+Z_i)i}{i\sqrt{n}}}}{e^{\frac{2}{\sqrt{n}}}} =
				e^{\frac{(2+Z_i)}{\sqrt{n}}-\frac{2}{\sqrt{n}}} = % ≠ !!!
				e^{\frac{Z_i}{n}}, \\
			∏_{i=1}^{n+2018} \frac{X_i^{\frac{1}{i\sqrt{n}}}}{e^{\frac{2}{\sqrt{n}}}} &\rightarrow e^{N}. \\
			\Pr\left(∏_{i=1}^{n+2018} \frac{X_i^{\frac{1}{i\sqrt{n}}}}{e^{\frac{2}{\sqrt{n}}}} < x\right) &≈
			\Pr(N < \ln x) = Φ\left( \ln x \right).
		\end{align*}
	\fi
\end{enumerate}


\end{document}
