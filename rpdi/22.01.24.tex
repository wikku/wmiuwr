\documentclass[a4paper, 12pt]{article}
\usepackage[utf8]{inputenc}
\usepackage{silence}
\usepackage{polski}
\usepackage{parskip}
\usepackage{amsmath,amsfonts,amssymb,amsthm}
\usepackage{mathtools}
\usepackage{newunicodechar}
\usepackage[margin=1.2in]{geometry}

\title{}
\author{Wiktor Kuchta}
\date{\vspace{-4ex}}

\DeclareMathOperator{\sgn}{sgn}
\newcommand{\modulus}[1]{\left| #1 \right|}
\newcommand{\abs}{\modulus}
\DeclareMathOperator{\EX}{E}
\DeclareMathOperator{\Var}{Var}

\newunicodechar{∞}{\infty} % Digr 00
\newunicodechar{μ}{\mu}
\newunicodechar{σ}{\sigma}
\newunicodechar{∑}{\sum}
\newunicodechar{∏}{\prod}
\newunicodechar{≤}{\le}
\newunicodechar{≥}{\ge}
\newunicodechar{≠}{\ne}

% cursed
\WarningFilter{newunicodechar}{Redefining Unicode}
\newunicodechar{→}{\ifmmode\rightarrow\else\textrightarrow\fi} % Digr ->
\newunicodechar{…}{\ifmmode\dots\else\textellipsis\fi} % Digr .,

\begin{document}

\maketitle

\section*{5/16}
Niech $B, B_1, …, B_n$ to niezależne zmienne losowe przyjmujące z równym
prawdopodobieństwem wartości $-1$ i $1$.

Niech $S = B + p_1B_1 + … + p_nB_n.$
Interesuje nas
$$\Pr\left( \sgn S ≠ \sgn B\right),$$
co możemy oszacować z góry przez
$$\frac{1}{2}\Pr \left(\abs{∑_{i=1}^n p_i B_i} > 1\right),$$
bo $X = ∑_{i=1}^n p_i B_i$ musi mieć moduł większy od jeden, ale
z symetrii z prawdopodobieństwem $\frac{1}{2}$ ma znak równy $B$.

Zauważmy, że $μ = \EX[X] = \EX[B_i] = 0$ i z wzoru na wariancję kombinacji liniowej:
\begin{align*}
	σ^2 = \Var(X) &= ∑_{i=1}^n p_i^2 \Var(B_i) = ∑_{i=1}^n p_i^2 (\EX[B_i^2] - \EX[B_i]^2) = ∑_{i=1}^n p_i^2.
\end{align*}

Nierówność Czebyszewa daje nam
$$\Pr(|X|>1) ≤ \Pr\left(|X-μ| ≥ \frac{1}{σ}σ\right) ≤ σ^2,$$
zatem
$$\Pr(\sgn S ≠ \sgn B) ≤ \frac{1}{2}∑_{i=1}^n p_i^2.$$

\newpage

\section*{5/18}

Mamy graf $n$-wierzchołkowy,
w którym krawędź między wierzchołkami istnieje z prawdopodobieństem $p$.

Niech $X_i$ to indykator mówiący, czy wierzchołek $i$ jest izolowany,
tzn. jego stopień wynosi $0$, a $X = ∑_{i=1}^n X_i$.
Wtedy
\begin{align*}
	\EX [X_i] &= (1-p)^{n-1}, \\
	\EX [X] &= n \EX [X_i] = n (1-p)^{n-1}, \\
	\Var(X) &= \EX[X^2] - \EX[X]^2 \\
			&= ∑_{i=1}^n \EX[X_i^2] + ∑_{\stackrel{1 ≤ i, j ≤ n}{i≠j}} \EX[X_i X_j] - (n(1-p)^{n-1})^2 \\
			&= ∑_{i=1}^n \EX[X_i] + ∑_{\stackrel{1 ≤ i, j ≤ n}{i≠j}} (1-p)^{2n-3} - n^2(1-p)^{2n-2} \\
			&= n (1-p)^{n-1} + n(n-1) (1-p)^{2n-3} - n^2(1-p)^{2n-2}.
\end{align*}

Teraz przyjmijmy $p = \frac{c \ln n}{n}$.
Zauważmy, że
$$\left(1- \frac{c \ln n}{n}\right)^{n-1} \sim \left(1- \frac{c \ln n}{n}\right)^n \sim e^{-c \ln n} \sim n^{-c},$$
zatem
$$\EX[X] = n (1-p)^{n-1} \sim n^{1-c} →
\begin{cases}
	0, &\text{ gdy }c > 1 \\
	1, &\text{ gdy }c = 1 \\
	∞, &\text{ gdy }c < 1
\end{cases}.$$

Korzystając z nierówności Markowa otrzymujemy,
że dla $c > 1$ duże grafy prawie na pewno nie mają wierzchołków izolowanych:
$$\Pr(X ≥ 1) ≤ \frac{\EX[X]}{1} → 0.$$

% Tu może coś nie tak/nie o to chodzi
Prawdopodobieństwo, że jakiś wierzchołek jest izolowany wynosi
$$(1-p)^{n-1} \stackrel{\text{Bernoulli}}{≥} 1-(n-1)p ≥ 1- \frac{c(n-1)}{n \ln n} ≥ 1 - \frac{c}{\ln n},$$
zatem
$$\Pr(X=0) ≤ \frac{c}{\ln n}.$$


\end{document}
