\documentclass[a4paper, 12pt]{article}
\usepackage[utf8]{inputenc}
\usepackage{silence}
\usepackage{polski}
\usepackage{parskip}
\usepackage{amsmath,amsfonts,amssymb,amsthm}
\usepackage{mathtools}
\usepackage{enumitem}
%\usepackage{pgfplots}
%\pgfplotsset{compat=1.16}
\usepackage{newunicodechar}
\usepackage{etoolbox}
\usepackage[margin=1.2in]{geometry}
\usepackage{algorithm}
\usepackage{algorithmicx}
\usepackage{algpseudocode}
\setcounter{secnumdepth}{0}

\title{AISD}
\author{Wiktor Kuchta}
%\date{}

\DeclareMathOperator{\im}{Im}
\DeclareMathOperator{\rank}{rank}
\DeclareMathOperator{\Lin}{Lin}
\DeclareMathOperator{\sgn}{sgn}
\DeclareMathOperator{\Char}{char}
\newcommand{\N}{\mathbb{N}}
\newcommand{\Z}{\mathbb{Z}}
\newcommand{\Q}{\mathbb{Q}}
\newcommand{\R}{\mathbb{R}}
\newcommand{\C}{\mathbb{C}}
\newcommand{\inner}[2]{( #1 \, | \, #2)}
\newcommand{\norm}[1]{\left\lVert #1 \right\rVert}
\newcommand{\modulus}[1]{\left| #1 \right|}
\newcommand{\abs}{\modulus}
\newtheorem{theorem}{Twierdzenie}
\newtheorem{lemat}{Lemat}
\newcommand{\ol}{\overline}
\DeclareMathOperator{\tr}{tr}
\DeclareMathOperator{\diag}{diag}
\newcommand{\+}{\enspace}
\newcommand{\sump}{\sideset{}{'}{∑}} % sum prime
\newcommand{\sumb}{\sideset{}{"}{∑}} % sum bis

\newunicodechar{∅}{\emptyset} % Digr /0
\newunicodechar{∞}{\infty} % Digr 00
\newunicodechar{∂}{\partial} % Digr dP
\newunicodechar{α}{\alpha}
\newunicodechar{β}{\beta}
\newunicodechar{ξ}{\xi} % Digr c*
\newunicodechar{δ}{\delta} % Digr d*
\newunicodechar{ε}{\varepsilon}
\newunicodechar{φ}{\varphi}
\newunicodechar{θ}{\theta} % Digr h*
\newunicodechar{λ}{\lambda}
\newunicodechar{μ}{\mu}
\newunicodechar{π}{\pi}
\newunicodechar{σ}{\sigma}
\newunicodechar{τ}{\tau}
\newunicodechar{ω}{\omega}
\newunicodechar{η}{\eta} % Digr y*
\newunicodechar{Δ}{\Delta}
\newunicodechar{Θ}{\Theta}
\newunicodechar{Φ}{\Phi} % Digr F*
\newunicodechar{Π}{\Pi}
\newunicodechar{Ψ}{\Psi} % digr Q*
\newunicodechar{ℕ}{\N} % Digr NN 8469 nonstandard
\newunicodechar{ℤ}{\Z} % Digr ZZ 8484 nonstandard
\newunicodechar{ℚ}{\Q} % Digr QQ 8474 nonstandard
\newunicodechar{ℝ}{\R} % Digr RR 8477 nonstandard
\newunicodechar{ℂ}{\C} % Digr CC 8450 nonstandard
\newunicodechar{∑}{\sum}
\newunicodechar{∏}{\prod}
\newunicodechar{∫}{\int}
\newunicodechar{∓}{\mp}
\newunicodechar{⌈}{\lceil} % Digr <7
\newunicodechar{⌉}{\rceil} % Digr >7
\newunicodechar{⌊}{\lfloor} % Digr 7<
\newunicodechar{⌋}{\rfloor} % Digr 7>
\newunicodechar{≅}{\cong} % Digr ?=
\newunicodechar{≡}{\equiv} % Digr 3=
\newunicodechar{◁}{\triangleleft} % Digr Tl
\newunicodechar{▷}{\triangleright} % Digr Tr
\newunicodechar{≤}{\le}
\newunicodechar{≥}{\ge}
\newunicodechar{≪}{\ll} % Digr <*
\newunicodechar{≫}{\gg} % Digr *>
\newunicodechar{≠}{\ne}
\newunicodechar{⊆}{\subseteq} % Digr (_
\newunicodechar{⊇}{\supseteq} % Digr _)
\newunicodechar{⊂}{\subset} % Digr (C
\newunicodechar{⊃}{\supset} % Digr C)
\newunicodechar{∩}{\cap} % Digr (U
\newunicodechar{∪}{\cup} % Digr )U
\newunicodechar{∼}{\sim} % Digr ?1
\newunicodechar{≈}{\approx} % Digr ?2
\newunicodechar{∈}{\in} % Digr (-
\newunicodechar{∋}{\ni} % Digr -)
\newunicodechar{∇}{\nabla} % Digr NB
\newunicodechar{∃}{\exists} % Digr TE
\newunicodechar{∀}{\forall} % Digr FA
\newunicodechar{∧}{\wedge} % Digr AN
\newunicodechar{∨}{\vee} % Digr OR
\newunicodechar{⊥}{\bot} % Digr -T
\newunicodechar{⊤}{\top} % Digr TO 8868 nonstandard
\newunicodechar{⇒}{\implies} % Digr =>
\newunicodechar{⇐}{\impliedby} % Digr <=
\newunicodechar{⇔}{\iff} % Digr ==
\newunicodechar{↔}{\leftrightarrow} % Digr <>
\newunicodechar{↦}{\mapsto} % Digr T> 8614 nonstandard
\newunicodechar{∘}{\circ} % Digr Ob

% cursed
\WarningFilter{newunicodechar}{Redefining Unicode}
\newunicodechar{·}{\ifmmode\cdot\else\textperiodcentered\fi} % Digr .M
\newunicodechar{×}{\ifmmode\times\else\texttimes\fi} % Digr *X
\newunicodechar{→}{\ifmmode\rightarrow\else\textrightarrow\fi} % Digr ->
\newunicodechar{←}{\ifmmode\leftarrow\else\textleftarrow\fi} % Digr ->
\newunicodechar{⟨}{\ifmmode\langle\else\textlangle\fi} % Digr LA 10216 nonstandard
\newunicodechar{⟩}{\ifmmode\rangle\else\textrangle\fi} % Digr RA 10217 nonstandard
\newunicodechar{…}{\ifmmode\dots\else\textellipsis\fi} % Digr .,
\newunicodechar{±}{\ifmmode\pm\else\textpm\fi} % Digr +-

% https://tex.stackexchange.com/a/438184
% https://tex.stackexchange.com/q/528480
\newunicodechar{∶}{\mathbin{\text{:}}}
\def\newcolon{%
  \nobreak\mskip2mu\mathpunct{}\nonscript\mkern-\thinmuskip{\text{:}}%
  \mskip 6mu plus 1 mu \relax}
\mathcode`:="8000
{\catcode`:=\active \global\let:\newcolon}
% colon: for types; ratio∶ (digr :R) for relations (set builder)


\begin{document}

\maketitle

\section*{2/2}
Sortujemy przedziały rosnąco według prawego końca.
Wybieramy zachłannie pierwszy w tym porządku przedział,
który jeszcze nie przecina poprzednio wybranych.

Niech $((p_{i_1}, k_{i_1}), …, (p_{i_l}, k_{i_l}))$ to wyjście
algorytmu, a $((p_{j_1}, k_{j_1}), …, (p_{j_k}, k_{j_k}))$ to jakieś
inne nieprzecinające się odcinki.

Pokażemy, że dla każdego $m≤k$
przedział $(p_{i_m}, k_{i_m})$ jest w rozwiązaniu i $k_{i_m} ≤ b_{j_m}$.

Dla $m=1$ oczywiste, bo na początku zawsze wybieramy przedział z najwcześniejszym końcem.

Jeśli $m>1$ to najmniejszy kontrprzykład, to
przedział $i_m$ musi istnieć (z własności dla poprzednich możemy wybrać kolejny).
Wtedy
$k_{i_{m-1}} ≤ k_{j_{m-1}} ≤ p_{j_m}$ i $k_{i_m} > k_{j_m}$.
Ale wtedy przedział $j_m$ zostałby wybrany przez algorytm, sprzeczność.

\section*{2/3}
Pokażemy, że iterowanie funkcji
$$f\left(\frac{a}{b}\right) = \frac{a}{b} - \frac{1}{n} = \frac{na-b}{nb},
\+ \text{gdzie }n = \left\lceil\frac{b}{a}\right\rceil$$
da nam kiedyś wartość $0$.

Dla $\frac{b}{a}$ naturalnego $f\left(\frac{a}{b}\right) = 0$.
W przeciwnym wypadku mamy
\begin{align*}
	\left\lfloor \frac{b}{a} \right\rfloor a - b &< 0, \\
	(n-1)a - b &< 0, \\
	na - b &< a,
\end{align*}
więc licznik $f\left(\frac{a}{b}\right)$ jest mniejszy od $a$.
Zatem tezę można udowodnić prostą indukcją względem licznika.

Teraz wystarczy pokazać, że odejmowane w kolejnych iteracjach
ułamki są różne.

Zauważmy, że
$$
\left\lceil \frac{nb}{na-b} \right\rceil
≥ \left\lceil \frac{nb}{a} \right\rceil
= n \left\lceil \frac{b}{a} \right\rceil = n^2.
$$
Mamy założenie $a≤b$.
Jeśli $a=b$, to $f(\frac{a}{b}) = 0$.
W przeciwnym wypadku mamy $n=\left\lceil \frac{b}{a} \right\rceil > 1$,
więc z powyższej nierówności mianowniki ułamków odejmowanych w kolejnych
iteracjach będą coraz większe.

Zachłanny algorytm odejmowania największej możliwej odwrotności
nie jest optymalny, bo wyliczy
$\frac{9}{20} = \frac{1}{3} + \frac{1}{9} + \frac{1}{180}$,
ale można krócej $\frac{9}{20} = \frac{1}{4} + \frac{1}{5}$.

\section*{2/4}
Niech warstwa $i$ to liście w drzewie pozostałym po usunięciu warstw $1, …, i-1$.

Rozwiązaniem maksymalnym jest pokolorowanie pierwszych $k/2$ warstw i
co najwyżej jednego innego wierzchołka, jeśli $k$ jest nieparzyste.
Pokażemy, że możemy do niego sprowadzić każde rozwiązanie optymalne.

Weźmy optymalne pokolorowanie drzewa.
Pokażemy, że musi ono mieć pokolorowane wierzchołki.
Wtedy problem się redukuje do grafu z usuniętymi liściami i parametru $k ↦ k-2$.

\section*{2/6}
Jeśli $e$ nie jest maksymalnej wagi na pewnym cyklu, to należy do pewnego MST.

Usuńmy $e=(v,w)$ i wszystkie krawędzie o wadze większej niż $e$ z grafu.
Jeśli $w$ jest nieosiągalny z $v$, to albo graf się rozspójnił, albo $e$ nie jest
maksymalny na żadnym cyklu, więc w obu przypadkach $e$ należy do pewnego MST.

Jeśli $w$ jest osiągalny z $v$, to $e$ jest maksymalny na pewnym cyklu,
więc nie należy do MST.



\end{document}
