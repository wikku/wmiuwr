\documentclass[a4paper, 12pt]{article}
\usepackage[utf8]{inputenc}
\usepackage{silence}
\usepackage{polski}
\usepackage{parskip}
\usepackage{amsmath,amsfonts,amssymb,amsthm}
\usepackage{mathtools}
\usepackage{newunicodechar}
\usepackage{etoolbox}
\usepackage[margin=1.2in]{geometry}
\usepackage{cmll}
\usepackage{bussproofs}
\setcounter{secnumdepth}{0}

\title{Logika liniowa: ćwiczenia}
\author{Wiktor Kuchta}
\date{27 kwietnia 2021}

\newunicodechar{Δ}{\Delta}
\newunicodechar{Γ}{\Gamma}
\newunicodechar{Π}{\Pi}
\newunicodechar{Σ}{\Sigma} % digr S*
\newunicodechar{⊃}{\supset} % Digr C)
\newunicodechar{⊥}{\bot} % Digr -T
\newunicodechar{⊢}{\vdash} % Digr \- 8866 nonstandard
\newunicodechar{⊕}{\oplus} % Digr O+ 8853
\newunicodechar{⊗}{\otimes} % Digr OX 8855

% cursed
\WarningFilter{newunicodechar}{Redefining Unicode}
\newunicodechar{·}{\ifmmode\cdot\else\textperiodcentered\fi} % Digr .M


\begin{document}

\maketitle

\textbf{Zadanie 1.}
Udowodnij
$⊢ (A \with B) \with C \multimapboth A \with (B \with C)$
oraz
$⊢ (A ⊗ B) ⊗ C \multimapboth A ⊗ (B ⊗ C)$
w rachunku sekwentów logiki liniowej.

\textbf{Zadanie 2.}
Udowodnij
$⊢ (\wn A \parr \wn B) \multimapboth (\wn (A ⊕ B))$
w rachunku sekwentów logiki liniowej.

\textbf{Zadanie 3.}
Uzasadnij, że
w rachunku sekwentów logiki liniowej
nie da się udowodnić pustego sekwentu $⊢$.
Wywnioskuj, że nie da się mieć jednocześnie $⊢ A$ i $⊢ A^⊥$.

\textbf{Zadanie 4.}
W liniowej intuicjonistycznej dedukcji naturalnej
możemy średnikiem oddzielać formuły nieograniczone (po lewej) i formuły liniowe.
Mamy aksjomaty
\begin{center}
	\AxiomC{}
	\RightLabel{$u$}
	\UnaryInfC{$Γ; A ⊢ A$}
	\DisplayProof
	\hskip 1em
	\AxiomC{}
	\RightLabel{$v$}
	\UnaryInfC{$(Γ, A); · ⊢ A$}
	\DisplayProof
\end{center}
Kropka ($·$) oznacza brak formuł.
Możemy wprowadzić nowy spójnik ,,nieograniczonej implikacji'':
\begin{center}
	\AxiomC{$(Γ, A); Δ ⊢ B$}
	\RightLabel{$⊃$I}
	\UnaryInfC{$Γ; Δ ⊢ A ⊃ B$}
	\DisplayProof
	\hskip 1em
	\AxiomC{$Γ; Δ ⊢ A ⊃ B$}
	\AxiomC{$Γ; · ⊢ A$}
	\RightLabel{$⊃$E}
	\BinaryInfC{$Γ; Δ ⊢ B$}
	\DisplayProof
\end{center}
Uzasadnij, że w prawej przesłance reguły eliminacji niepoprawne byłoby zastąpienie
$·$ przez $Σ$,
tzn. niepoprawne byłoby zezwolenie na liniowe założenia w prawej przesłance.
\end{document}
