\documentclass[a4paper, 12pt]{article}
\usepackage[utf8]{inputenc}
\usepackage{silence}
\usepackage{polski}
\usepackage{parskip}
\usepackage{amsmath,amsfonts,amssymb,amsthm}
\usepackage{mathtools}
\usepackage{enumitem}
%\usepackage{pgfplots}
%\pgfplotsset{compat=1.16}
\usepackage{newunicodechar}
\usepackage{etoolbox}
\usepackage[margin=1.2in]{geometry}
\usepackage{algorithm}
\usepackage{algorithmicx}
\usepackage{algpseudocode}
\setcounter{secnumdepth}{0}

\title{}
\author{Wiktor Kuchta}
\date{\vspace{-4ex}}

\DeclareMathOperator{\im}{Im}
\DeclareMathOperator{\rank}{rank}
\DeclareMathOperator{\Lin}{Lin}
\DeclareMathOperator{\End}{End}
\DeclareMathOperator{\sgn}{sgn}
\DeclareMathOperator{\Char}{char}
\newcommand{\N}{\mathbb{N}}
\newcommand{\Z}{\mathbb{Z}}
\newcommand{\Q}{\mathbb{Q}}
\newcommand{\R}{\mathbb{R}}
\newcommand{\C}{\mathbb{C}}
\newcommand{\inner}[2]{( #1 \, | \, #2)}
\newcommand{\norm}[1]{\left\lVert #1 \right\rVert}
\newcommand{\modulus}[1]{\left| #1 \right|}
\newcommand{\abs}{\modulus}
\newtheorem{theorem}{Twierdzenie}
\newtheorem{lemat}{Lemat}
\newcommand{\ol}{\overline}
\DeclareMathOperator{\tr}{tr}
\DeclareMathOperator{\diag}{diag}
\newcommand{\+}{\enspace}
\newcommand{\sump}{\sideset{}{'}{∑}} % sum prime
\newcommand{\sumb}{\sideset{}{"}{∑}} % sum bis

\newunicodechar{∅}{\emptyset} % Digr /0
\newunicodechar{∞}{\infty} % Digr 00
\newunicodechar{∂}{\partial} % Digr dP
\newunicodechar{α}{\alpha}
\newunicodechar{β}{\beta}
\newunicodechar{ξ}{\xi} % Digr c*
\newunicodechar{δ}{\delta} % Digr d*
\newunicodechar{ε}{\varepsilon}
\newunicodechar{φ}{\varphi}
\newunicodechar{θ}{\theta} % Digr h*
\newunicodechar{λ}{\lambda}
\newunicodechar{μ}{\mu}
\newunicodechar{π}{\pi}
\newunicodechar{ψ}{\psi}
\newunicodechar{ρ}{\rho}
\newunicodechar{σ}{\sigma}
\newunicodechar{τ}{\tau}
\newunicodechar{ω}{\omega}
\newunicodechar{η}{\eta} % Digr y*
\newunicodechar{ζ}{\zeta} % Digr z*
\newunicodechar{Δ}{\Delta}
\newunicodechar{Γ}{\Gamma}
\newunicodechar{Λ}{\Lambda}
\newunicodechar{Θ}{\Theta}
\newunicodechar{Φ}{\Phi} % Digr F*
\newunicodechar{Π}{\Pi}
\newunicodechar{Ψ}{\Psi} % digr Q*
\newunicodechar{Σ}{\Sigma} % digr S*
\newunicodechar{Ω}{\Omega} % digr W*
\newunicodechar{ℕ}{\N} % Digr NN 8469 nonstandard
\newunicodechar{ℤ}{\Z} % Digr ZZ 8484 nonstandard
\newunicodechar{ℚ}{\Q} % Digr QQ 8474 nonstandard
\newunicodechar{ℝ}{\R} % Digr RR 8477 nonstandard
\newunicodechar{ℂ}{\C} % Digr CC 8450 nonstandard
\newunicodechar{∑}{\sum}
\newunicodechar{∏}{\prod}
\newunicodechar{∫}{\int}
\newunicodechar{∓}{\mp}
\newunicodechar{⌈}{\lceil} % Digr <7
\newunicodechar{⌉}{\rceil} % Digr >7
\newunicodechar{⌊}{\lfloor} % Digr 7<
\newunicodechar{⌋}{\rfloor} % Digr 7>
\newunicodechar{≅}{\cong} % Digr ?=
\newunicodechar{≡}{\equiv} % Digr 3=
\newunicodechar{◁}{\triangleleft} % Digr Tl
\newunicodechar{▷}{\triangleright} % Digr Tr
\newunicodechar{≤}{\le}
\newunicodechar{≥}{\ge}
\newunicodechar{≪}{\ll} % Digr <*
\newunicodechar{≫}{\gg} % Digr *>
\newunicodechar{≠}{\ne}
\newunicodechar{⊆}{\subseteq} % Digr (_
\newunicodechar{⊇}{\supseteq} % Digr _)
\newunicodechar{⊂}{\subset} % Digr (C
\newunicodechar{⊃}{\supset} % Digr C)
\newunicodechar{∩}{\cap} % Digr (U
\newunicodechar{∖}{\setminus} % Digr -\ 8726 nonstandard
\newunicodechar{∪}{\cup} % Digr )U
\newunicodechar{∼}{\sim} % Digr ?1
\newunicodechar{≈}{\approx} % Digr ?2
\newunicodechar{∈}{\in} % Digr (-
\newunicodechar{∋}{\ni} % Digr -)
\newunicodechar{∇}{\nabla} % Digr NB
\newunicodechar{∃}{\exists} % Digr TE
\newunicodechar{∀}{\forall} % Digr FA
\newunicodechar{∧}{\wedge} % Digr AN
\newunicodechar{∨}{\vee} % Digr OR
\newunicodechar{⊥}{\bot} % Digr -T
\newunicodechar{⊢}{\vdash} % Digr \- 8866 nonstandard
\newunicodechar{⊤}{\top} % Digr TO 8868 nonstandard
\newunicodechar{⇒}{\implies} % Digr =>
\newunicodechar{⊸}{\multimap} % Digr #> nonstandard
\newunicodechar{⇐}{\impliedby} % Digr <=
\newunicodechar{⇔}{\iff} % Digr ==
\newunicodechar{↔}{\leftrightarrow} % Digr <>
\newunicodechar{↦}{\mapsto} % Digr T> 8614 nonstandard
\newunicodechar{∘}{\circ} % Digr Ob
\newunicodechar{⊕}{\oplus} % Digr O+ 8853
\newunicodechar{⊗}{\otimes} % Digr OX 8855

% cursed
\WarningFilter{newunicodechar}{Redefining Unicode}
\newunicodechar{·}{\ifmmode\cdot\else\textperiodcentered\fi} % Digr .M
\newunicodechar{×}{\ifmmode\times\else\texttimes\fi} % Digr *X
\newunicodechar{→}{\ifmmode\rightarrow\else\textrightarrow\fi} % Digr ->
\newunicodechar{←}{\ifmmode\leftarrow\else\textleftarrow\fi} % Digr ->
\newunicodechar{⟨}{\ifmmode\langle\else\textlangle\fi} % Digr LA 10216 nonstandard
\newunicodechar{⟩}{\ifmmode\rangle\else\textrangle\fi} % Digr RA 10217 nonstandard
\newunicodechar{…}{\ifmmode\dots\else\textellipsis\fi} % Digr .,
\newunicodechar{±}{\ifmmode\pm\else\textpm\fi} % Digr +-

% https://tex.stackexchange.com/a/438184
% https://tex.stackexchange.com/q/528480
\newunicodechar{∶}{\mathbin{\text{:}}}
\def\newcolon{%
  \nobreak\mskip2mu\mathpunct{}\nonscript\mkern-\thinmuskip{\text{:}}%
  \mskip 6mu plus 1 mu \relax}
\mathcode`:="8000
{\catcode`:=\active \global\let:\newcolon}
% colon: for types; ratio∶ (digr :R) for relations (set builder)


\begin{document}

\maketitle
\section*{8/6a}
Niech $K_1, K_2$ będą ciałami, zaś $R = K_1 × K_2$ i $M$ to $R$-moduł.

Zauważmy, że $V_1 = (1,0)M$ i $V_2 = (0,1)M$ to podmoduły $M$ i $M = V_1 + V_2$.
Jeśli
$$(1,0)m = (0,1)m' \quad \text{ dla }m, m' ∈ M,$$
to
$$(0,1)(1,0)m = (0,0)m = 0 = (0,1)(0,1)m' = (0,1)m',$$
zatem $V_1 ∩ V_2 = \{ 0 \}$.
Stąd $M = V_1 ⊕ V_2 ≅ V_1 × V_2$.

Na $V_1$ możemy określić strukturę $K_1$-modułu
przez $k_1 v_1 = (k_1,0)v_1$, podobnie z $V_2$.

Każdy $v_1∈V_1$ jest równy $(1,0)v_1'$ dla pewnego $v_1' ∈ M$,
analogicznie z $v_2 ∈ V_2$.
Zatem
\begin{align*}
(k_1, k_2)(v_1 + v_2)
&= (k_1,k_2)v_1 + (k_1,k_2)v_2
= (k_1,k_2)(1,0)v_1' + (k_1,k_2)(0,1)v_2' \\
&= (k_1,0)(1,0)v_1' + (0,k_2)(0,1)v_2'
= (k_1,0)v_1 + (0,k_2)v_2 \\
&= k_1v_1 + k_2v_2.
\end{align*}


\section*{9/1b}
Niech $I,J ◁ R$.

Niech
$f: R/I × R/J → R/(I+J)$,
$f(r+I,s+J) = rs + (I + J)$.
Jest to epimorfizm dwuliniowy,
więc z uniwersalnej własności produktu tensorowego
istnieje epimorfizm $\tilde{f}: R/I ⊗_R R/J → R/(I+J)$
taki,
że $f = \tilde{f} ∘ ⊗$.

Weźmy $(r+I)⊗(s+J) ∈ \ker \tilde{f}$.
Z dwuliniowości mamy
$$(r+I)⊗(s+J) = (r+I)⊗s(1+J) = (rs+I)⊗(1+J).$$
Z definicji $f$ mamy $rs ∈ I+J$, więc $r = x+y$,
gdzie $x∈I$ i $y ∈ J$.
Stąd
\begin{align*}
	(rs+I)⊗(1+J) &= (x+y+I)⊗(1+J) = ((x+I)+(y+I))⊗(1+J) \\
				 &= (x+I)⊗(1+J)+(y+I)⊗(1+J) \\
				 &= 0⊗(1+J) + (1+I)⊗(y+J) = 0.
\end{align*}
Jądro $\tilde{f}$ jest trywialne, zatem jest to izomorfizm.

\section*{9/2a}
Niech $G$ będzie grupą abelową.

Każdy element $ℚ ⊗_ℤ G$ jest postaci
$$∑_{i∈I} \frac{p_i}{q_i} ⊗ g_i, \quad \text{ gdzie }|I|<∞, p_i∈ℤ, q_i ∈ ℕ_+, g_i ∈ G.$$

Możemy zdefiniować mnożenie elementów tej grupy przez
liczby wymierne zgodne z mnożeniem liczb wymiernych:
$$ \frac{p}{q} \left( \frac{r}{s} ∑_{i∈I} \frac{p_i}{q_i} ⊗ g_i \right)
=  \frac{p}{q} \left( ∑_{i∈I} \frac{r p_i}{s q_i} ⊗ g_i \right)
=  ∑_{i∈I} \frac{p r p_i}{q s q_i} ⊗ g_i
=  \frac{pr}{qs} \left( ∑_{i∈I} \frac{p_i}{q_i} ⊗ g_i \right). $$
To działanie naturalnie rozszerza mnożenie skalarne przez liczby całkowite.
Rozdzielność mnożenia względem dodawania wektorów i to, że mnożenie
przez $\frac{1}{1}$ jest identycznością są jasne.

Jest to grupa podzielna, bo możemy mnożyć przez $\frac{1}{n}$.

Załóżmy, że $nx = 0$ dla pewnego $n∈ℕ_+$ i $x∈ ℚ ⊗_ℤ G$.
Wtedy $$x = \frac{n}{n} x = \frac{1}{n} nx = 0,$$
więc grupa jest beztorsyjna.

Można sprawdzić rozdzielność mnożenia względem dodawania skalarów:
\begin{align*}
\left(\frac{p}{q} + \frac{r}{s} \right) ∑_{i∈I} \frac{p_i}{q_i} ⊗ g_i
&= \frac{ps+qr}{qs} ∑_{i∈I} \frac{p_i}{q_i} ⊗ g_i
= ∑_{i∈I} \frac{(ps+qr)p_i}{qs q_i} ⊗ g_i \\
&= ∑_{i∈I} \left( \frac{ps p_i}{qs q_i} + \frac{qr p_i}{qs q_i} \right) ⊗ g_i
= ∑_{i∈I} \frac{ps p_i}{qs q_i} ⊗ g_i + ∑_{i∈I} \frac{qr p_i}{qs q_i} ⊗ g_i \\
&= \frac{p}{q} ∑_{i∈I} \frac{p_i}{q_i} ⊗ g_i + \frac{r}{s} ∑_{i∈I} \frac{ p_i}{q_i} ⊗ g_i
\end{align*}
Zatem jest to $ℚ$-moduł, czyli przestrzeń liniowa nad $ℚ$.

\section*{9/3}
Załóżmy, że $R$ jest pierścieniem przemiennym z jedynką i
$M$ jest $R$-modułem prostym.
Ustalmy pewien niezerowy $a∈M$.

Homomorfizm $(r ↦ ra) : R → M$ jest surjektywny, bo
jest niezerowy, a jego obraz jest podmodułem modułu prostego $M$.

Niech $f: R → \End_R(M)$, $f(r)(m) = rm$.
Pokażemy, że to epimorfizm.

Weźmy $h ∈ \End_R(M)$.
Mamy $h(a) = ra$ dla pewnego $r∈R$.
Więc
$$(m ↦ h(m) - rm) = h - f(r)$$
jest endomorfizmem $M$
z nietrywialnym jądrem, a więc homomorfizmem zerowym.
Zatem $h=f(r)$.

Z zasadniczego twierdzenia o homomorfizmie
$R/I ≅ \End_R(M)$ dla $I=\ker f ◁ R$.
$\End_R(M)$ jest ciałem z lematu Schura,
a skoro $R/I$ jest ciałem, to $I$ jest ideałem
maksymalnym $R$.

\end{document}
