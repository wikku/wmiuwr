\documentclass[a4paper, 12pt]{article}
\usepackage[utf8]{inputenc}
\usepackage{silence}
\usepackage{polski}
\usepackage{parskip}
\usepackage{amsmath,amsfonts,amssymb,amsthm}
\usepackage{mathtools}
\usepackage{enumitem}
%\usepackage{pgfplots}
%\pgfplotsset{compat=1.16}
\usepackage{newunicodechar}
\usepackage{etoolbox}
\usepackage[margin=1.2in]{geometry}
\usepackage{algorithm}
\usepackage{algorithmicx}
\usepackage{algpseudocode}
\setcounter{secnumdepth}{0}

\title{}
\author{Wiktor Kuchta}
\date{\vspace{-4ex}}

\DeclareMathOperator{\im}{Im}
\DeclareMathOperator{\rank}{rank}
\DeclareMathOperator{\Lin}{Lin}
\DeclareMathOperator{\sgn}{sgn}
\DeclareMathOperator{\Char}{char}
\DeclareMathOperator{\id}{id}
\newcommand{\N}{\mathbb{N}}
\newcommand{\Z}{\mathbb{Z}}
\newcommand{\Q}{\mathbb{Q}}
\newcommand{\R}{\mathbb{R}}
\newcommand{\C}{\mathbb{C}}
\newcommand{\inner}[2]{( #1 \, | \, #2)}
\newcommand{\norm}[1]{\left\lVert #1 \right\rVert}
\newcommand{\modulus}[1]{\left| #1 \right|}
\newcommand{\abs}{\modulus}
\newtheorem{theorem}{Twierdzenie}
\newtheorem{lemat}{Lemat}
\newcommand{\ol}{\overline}
\DeclareMathOperator{\tr}{tr}
\DeclareMathOperator{\diag}{diag}
\newcommand{\+}{\enspace}
\newcommand{\sump}{\sideset{}{'}{∑}} % sum prime
\newcommand{\sumb}{\sideset{}{"}{∑}} % sum bis

\newunicodechar{∅}{\emptyset} % Digr /0
\newunicodechar{∞}{\infty} % Digr 00
\newunicodechar{∂}{\partial} % Digr dP
\newunicodechar{α}{\alpha}
\newunicodechar{β}{\beta}
\newunicodechar{ξ}{\xi} % Digr c*
\newunicodechar{δ}{\delta} % Digr d*
\newunicodechar{ε}{\varepsilon}
\newunicodechar{φ}{\varphi}
\newunicodechar{θ}{\theta} % Digr h*
\newunicodechar{λ}{\lambda}
\newunicodechar{μ}{\mu}
\newunicodechar{π}{\pi}
\newunicodechar{ψ}{\psi}
\newunicodechar{ρ}{\rho}
\newunicodechar{σ}{\sigma}
\newunicodechar{τ}{\tau}
\newunicodechar{ω}{\omega}
\newunicodechar{η}{\eta} % Digr y*
\newunicodechar{ζ}{\zeta} % Digr z*
\newunicodechar{Δ}{\Delta}
\newunicodechar{Γ}{\Gamma}
\newunicodechar{Λ}{\Lambda}
\newunicodechar{Θ}{\Theta}
\newunicodechar{Φ}{\Phi} % Digr F*
\newunicodechar{Π}{\Pi}
\newunicodechar{Ψ}{\Psi} % digr Q*
\newunicodechar{Σ}{\Sigma} % digr S*
\newunicodechar{Ω}{\Omega} % digr W*
\newunicodechar{ℕ}{\N} % Digr NN 8469 nonstandard
\newunicodechar{ℤ}{\Z} % Digr ZZ 8484 nonstandard
\newunicodechar{ℚ}{\Q} % Digr QQ 8474 nonstandard
\newunicodechar{ℝ}{\R} % Digr RR 8477 nonstandard
\newunicodechar{ℂ}{\C} % Digr CC 8450 nonstandard
\newunicodechar{∑}{\sum}
\newunicodechar{∏}{\prod}
\newunicodechar{∫}{\int}
\newunicodechar{∓}{\mp}
\newunicodechar{⌈}{\lceil} % Digr <7
\newunicodechar{⌉}{\rceil} % Digr >7
\newunicodechar{⌊}{\lfloor} % Digr 7<
\newunicodechar{⌋}{\rfloor} % Digr 7>
\newunicodechar{≅}{\cong} % Digr ?=
\newunicodechar{≡}{\equiv} % Digr 3=
\newunicodechar{◁}{\triangleleft} % Digr Tl
\newunicodechar{▷}{\triangleright} % Digr Tr
\newunicodechar{≤}{\le}
\newunicodechar{≥}{\ge}
\newunicodechar{≪}{\ll} % Digr <*
\newunicodechar{≫}{\gg} % Digr *>
\newunicodechar{≠}{\ne}
\newunicodechar{⊆}{\subseteq} % Digr (_
\newunicodechar{⊇}{\supseteq} % Digr _)
\newunicodechar{⊂}{\subset} % Digr (C
\newunicodechar{⊃}{\supset} % Digr C)
\newunicodechar{∩}{\cap} % Digr (U
\newunicodechar{∖}{\setminus} % Digr -\ 8726 nonstandard
\newunicodechar{∪}{\cup} % Digr )U
\newunicodechar{∼}{\sim} % Digr ?1
\newunicodechar{≈}{\approx} % Digr ?2
\newunicodechar{∈}{\in} % Digr (-
\newunicodechar{∋}{\ni} % Digr -)
\newunicodechar{∇}{\nabla} % Digr NB
\newunicodechar{∃}{\exists} % Digr TE
\newunicodechar{∀}{\forall} % Digr FA
\newunicodechar{∧}{\wedge} % Digr AN
\newunicodechar{∨}{\vee} % Digr OR
\newunicodechar{⊥}{\bot} % Digr -T
\newunicodechar{⊢}{\vdash} % Digr \- 8866 nonstandard
\newunicodechar{⊤}{\top} % Digr TO 8868 nonstandard
\newunicodechar{⇒}{\implies} % Digr =>
\newunicodechar{⊸}{\multimap} % Digr #> nonstandard
\newunicodechar{⇐}{\impliedby} % Digr <=
\newunicodechar{⇔}{\iff} % Digr ==
\newunicodechar{↔}{\leftrightarrow} % Digr <>
\newunicodechar{↦}{\mapsto} % Digr T> 8614 nonstandard
\newunicodechar{∘}{\circ} % Digr Ob
\newunicodechar{⊕}{\oplus} % Digr O+ 8853
\newunicodechar{⊗}{\otimes} % Digr OX 8855

% cursed
\WarningFilter{newunicodechar}{Redefining Unicode}
\newunicodechar{·}{\ifmmode\cdot\else\textperiodcentered\fi} % Digr .M
\newunicodechar{×}{\ifmmode\times\else\texttimes\fi} % Digr *X
\newunicodechar{→}{\ifmmode\rightarrow\else\textrightarrow\fi} % Digr ->
\newunicodechar{←}{\ifmmode\leftarrow\else\textleftarrow\fi} % Digr ->
\newunicodechar{⟨}{\ifmmode\langle\else\textlangle\fi} % Digr LA 10216 nonstandard
\newunicodechar{⟩}{\ifmmode\rangle\else\textrangle\fi} % Digr RA 10217 nonstandard
\newunicodechar{…}{\ifmmode\dots\else\textellipsis\fi} % Digr .,
\newunicodechar{±}{\ifmmode\pm\else\textpm\fi} % Digr +-

% https://tex.stackexchange.com/a/438184
% https://tex.stackexchange.com/q/528480
\newunicodechar{∶}{\mathbin{\text{:}}}
\def\newcolon{%
  \nobreak\mskip2mu\mathpunct{}\nonscript\mkern-\thinmuskip{\text{:}}%
  \mskip 6mu plus 1 mu \relax}
\mathcode`:="8000
{\catcode`:=\active \global\let:\newcolon}
% colon: for types; ratio∶ (digr :R) for relations (set builder)


\begin{document}

\maketitle

\section*{8/1b}
Załóżmy, że $g: M → N$ jest monomorfizmem.
Udowodnić, że $g(M)$ jest składnikiem prostym modułu $N$
$⇔$
$∃f: N → M,\, f∘g = \id_M$.
% Obraz monomorfizmu jest składnikiem prostym przeciwdziedziny
% wtedy i tylko wtedy, gdy ten homomorfizm jest lewostronnie odwracalny.

\subsection*{$(⇒)$}
Monomorfizm $g$ indukuje izomorfizm $M → g(M)$,
więc mamy też izomorfizm odwrotny $g': g(M) → M$.

Z własności uniwersalnej koproduktu
$g'$ się faktoryzuje na pewne $f: N → M$ i włożenie $i: g(M) → N$, tzn.
$g' = f ∘ i$.
Składając prawostronnie z $g$ otrzymujemy
$\id_M = f ∘ i ∘ g = f ∘ g,$
bo $i$ to w tym wypadku po prostu inkluzja.

\subsection*{$(⇐)$}
Zauważmy, że
$$n = (n - g(f(n))) + g(f(n)),$$
więc $N = \ker f + \im g$.
Jeśli $n ∈ \ker f ∩ \im g$, to $f(n) = 0$ i $n = g(m)$ dla pewnego $m$.
Mamy $0 = f(n) = f(g(m)) = m$, więc $N = \ker f ⊕ \im g$.

\section*{8/2}

\subsection*{(a) $⇒$ (b)}

Załóżmy, że dla każdego epimorfizmu $f: M → N$ dowolnych modułów
$M, N$ i każdego $g: P → N$ istnieje $h: P → M$ takie, że $f ∘ h = g$.

% $P$ projektywny $⇔$ dla każdego epimorfizmu $f: M → P$ mamy $M = \ker f ⊕ M'$
% jeśli $f: M → P$ i $g: P → P$, to $h: P → M$ i $f ∘ h = g$
Weźmy epimorfizm $f: M → P$ i $g = id_P$,
wtedy mamy $h: P → M$ takie, że $f ∘ h = id_P$.
Z 8/1a to oznacza, że $f$ się rozszczepia, zatem $P$ jest projektywny.

\subsection*{(b) $⇒$ (c)}
Załóżmy, że moduł $P$ jest projektywny.
To oznacza, że dla każdego
epimorfizmu $f: M → P$ jądro $f$ jest składnikiem prostym $M$.

Niech $M$ to moduł wolny składający się z formalnych
kombinacji liniowych elementów $P$.
Istnieje epimorfizm $f: M → P$ interpretujący
napis formalny operacjami modułowymi $P$,
więc $M = \ker f ⊕ M'$ dla pewnego podmodułu $M' ⊆ M$.
Z zasadniczego twierdzenia o homomorfizmie mamy
$$P ≅ M/(\ker f) = (\ker f ⊕ M')/(\ker f) ≅ M'.$$

\subsection*{(c) $⇒$ (a)}
Załóżmy, że istnieje moduł $L$ taki, że $P ⊕ L$ jest wolny z bazą $X$.

Niech $p$ to rzut $P ⊕ L → P$.
% rzutowanie p: P ⊕ L → P
% włożenie i: P → P ⊕ L

Weźmy epimorfizm $f: M → N$ i homomorfizm $g: P → N$.
% Ten drugi się rozszerza do $g ∘ p: P ⊕ L$, gdzie $p$ to rzut $P ⊕ L → P$.
% cel: ∃h: P → M, f∘h = g
% cel?: ∃h : P ⊕ L → M,  f∘h = g∘p
Dla każdego $x∈X$ mamy wartość $n=g(p(x))∈N$,
więc istnieje
$m∈M$ taki,
że $f(m) = n$.
To wyznacza pewną funkcję $X → M$,
która się rozszerza do homomorfizmu
$h: P ⊕ L → M$ takiego,
że $$f ∘ h = g ∘ p.$$
Składając prawostronnie z włożeniem $i: P → P ⊕ L$ otrzymujemy
$$f ∘ (h ∘ i) = g ∘ p ∘ i = g,$$
gdzie $h ∘ i: P → M$.

% g ∘ p : P ⊕ L → N   wyznaczony jednoznacznie przez wartości dla bazy
% g ∘ p = g ⊕ 0


\section*{8/4a}

\subsection*{($⇒$)}
Załóżmy, że $M=⊕_{i∈I} M_i$ jest projektywny.
To oznacza, że $M$ jest składnikiem prostym pewnego modułu wolnego,
zatem każdy jego składnik prosty $M_i$ też jest składnikiem prostym modułu wolnego.

\subsection*{($⇐$)}
Załóżmy,
że $M_i$ jest projektywny dla każdego $i ∈ I$.

Weźmy epimorfizm $f: L → N$ i homomorfizm $g: M → N$.
Niech $j_i$ to włożenia $M_i → M$, wtedy
$g ∘ j_i: M_i → N$.
Z projektywności $M_i$ i zad. 8/2a
istnieją $h_i: M_i → L$ takie, że $f∘h_i=g∘j_i$.
Z uniwersalnej własności koproduktu istnieje $h: M → L$ takie,
że $h ∘ j_i = h_i$, więc
$$f∘h∘j_i = g∘j_i.$$
Z uniwersalnej własności koproduktu $φ=f∘h-g$ jest jedynym
homomorfizmem spełniającym
$$φ∘j_i = 0,$$
a więc jest homomorfizmem zerowym i $f∘h=g$.
Zatem $M$ jest projektywny.

\end{document}
