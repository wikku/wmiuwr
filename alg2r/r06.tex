\documentclass[a4paper, 12pt]{article}
\usepackage[utf8]{inputenc}
\usepackage{silence}
\usepackage{polski}
\usepackage{parskip}
\usepackage{amsmath,amsfonts,amssymb,amsthm}
\usepackage{mathtools}
\usepackage{enumitem}
%\usepackage{pgfplots}
%\pgfplotsset{compat=1.16}
\usepackage{newunicodechar}
\usepackage{etoolbox}
\usepackage[margin=1.2in]{geometry}
\usepackage{algorithm}
\usepackage{algorithmicx}
\usepackage{algpseudocode}
\setcounter{secnumdepth}{0}

\title{}
\author{Wiktor Kuchta}
\date{\vspace{-4ex}}

\DeclareMathOperator{\im}{Im}
\DeclareMathOperator{\rank}{rank}
\DeclareMathOperator{\Lin}{Lin}
\DeclareMathOperator{\sgn}{sgn}
\DeclareMathOperator{\Char}{char}
\DeclareMathOperator{\sep}{sep}
\DeclareMathOperator{\rad}{rad}
\newcommand{\N}{\mathbb{N}}
\newcommand{\Z}{\mathbb{Z}}
\newcommand{\Q}{\mathbb{Q}}
\newcommand{\R}{\mathbb{R}}
\newcommand{\C}{\mathbb{C}}
\newcommand{\inner}[2]{( #1 \, | \, #2)}
\newcommand{\norm}[1]{\left\lVert #1 \right\rVert}
\newcommand{\modulus}[1]{\left| #1 \right|}
\newcommand{\abs}{\modulus}
\newtheorem{theorem}{Twierdzenie}
\newtheorem{lemat}{Lemat}
\newcommand{\ol}{\overline}
\DeclareMathOperator{\tr}{tr}
\DeclareMathOperator{\diag}{diag}
\newcommand{\+}{\enspace}
\newcommand{\sump}{\sideset{}{'}{∑}} % sum prime
\newcommand{\sumb}{\sideset{}{"}{∑}} % sum bis

\newunicodechar{∅}{\emptyset} % Digr /0
\newunicodechar{∞}{\infty} % Digr 00
\newunicodechar{∂}{\partial} % Digr dP
\newunicodechar{α}{\alpha}
\newunicodechar{β}{\beta}
\newunicodechar{ξ}{\xi} % Digr c*
\newunicodechar{δ}{\delta} % Digr d*
\newunicodechar{ε}{\varepsilon}
\newunicodechar{φ}{\varphi}
\newunicodechar{θ}{\theta} % Digr h*
\newunicodechar{λ}{\lambda}
\newunicodechar{μ}{\mu}
\newunicodechar{π}{\pi}
\newunicodechar{σ}{\sigma}
\newunicodechar{τ}{\tau}
\newunicodechar{ω}{\omega}
\newunicodechar{η}{\eta} % Digr y*
\newunicodechar{ζ}{\zeta} % Digr z*
\newunicodechar{Δ}{\Delta}
\newunicodechar{Θ}{\Theta}
\newunicodechar{Φ}{\Phi} % Digr F*
\newunicodechar{Π}{\Pi}
\newunicodechar{Ψ}{\Psi} % digr Q*
\newunicodechar{ℕ}{\N} % Digr NN 8469 nonstandard
\newunicodechar{ℤ}{\Z} % Digr ZZ 8484 nonstandard
\newunicodechar{ℚ}{\Q} % Digr QQ 8474 nonstandard
\newunicodechar{ℝ}{\R} % Digr RR 8477 nonstandard
\newunicodechar{ℂ}{\C} % Digr CC 8450 nonstandard
\newunicodechar{∑}{\sum}
\newunicodechar{∏}{\prod}
\newunicodechar{∫}{\int}
\newunicodechar{∓}{\mp}
\newunicodechar{⌈}{\lceil} % Digr <7
\newunicodechar{⌉}{\rceil} % Digr >7
\newunicodechar{⌊}{\lfloor} % Digr 7<
\newunicodechar{⌋}{\rfloor} % Digr 7>
\newunicodechar{≅}{\cong} % Digr ?=
\newunicodechar{≡}{\equiv} % Digr 3=
\newunicodechar{◁}{\triangleleft} % Digr Tl
\newunicodechar{▷}{\triangleright} % Digr Tr
\newunicodechar{≤}{\le}
\newunicodechar{≥}{\ge}
\newunicodechar{≪}{\ll} % Digr <*
\newunicodechar{≫}{\gg} % Digr *>
\newunicodechar{≠}{\ne}
\newunicodechar{⊆}{\subseteq} % Digr (_
\newunicodechar{⊇}{\supseteq} % Digr _)
\newunicodechar{⊂}{\subset} % Digr (C
\newunicodechar{⊃}{\supset} % Digr C)
\newunicodechar{∩}{\cap} % Digr (U
\newunicodechar{∪}{\cup} % Digr )U
\newunicodechar{∼}{\sim} % Digr ?1
\newunicodechar{≈}{\approx} % Digr ?2
\newunicodechar{∈}{\in} % Digr (-
\newunicodechar{∋}{\ni} % Digr -)
\newunicodechar{∇}{\nabla} % Digr NB
\newunicodechar{∃}{\exists} % Digr TE
\newunicodechar{∀}{\forall} % Digr FA
\newunicodechar{∧}{\wedge} % Digr AN
\newunicodechar{∨}{\vee} % Digr OR
\newunicodechar{⊥}{\bot} % Digr -T
\newunicodechar{⊤}{\top} % Digr TO 8868 nonstandard
\newunicodechar{⇒}{\implies} % Digr =>
\newunicodechar{⇐}{\impliedby} % Digr <=
\newunicodechar{⇔}{\iff} % Digr ==
\newunicodechar{↔}{\leftrightarrow} % Digr <>
\newunicodechar{↦}{\mapsto} % Digr T> 8614 nonstandard
\newunicodechar{∘}{\circ} % Digr Ob

% cursed
\WarningFilter{newunicodechar}{Redefining Unicode}
\newunicodechar{·}{\ifmmode\cdot\else\textperiodcentered\fi} % Digr .M
\newunicodechar{×}{\ifmmode\times\else\texttimes\fi} % Digr *X
\newunicodechar{→}{\ifmmode\rightarrow\else\textrightarrow\fi} % Digr ->
\newunicodechar{←}{\ifmmode\leftarrow\else\textleftarrow\fi} % Digr ->
\newunicodechar{⟨}{\ifmmode\langle\else\textlangle\fi} % Digr LA 10216 nonstandard
\newunicodechar{⟩}{\ifmmode\rangle\else\textrangle\fi} % Digr RA 10217 nonstandard
\newunicodechar{…}{\ifmmode\dots\else\textellipsis\fi} % Digr .,
\newunicodechar{±}{\ifmmode\pm\else\textpm\fi} % Digr +-

% https://tex.stackexchange.com/a/438184
% https://tex.stackexchange.com/q/528480
\newunicodechar{∶}{\mathbin{\text{:}}}
\def\newcolon{%
  \nobreak\mskip2mu\mathpunct{}\nonscript\mkern-\thinmuskip{\text{:}}%
  \mskip 6mu plus 1 mu \relax}
\mathcode`:="8000
{\catcode`:=\active \global\let:\newcolon}
% colon: for types; ratio∶ (digr :R) for relations (set builder)


\begin{document}

\maketitle
\section*{4/6}
Załóżmy, że $\Char(K) = p > 0$ oraz $W(X) ∈ K[X]$ jest nierozkładalny
i nierozdzielczy.

Wielomian $W$ ma pewien pierwiastek wielokrotny $a∈\hat{K}$.
Wielomian minimalny $a$ nad $K$ dzieli $W$, ale skoro $W$ jest
nierozdzielczy, to jest równy $W$.
Skoro $a$ jest pierwiastkiem wielokrotnym, to $W'(a) = 0$.
Wielomian $W'$ jest stopnia niższego niż wielomian minimalny $W$,
więc $W=0$.
Jeśli $W(X) = ∑_{i=0}^n k_i X^i$, to $W'(X) = ∑_{i=1}^{n-1} i k_i X^{i-1}$.
Zatem dla każdego $i$ mamy $ik_i = 0$, więc $k_i = 0$ lub $i$ dzieli $p$.
Wynika stąd, że $W∈K[X^p]$.

\section*{4/7}
Załóżmy, że $\Char(K) = p > 0$.

% W_a ma pierwiastki jednokrotne
% Mamy K(a^p) ⊆ K(a)
Załóżmy, że $K(a^p) \subsetneq K(a)$.
Wtedy $x^p-a^p=(x-a)^p$ jest nierozkładalny nad
$K(a^p)$, bo jego czynniki są postaci $(x-a)^k$,
co da się wyrazić w $K(a^p)$ tylko dla $k=p$.
Zatem jest to wielomian minimalny $a$ nad $K(a^p)$
i $a$ jest nierozdzielczy nad $K(a^p)$, więc też nad $K$.

Kontraponując, jeśli $a∈\hat{K}$ jest rozdzielczy nad $K$,
to $K(a) = K(a^p)$.

% W_a/K należy do K(a^p)[X] i jest podzielny przez W_a/K(a^p)

%Cel: a nierozdzielczy nad K

\section*{4/8}
\iffalse
Załóżmy, że $\Char(K) = p > 0$.
Udowodnimy indukcyjnie,
że stopień elementu $a$ radykalnego nad $K$
równa się $\min \{p^n ∶ a^{p^n} ∈ K \}$.

Jeśli stopień elementu radykalnego $a$ nad $K$
to $1 = p^0$, to $W_a(x) = x-a$ należy do $K[X]$,
więc $a$ należy do $K$.

Jeśli stopień elementu radykalnego $a$ nad $K$
to $n > 1$, to $W_a(x) = (x-a)^n$ należy do $K[X]$.
Ten element jest także pierwiastkiem $W'_a(X) = n(X-a)^{n-1}$,
więc z minimalności $p$ dzieli $n=p n_1$.
Skoro $W_a(X) = (X^p-a^p)^{n_1}$ należy do $K[X]$,
to $(X-a^p)^{n_1}$ także i ma on pierwiastek $a^p$.
Wielomian minimalny $a^p$ nad $K$ dzieli ten wielomian,
więc $a$ jest radykalny i stopnia $\deg(a^p/K) ≤ n_1 < n$.
Z założenia indukcyjnego ten stopień to 

\fi
% Wykład05.pdf 7.1.3
Wiemy z wykładu, że jeśli $a$ jest radykalny nad ciałem charakterystyki $p>0$, to
istnieje najmniejsze $n$ takie, że $a^{p^n}$ należy do tego ciała.

Stopień $a$ nad $K$ wynosi co najwyżej $p^n$, bo
$X^{p^n}-a^{p^n} ∈ K[X]$.

Skoro $\sep_{K(a)}(K) \cap \rad_{K(a)}(K) = K$ i $\rad_{K(a)}(K) = K(a)$, to $\sep_{K(a)}(K) = K$.

% Wykład06.pdf ≈Uwaga 7.6
Wiemy z wykładu, że skoro $[K(a) ∶ K] < ∞$, to
$[K(A) ∶ \sep_{K(a)}(K)] = [K(a) ∶ K]$ wynosi $p^k$ dla pewnego $k$.
Z radykalności, $W_a(X) = (X-a)^{p^k} = X^{p^k} - a^{p^k} ∈ K[X]$,
więc $a^{p^k} ∈ K$ i $k=n$.
Zatem $\deg(a/K) = p^n$.

Jeśli rozszerzenie $K⊂L$ jest skończone radykalne,
to $L=K(a_1,…,a_l)$ dla pewnych $a_i$.
Możemy $L$ otrzymać jako ciąg rozszerzeń $K$ o jeden element radykalny:
$$L=K(a_1,…,a_l) ⊃ K(a_1,…,a_{l-1}) ⊃ … ⊃ K(a_1) ⊃ K,$$
gdzie wszystkie stopnie są potęgą $p$, więc się mnożą do potęgi $p$ równej $[L ∶ K]$.


\iffalse
\section*{4/11}
Załóżmy, że liczby $m,n>1$ są względnie pierwsze
i $ζ_n, ζ_m ∈ ℂ$ to pierwiastki pierwotne z $1$ stopni $n,m$ odpowiednio.

Bez straty ogólności załóżmy, że $ζ_i = e^{2πi\frac{1}{n}}.$
Zauważmy, że $ζ_n = ζ_{nm}^{m}, ζ_m = ζ_{nm}^{n}$,
więc z względnej pierwszości $n$ i $m$, dla pewnych $x,y∈ℤ$
mamy $ζ_n^x ζ_m^y = ζ_{nm}^{mx+ny} = ζ_{nm}^{1}$.
Zatem $ℚ(ζ_n,ζ_m) = ℚ(ζ_{nm})$.

$[ℚ(ζ_k) ∶ ℚ] = φ(k)$.

$φ(nm) = [ℚ(ζ_{nm}) ∶ ℚ] ≤ [ℚ(ζ_n) ∶ ℚ][ℚ(ζ_m) ∶ ℚ] = φ(n)φ(m)
\fi


\end{document}
