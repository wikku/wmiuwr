\documentclass[a4paper, 12pt]{article}
\usepackage[utf8]{inputenc}
\usepackage{silence}
\usepackage{polski}
\usepackage{parskip}
\usepackage{amsmath,amsfonts,amssymb,amsthm}
\usepackage{mathtools}
\usepackage{enumitem}
%\usepackage{pgfplots}
%\pgfplotsset{compat=1.16}
\usepackage{newunicodechar}
\usepackage{etoolbox}
\usepackage[margin=1.2in]{geometry}
\usepackage{algorithm}
\usepackage{algorithmicx}
\usepackage{algpseudocode}
\setcounter{secnumdepth}{0}

\title{}
\author{Wiktor Kuchta}
\date{\vspace{-4ex}}

\DeclareMathOperator{\im}{Im}
\DeclareMathOperator{\rank}{rank}
\DeclareMathOperator{\Lin}{Lin}
\DeclareMathOperator{\sgn}{sgn}
\DeclareMathOperator{\Char}{char}
\DeclareMathOperator{\id}{id}
\newcommand{\N}{\mathbb{N}}
\newcommand{\Z}{\mathbb{Z}}
\newcommand{\Q}{\mathbb{Q}}
\newcommand{\R}{\mathbb{R}}
\newcommand{\C}{\mathbb{C}}
\newcommand{\inner}[2]{( #1 \, | \, #2)}
\newcommand{\norm}[1]{\left\lVert #1 \right\rVert}
\newcommand{\modulus}[1]{\left| #1 \right|}
\newcommand{\abs}{\modulus}
\newtheorem{theorem}{Twierdzenie}
\newtheorem{lemat}{Lemat}
\newcommand{\ol}{\overline}
\DeclareMathOperator{\tr}{tr}
\DeclareMathOperator{\diag}{diag}
\DeclareMathOperator{\End}{End}
\newcommand{\+}{\enspace}
\newcommand{\sump}{\sideset{}{'}{∑}} % sum prime
\newcommand{\sumb}{\sideset{}{"}{∑}} % sum bis

\newunicodechar{∅}{\emptyset} % Digr /0
\newunicodechar{∞}{\infty} % Digr 00
\newunicodechar{∂}{\partial} % Digr dP
\newunicodechar{α}{\alpha}
\newunicodechar{β}{\beta}
\newunicodechar{ξ}{\xi} % Digr c*
\newunicodechar{δ}{\delta} % Digr d*
\newunicodechar{ε}{\varepsilon}
\newunicodechar{φ}{\varphi}
\newunicodechar{θ}{\theta} % Digr h*
\newunicodechar{λ}{\lambda}
\newunicodechar{μ}{\mu}
\newunicodechar{π}{\pi}
\newunicodechar{ψ}{\psi}
\newunicodechar{ρ}{\rho}
\newunicodechar{σ}{\sigma}
\newunicodechar{τ}{\tau}
\newunicodechar{ω}{\omega}
\newunicodechar{η}{\eta} % Digr y*
\newunicodechar{ζ}{\zeta} % Digr z*
\newunicodechar{Δ}{\Delta}
\newunicodechar{Γ}{\Gamma}
\newunicodechar{Λ}{\Lambda}
\newunicodechar{Θ}{\Theta}
\newunicodechar{Φ}{\Phi} % Digr F*
\newunicodechar{Π}{\Pi}
\newunicodechar{Ψ}{\Psi} % digr Q*
\newunicodechar{Σ}{\Sigma} % digr S*
\newunicodechar{Ω}{\Omega} % digr W*
\newunicodechar{ℕ}{\N} % Digr NN 8469 nonstandard
\newunicodechar{ℤ}{\Z} % Digr ZZ 8484 nonstandard
\newunicodechar{ℚ}{\Q} % Digr QQ 8474 nonstandard
\newunicodechar{ℝ}{\R} % Digr RR 8477 nonstandard
\newunicodechar{ℂ}{\C} % Digr CC 8450 nonstandard
\newunicodechar{∑}{\sum}
\newunicodechar{∏}{\prod}
\newunicodechar{∫}{\int}
\newunicodechar{∓}{\mp}
\newunicodechar{⌈}{\lceil} % Digr <7
\newunicodechar{⌉}{\rceil} % Digr >7
\newunicodechar{⌊}{\lfloor} % Digr 7<
\newunicodechar{⌋}{\rfloor} % Digr 7>
\newunicodechar{≅}{\cong} % Digr ?=
\newunicodechar{≡}{\equiv} % Digr 3=
\newunicodechar{◁}{\triangleleft} % Digr Tl
\newunicodechar{▷}{\triangleright} % Digr Tr
\newunicodechar{≤}{\le}
\newunicodechar{≥}{\ge}
\newunicodechar{≪}{\ll} % Digr <*
\newunicodechar{≫}{\gg} % Digr *>
\newunicodechar{≠}{\ne}
\newunicodechar{⊆}{\subseteq} % Digr (_
\newunicodechar{⊇}{\supseteq} % Digr _)
\newunicodechar{⊂}{\subset} % Digr (C
\newunicodechar{⊃}{\supset} % Digr C)
\newunicodechar{∩}{\cap} % Digr (U
\newunicodechar{∖}{\setminus} % Digr -\ 8726 nonstandard
\newunicodechar{∪}{\cup} % Digr )U
\newunicodechar{∼}{\sim} % Digr ?1
\newunicodechar{≈}{\approx} % Digr ?2
\newunicodechar{∈}{\in} % Digr (-
\newunicodechar{∋}{\ni} % Digr -)
\newunicodechar{∇}{\nabla} % Digr NB
\newunicodechar{∃}{\exists} % Digr TE
\newunicodechar{∀}{\forall} % Digr FA
\newunicodechar{∧}{\wedge} % Digr AN
\newunicodechar{∨}{\vee} % Digr OR
\newunicodechar{⊥}{\bot} % Digr -T
\newunicodechar{⊢}{\vdash} % Digr \- 8866 nonstandard
\newunicodechar{⊤}{\top} % Digr TO 8868 nonstandard
\newunicodechar{⇒}{\implies} % Digr =>
\newunicodechar{⊸}{\multimap} % Digr #> nonstandard
\newunicodechar{⇐}{\impliedby} % Digr <=
\newunicodechar{⇔}{\iff} % Digr ==
\newunicodechar{↔}{\leftrightarrow} % Digr <>
\newunicodechar{↦}{\mapsto} % Digr T> 8614 nonstandard
\newunicodechar{∘}{\circ} % Digr Ob
\newunicodechar{⊕}{\oplus} % Digr O+ 8853
\newunicodechar{⊗}{\otimes} % Digr OX 8855

% cursed
\WarningFilter{newunicodechar}{Redefining Unicode}
\newunicodechar{·}{\ifmmode\cdot\else\textperiodcentered\fi} % Digr .M
\newunicodechar{×}{\ifmmode\times\else\texttimes\fi} % Digr *X
\newunicodechar{→}{\ifmmode\rightarrow\else\textrightarrow\fi} % Digr ->
\newunicodechar{←}{\ifmmode\leftarrow\else\textleftarrow\fi} % Digr ->
\newunicodechar{⟨}{\ifmmode\langle\else\textlangle\fi} % Digr LA 10216 nonstandard
\newunicodechar{⟩}{\ifmmode\rangle\else\textrangle\fi} % Digr RA 10217 nonstandard
\newunicodechar{…}{\ifmmode\dots\else\textellipsis\fi} % Digr .,
\newunicodechar{±}{\ifmmode\pm\else\textpm\fi} % Digr +-

% https://tex.stackexchange.com/a/438184
% https://tex.stackexchange.com/q/528480
\newunicodechar{∶}{\mathbin{\text{:}}}
\def\newcolon{%
  \nobreak\mskip2mu\mathpunct{}\nonscript\mkern-\thinmuskip{\text{:}}%
  \mskip 6mu plus 1 mu \relax}
\mathcode`:="8000
{\catcode`:=\active \global\let:\newcolon}
% colon: for types; ratio∶ (digr :R) for relations (set builder)


\begin{document}

\maketitle

\iffalse
\section*{7/1}
Załóżmy, że $a_0, …, a_{n-1} ∈ ℝ$ są algebraicznie niezależne (nad $ℚ$).
Udowodnić, że wielomian
$$W(X) = X^n + a_{n-1}X^{n-1} + … + a_1X + a_0$$
jest nierozkładalny nad ciałem $ℚ(a_0,…,a_{n-1})$.

\section*{7/2}
Załóżmy, że $a,b,c ∈ ℝ$ są algebraicznie niezależne.
To oznacza, że homomorfizm ewaluacji w $(a,b,c)$ ma trywialne jądro.

Niech $g_1(X) =
Algebraiczna niezależność $a^2+b+c, b^2c, ab+ac^2$ oznacza,
że
ewaluacja $
\fi

\section*{7/4bD}
Pojęcie $ℤ$-modułu jest równoważne pojęciu grupy abelowej,
więc $ℤ$-moduły proste to grupy abelowe
bez nietrywialnych podgrup właściwych.
Takie grupy muszą być równe podgrupie generowanej
przez dowolny element niezerowy, czyli są cykliczne.
Grupa $ℤ$ ma nietrywialne podgrupy właściwe, zatem $ℤ$-moduły proste
to grupy cykliczne rzędu pierwszego.

Homomorfizm z grupy cyklicznej jest wyznaczony przez wartość dla $1$.
Dla każdej wartości w $ℤ/pℤ$ otrzymujemy inny endomorfizm $ℤ/pℤ$.
Czyli pierścień endomorfizmów $ℤ/pℤ$ jest mocy $p$, więc z tw.
Wedderburna jest izomorficzny z ciałem $F(p)$.

\section*{7/5D}
Niech $V$ to niezerowa przestrzeń liniowa nad ciałem $K$.
Ta przestrzeń ma pewną bazę $\mathcal{B}$ i każdy element możemy utożsamić
z jego współrzędnymi w tej bazie (indeksowanymi elementami bazy).

Jest to oczywiście $\End_K(V)$-moduł z operacją stosowania endomorfizmu na
wektorze.
To, że $\End_K(V)$ jest pierścieniem, wynika z nietrywialności $V$.

Taki endomorfizm jest jednoznacznie wyznaczony przez wartości dla elementów
$\mathcal{B}$.
Z kolei taka wartość to dowolna kombinacja liniowa współrzędnych, tzn.
należy do wolnej grupy abelowej $V ≅ \bigoplus_{b∈\mathcal{B}} K$.
Czyli pierścień $\End_K(V)$ jest izomorficzny z $V^{\mathcal{B}}$.

Załóżmy nie wprost, że mamy nietrywialny podmoduł właściwy $V' < V$ jako
$\End_K(V)$-moduł,
czyli podprzestrzeń liniową niezmienniczą na wszystkie endomorfizmy $V$.
Wtedy przestrzeń $V'$ jest rozpinana przez pewnien układ wektorów $\mathcal{C}$,
który się rozszerza do $\mathcal{D}$ bazy $V$.
Elementy $V'$ mają zerowe współrzędne dla wektorów z
$\mathcal{D} ∖ \mathcal{C}$.
Istnieje endomorfizm $V$, który wektor z $\mathcal{C}$ przekształca
na wektor w $\mathcal{D} ∖ \mathcal{C}$, więc $V'$ nie jest zamknięte
na mnożenie przez skalar.
Sprzeczność.

\newpage
\section*{7/6D}
Załóżmy, że $M$ jest $R$-modułem.
Wówczas $M$ jest też $R'$-modułem, gdzie $R' = \End_R(M)$.
Dla $r ∈ R$ definiujemy $f_r: M → M$ przez $f_r(m) = rm$.
%Zauważmy, że $f_r ∈ \End_R(M)$.

Niech $r', s' ∈ \End_R(M)=R'$ i $m_1,m_2 ∈ M$.
\begin{align*}
f_r(r'(m_1) + s'(m_2))
	&= r(r'(m_1) + s'(m_2)) &\text{definicja }f_r \\
	&= rr'(m_1) + rs'(m_2) &\text{rozdzielność mnożenia skalarnego} \\
	&= r'(r m_1) + s'(r m_2) &\text{$R$-liniowość} \\
	&= r'(f_r(m_1)) + s'(f_r(m_2)) &\text{definicja }f_r
\end{align*}
Zatem $f_r$ jest endomorfizmem $R'$-modułu $M$ (jest funkcją $R'$-liniową).

\end{document}
