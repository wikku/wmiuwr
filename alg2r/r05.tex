\documentclass[a4paper, 12pt]{article}
\usepackage[utf8]{inputenc}
\usepackage{silence}
\usepackage{polski}
\usepackage{parskip}
\usepackage{amsmath,amsfonts,amssymb,amsthm}
\usepackage{mathtools}
\usepackage{enumitem}
%\usepackage{pgfplots}
%\pgfplotsset{compat=1.16}
\usepackage{newunicodechar}
\usepackage{etoolbox}
\usepackage[margin=1.2in]{geometry}
\usepackage{algorithm}
\usepackage{algorithmicx}
\usepackage{algpseudocode}
\setcounter{secnumdepth}{0}

\title{Algebra 2R, lista 4}
\author{Wiktor Kuchta}
\date{\vspace{-4ex}}

\DeclareMathOperator{\im}{Im}
\DeclareMathOperator{\rank}{rank}
\DeclareMathOperator{\Lin}{Lin}
\DeclareMathOperator{\sgn}{sgn}
\DeclareMathOperator{\Char}{char}
\newcommand{\N}{\mathbb{N}}
\newcommand{\Z}{\mathbb{Z}}
\newcommand{\Q}{\mathbb{Q}}
\newcommand{\R}{\mathbb{R}}
\newcommand{\C}{\mathbb{C}}
\newcommand{\inner}[2]{( #1 \, | \, #2)}
\newcommand{\norm}[1]{\left\lVert #1 \right\rVert}
\newcommand{\modulus}[1]{\left| #1 \right|}
\newcommand{\abs}{\modulus}
\newtheorem{theorem}{Twierdzenie}
\newtheorem{lemat}{Lemat}
\newcommand{\ol}{\overline}
\DeclareMathOperator{\tr}{tr}
\DeclareMathOperator{\diag}{diag}
\newcommand{\+}{\enspace}
\newcommand{\sump}{\sideset{}{'}{∑}} % sum prime
\newcommand{\sumb}{\sideset{}{"}{∑}} % sum bis

\newunicodechar{∅}{\emptyset} % Digr /0
\newunicodechar{∞}{\infty} % Digr 00
\newunicodechar{∂}{\partial} % Digr dP
\newunicodechar{α}{\alpha}
\newunicodechar{β}{\beta}
\newunicodechar{ξ}{\xi} % Digr c*
\newunicodechar{δ}{\delta} % Digr d*
\newunicodechar{ε}{\varepsilon}
\newunicodechar{φ}{\varphi}
\newunicodechar{θ}{\theta} % Digr h*
\newunicodechar{λ}{\lambda}
\newunicodechar{μ}{\mu}
\newunicodechar{π}{\pi}
\newunicodechar{σ}{\sigma}
\newunicodechar{τ}{\tau}
\newunicodechar{ω}{\omega}
\newunicodechar{η}{\eta} % Digr y*
\newunicodechar{Δ}{\Delta}
\newunicodechar{Θ}{\Theta}
\newunicodechar{Φ}{\Phi} % Digr F*
\newunicodechar{Π}{\Pi}
\newunicodechar{Ψ}{\Psi} % digr Q*
\newunicodechar{ℕ}{\N} % Digr NN 8469 nonstandard
\newunicodechar{ℤ}{\Z} % Digr ZZ 8484 nonstandard
\newunicodechar{ℚ}{\Q} % Digr QQ 8474 nonstandard
\newunicodechar{ℝ}{\R} % Digr RR 8477 nonstandard
\newunicodechar{ℂ}{\C} % Digr CC 8450 nonstandard
\newunicodechar{∑}{\sum}
\newunicodechar{∏}{\prod}
\newunicodechar{∫}{\int}
\newunicodechar{∓}{\mp}
\newunicodechar{⌈}{\lceil} % Digr <7
\newunicodechar{⌉}{\rceil} % Digr >7
\newunicodechar{⌊}{\lfloor} % Digr 7<
\newunicodechar{⌋}{\rfloor} % Digr 7>
\newunicodechar{≅}{\cong} % Digr ?=
\newunicodechar{≡}{\equiv} % Digr 3=
\newunicodechar{◁}{\triangleleft} % Digr Tl
\newunicodechar{▷}{\triangleright} % Digr Tr
\newunicodechar{≤}{\le}
\newunicodechar{≥}{\ge}
\newunicodechar{≪}{\ll} % Digr <*
\newunicodechar{≫}{\gg} % Digr *>
\newunicodechar{≠}{\ne}
\newunicodechar{⊆}{\subseteq} % Digr (_
\newunicodechar{⊇}{\supseteq} % Digr _)
\newunicodechar{⊂}{\subset} % Digr (C
\newunicodechar{⊃}{\supset} % Digr C)
\newunicodechar{∩}{\cap} % Digr (U
\newunicodechar{∪}{\cup} % Digr )U
\newunicodechar{∼}{\sim} % Digr ?1
\newunicodechar{≈}{\approx} % Digr ?2
\newunicodechar{∈}{\in} % Digr (-
\newunicodechar{∋}{\ni} % Digr -)
\newunicodechar{∇}{\nabla} % Digr NB
\newunicodechar{∃}{\exists} % Digr TE
\newunicodechar{∀}{\forall} % Digr FA
\newunicodechar{∧}{\wedge} % Digr AN
\newunicodechar{∨}{\vee} % Digr OR
\newunicodechar{⊥}{\bot} % Digr -T
\newunicodechar{⊤}{\top} % Digr TO 8868 nonstandard
\newunicodechar{⇒}{\implies} % Digr =>
\newunicodechar{⇐}{\impliedby} % Digr <=
\newunicodechar{⇔}{\iff} % Digr ==
\newunicodechar{↔}{\leftrightarrow} % Digr <>
\newunicodechar{↦}{\mapsto} % Digr T> 8614 nonstandard
\newunicodechar{∘}{\circ} % Digr Ob

% cursed
\WarningFilter{newunicodechar}{Redefining Unicode}
\newunicodechar{·}{\ifmmode\cdot\else\textperiodcentered\fi} % Digr .M
\newunicodechar{×}{\ifmmode\times\else\texttimes\fi} % Digr *X
\newunicodechar{→}{\ifmmode\rightarrow\else\textrightarrow\fi} % Digr ->
\newunicodechar{←}{\ifmmode\leftarrow\else\textleftarrow\fi} % Digr ->
\newunicodechar{⟨}{\ifmmode\langle\else\textlangle\fi} % Digr LA 10216 nonstandard
\newunicodechar{⟩}{\ifmmode\rangle\else\textrangle\fi} % Digr RA 10217 nonstandard
\newunicodechar{…}{\ifmmode\dots\else\textellipsis\fi} % Digr .,
\newunicodechar{±}{\ifmmode\pm\else\textpm\fi} % Digr +-

% https://tex.stackexchange.com/a/438184
% https://tex.stackexchange.com/q/528480
\newunicodechar{∶}{\mathbin{\text{:}}}
\def\newcolon{%
  \nobreak\mskip2mu\mathpunct{}\nonscript\mkern-\thinmuskip{\text{:}}%
  \mskip 6mu plus 1 mu \relax}
\mathcode`:="8000
{\catcode`:=\active \global\let:\newcolon}
% colon: for types; ratio∶ (digr :R) for relations (set builder)


\begin{document}

\maketitle

\section*{4/1}
Pierwiastki $F_m$ to dokładnie $m$-te pierwiastki pierwotne z jedynki,
tzn.
$$F_m(x) = ∏_{\substack{1≤k≤m \\ \gcd(k,m)=1}} \left(x-\exp(2πi\frac{k}{m})\right).$$

Łatwo sprawdzić, że
\begin{align*}
	F_1(x) &= x-1, \\
	F_2(x) &= x+1, \\
	F_3(x) &= (x-\exp(\tfrac{4}{3}πi))(x-\exp(-\tfrac{4}{3}πi))
		= x^2 - 2x\cos \tfrac{4π}{3} + 1 \\
		&= x^2 + x + 1.
\end{align*}

Natomiast dla $m>1$
$$F_{2^m}(x) = ∏_{1≤k≤2^{m-1}} \left(x-\exp(2πi\frac{2k-1}{2^m})\right),$$
bo $\{1, 3, …, 2^m-1\}$ to wszystkie liczby naturalne $≤2^m$ względnie pierwsze z $2^m$.
Zauważmy, że
$$ \exp(2πi\frac{2k-1}{2^m})^{2^{m-1}} = \exp(πi\frac{2k-1}{2^{m-1}}2^{m-1})
= \exp(πi(2k-1)) = \exp(-πi) = -1,$$
czyli $2^m$-te pierwiastki pierwotne z $1$ to $2^{m-1}$-e pierwiastki z $-1$.
Jest ich $2^{m-1}$, więc są to wszystkie takie pierwiastki i $F_{2^m}(x) = x^{2^{m-1}}+1$.
W szczególności
\begin{align*}
	F_4(x) = x^2+1, \\
	F_8(x) = x^4+1, \\
	F_{16}(x) = x^8+1.
\end{align*}

Ze wzoru
$$W_m(x) = x^m-1 = ∏_{d|m} F_d(x)$$
otrzymujemy
\begin{align*}
F_{15}(x)
&= \frac{x^{15}-1}{F_1(x)F_3(x)F_5(x)}
= \frac{(x^{5})^3-1^3}{F_3(x)(x^5-1)}
= \frac{x^{10}+x^5+1}{x^2+x+1} \\
&= x^8-x^7+x^5-x^4+x^3-x+1.
\end{align*}

Dla homomorfizmu ilorazowego $j: ℤ → ℤ_3$,
$j(F_m)$ to wielomian $F_m$, w którym nałożono $j$ na współczynniki.

Wielomiany $F_1$ i $F_2$ są liniowe, a więc nierozkładalne.
Można sprawdzić, że $F_4(x) = x^2+1$ nie ma pierwiastków w $ℤ_3$, a
może się rozłożyć tylko na czynniki liniowe, więc jest nierozkładalny.

Niech $a$ to $2^m$-ty pierwotny pierwiastek z jedynki w pewnym
rozszerzeniu $ℤ_3$.
Element $a$ generuje rozszerzenie stopnia $n$, gdzie $n$ to najmniejsza
liczba taka, że $2^m | 3^n - 1$.
Dla $m=3$ jest to $n=2$, dla $m=4$ jest to $n=4$.
W obu przypadkach wielomian minimalny $a$ nad $ℤ_3$ ma stopień mniejszy
od stopnia $F_{2^m}$, więc są one rozkładalne.

Mamy $x^{15}-1 = (x^5-1)^3$, więc pierwiastki $F_{15}$ w $\hat{ℤ_3}$
są pierwiastkami $x^5-1$.
Zatem wielomian minimalny każdego pierwiastka $F_{15}$ nad $ℤ_3$
ma stopień co najwyżej $5 < \deg F_{15} = 8$.
Możemy podzielić $F_{15}$ przez ten wielomian minimalny, otrzymując
nietrywialny rozkład $F_{15}$ nad $ℤ_3$.


% F_n nierozkładalny ℤ₃ ⇔ 3 jest generatorem ℤ_n^* (?)

\section*{4/3}
Załóżmy, że $[L ∶ K] = 2$, $a ∈ L\setminus K$.
Mamy $[L ∶ K(a)][K(a) ∶ K] = 2$ i stąd $[K(a) ∶ K] = 2$,
bo inaczej $K(a)$ byłoby równe $K$.
Zatem $L = K(a)$ jest algebraicznym rozszerzeniem $K$.
Element $a$ ma wielomian minimalny $W_a$ stopnia $2$ nad $K$.
Ale $W_a(x) = (x-a)g(x)$, gdzie $g$ jest liniowy z pierwiastkiem w $L$,
więc $W_a$ rozkłada się na czynniki liniowe.
Zatem z rozszerzenie $K ⊆ L$ jest normalne.


\iffalse
\section*{4/4}
Załóżmy, że rozszerzenie $K⊂L$ jest algebraiczne i $f: L→L$ jest monomorfizmem,
$f|_K = id$.

Weźmy $α∈L$,
jest to pierwiastek pewnego wielomianu $W=∑_{i=0}^n k_iX^i$, gdzie $k_i ∈ K$.
Zatem
$$0=f(0)=f(W(α)) = f(∑_{i=0}^n k_iα^i) = ∑_{i=0}^n f(k_iα^i) = ∑_{i=0}^n k_if(α)^i = W(f(α)),$$
czyli $f(α)$ też jest pierwiastkiem tego wielomianu.
Takich pierwiastków jest skończenie wiele, a funkcja $f$ jest iniektywna,
więc je permutuje.
W szczególności $α$ jest wartością $f$ dla jakiegoś pierwiastka $W$.

Zatem monomorizm $f$ jest surjektywny.
% każdy l∈L jest pierwiastkiem
\fi



\section*{4/5}
Załóżmy, że $K ⊂ L ⊂ \hat{K}$ i rozszerzenie $K ⊂ L$ jest radykalne.
% dla każdego a∈L wielomian minimalny W_a ma tylko jeden pierwiastek
% (wielokrotny) w \hat{K}

Każdy homomorfizm, który jest identycznością na $L$, jest w szczególności
identycznością na $K$, więc $G(\hat{K}/K) ⊇ G(\hat{K}/L)$.

Każdy automorfizm $f: \hat{K} → \hat{K}$ stały na $K$ permutuje
pierwiastki wielomianów z $K[X]$.
Wielomian minimalny nad $K$ elementu radykalnego nad $K$
ma tylko jeden pierwiastek, więc jest on zachowywany przez $f$.
Każdy element $L$ jest radykalny nad $K$,
więc każdy taki $f$ jest identycznością na $L$.
Zatem $G(\hat{K}/K) ⊆ G(\hat{K}/L)$.

\end{document}
