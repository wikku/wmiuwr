\documentclass[a4paper, 12pt]{article}
\usepackage[utf8]{inputenc}
\usepackage{silence}
\usepackage{polski}
\usepackage{parskip}
\usepackage{amsmath,amsfonts,amssymb,amsthm}
\usepackage{mathtools}
\usepackage{enumitem}
%\usepackage{pgfplots}
%\pgfplotsset{compat=1.16}
\usepackage{newunicodechar}
\usepackage{etoolbox}
\usepackage[margin=1.2in]{geometry}
\usepackage{algorithm}
\usepackage{algorithmicx}
\usepackage{algpseudocode}
\setcounter{secnumdepth}{0}

\title{Algebra 2R lista 2}
\author{Wiktor Kuchta}
\date{\vspace{-4ex}}

\DeclareMathOperator{\im}{Im}
\DeclareMathOperator{\rank}{rank}
\DeclareMathOperator{\Lin}{Lin}
\DeclareMathOperator{\sgn}{sgn}
\DeclareMathOperator{\Char}{char}
\newcommand{\N}{\mathbb{N}}
\newcommand{\Z}{\mathbb{Z}}
\newcommand{\Q}{\mathbb{Q}}
\newcommand{\R}{\mathbb{R}}
\newcommand{\C}{\mathbb{C}}
\newcommand{\inner}[2]{( #1 \, | \, #2)}
\newcommand{\norm}[1]{\left\lVert #1 \right\rVert}
\newcommand{\modulus}[1]{\left| #1 \right|}
\newcommand{\abs}{\modulus}
\newtheorem{theorem}{Twierdzenie}
\newtheorem{lemat}{Lemat}
\newcommand{\ol}{\overline}
\DeclareMathOperator{\tr}{tr}
\DeclareMathOperator{\diag}{diag}
\newcommand{\+}{\enspace}
\newcommand{\sump}{\sideset{}{'}{∑}} % sum prime
\newcommand{\sumb}{\sideset{}{"}{∑}} % sum bis

\newunicodechar{∅}{\emptyset} % Digr /0
\newunicodechar{∞}{\infty} % Digr 00
\newunicodechar{∂}{\partial} % Digr dP
\newunicodechar{α}{\alpha}
\newunicodechar{β}{\beta}
\newunicodechar{ξ}{\xi} % Digr c*
\newunicodechar{δ}{\delta} % Digr d*
\newunicodechar{ε}{\varepsilon}
\newunicodechar{φ}{\varphi}
\newunicodechar{θ}{\theta} % Digr h*
\newunicodechar{λ}{\lambda}
\newunicodechar{μ}{\mu}
\newunicodechar{π}{\pi}
\newunicodechar{σ}{\sigma}
\newunicodechar{τ}{\tau}
\newunicodechar{ω}{\omega}
\newunicodechar{η}{\eta} % Digr y*
\newunicodechar{Δ}{\Delta}
\newunicodechar{Θ}{\Theta}
\newunicodechar{Φ}{\Phi} % Digr F*
\newunicodechar{Π}{\Pi}
\newunicodechar{Ψ}{\Psi} % digr Q*
\newunicodechar{ℕ}{\N} % Digr NN 8469 nonstandard
\newunicodechar{ℤ}{\Z} % Digr ZZ 8484 nonstandard
\newunicodechar{ℚ}{\Q} % Digr QQ 8474 nonstandard
\newunicodechar{ℝ}{\R} % Digr RR 8477 nonstandard
\newunicodechar{ℂ}{\C} % Digr CC 8450 nonstandard
\newunicodechar{∑}{\sum}
\newunicodechar{∏}{\prod}
\newunicodechar{∫}{\int}
\newunicodechar{∓}{\mp}
\newunicodechar{⌈}{\lceil} % Digr <7
\newunicodechar{⌉}{\rceil} % Digr >7
\newunicodechar{⌊}{\lfloor} % Digr 7<
\newunicodechar{⌋}{\rfloor} % Digr 7>
\newunicodechar{≅}{\cong} % Digr ?=
\newunicodechar{≡}{\equiv} % Digr 3=
\newunicodechar{◁}{\triangleleft} % Digr Tl
\newunicodechar{▷}{\triangleright} % Digr Tr
\newunicodechar{≤}{\le}
\newunicodechar{≥}{\ge}
\newunicodechar{≪}{\ll} % Digr <*
\newunicodechar{≫}{\gg} % Digr *>
\newunicodechar{≠}{\ne}
\newunicodechar{⊆}{\subseteq} % Digr (_
\newunicodechar{⊇}{\supseteq} % Digr _)
\newunicodechar{⊂}{\subset} % Digr (C
\newunicodechar{⊃}{\supset} % Digr C)
\newunicodechar{∩}{\cap} % Digr (U
\newunicodechar{∪}{\cup} % Digr )U
\newunicodechar{∼}{\sim} % Digr ?1
\newunicodechar{≈}{\approx} % Digr ?2
\newunicodechar{∈}{\in} % Digr (-
\newunicodechar{∋}{\ni} % Digr -)
\newunicodechar{∇}{\nabla} % Digr NB
\newunicodechar{∃}{\exists} % Digr TE
\newunicodechar{∀}{\forall} % Digr FA
\newunicodechar{∧}{\wedge} % Digr AN
\newunicodechar{∨}{\vee} % Digr OR
\newunicodechar{⊥}{\bot} % Digr -T
\newunicodechar{⊤}{\top} % Digr TO 8868 nonstandard
\newunicodechar{⇒}{\implies} % Digr =>
\newunicodechar{⇐}{\impliedby} % Digr <=
\newunicodechar{⇔}{\iff} % Digr ==
\newunicodechar{↔}{\leftrightarrow} % Digr <>
\newunicodechar{↦}{\mapsto} % Digr T> 8614 nonstandard
\newunicodechar{∘}{\circ} % Digr Ob

% cursed
\WarningFilter{newunicodechar}{Redefining Unicode}
\newunicodechar{·}{\ifmmode\cdot\else\textperiodcentered\fi} % Digr .M
\newunicodechar{×}{\ifmmode\times\else\texttimes\fi} % Digr *X
\newunicodechar{→}{\ifmmode\rightarrow\else\textrightarrow\fi} % Digr ->
\newunicodechar{←}{\ifmmode\leftarrow\else\textleftarrow\fi} % Digr ->
\newunicodechar{⟨}{\ifmmode\langle\else\textlangle\fi} % Digr LA 10216 nonstandard
\newunicodechar{⟩}{\ifmmode\rangle\else\textrangle\fi} % Digr RA 10217 nonstandard
\newunicodechar{…}{\ifmmode\dots\else\textellipsis\fi} % Digr .,
\newunicodechar{±}{\ifmmode\pm\else\textpm\fi} % Digr +-

% https://tex.stackexchange.com/a/438184
% https://tex.stackexchange.com/q/528480
\newunicodechar{∶}{\mathbin{\text{:}}}
\def\newcolon{%
  \nobreak\mskip2mu\mathpunct{}\nonscript\mkern-\thinmuskip{\text{:}}%
  \mskip 6mu plus 1 mu \relax}
\mathcode`:="8000
{\catcode`:=\active \global\let:\newcolon}
% colon: for types; ratio∶ (digr :R) for relations (set builder)


\begin{document}

\maketitle

\iffalse
\section*{1/5d}
\begin{align*}
 & w ∈ ℚ[X_1, X_2], w(X^3, X^2) = 0 \\
⇒& w ∈ ℚ[X_2][X_1], w(X^3)(X^2) = 0 \\
⇒& v=w(X^3) ∈ ℚ[X_2][X^3] ≅ ℚ[X^3][X_2], v(X^2) = 0 \\
⇒& v=(X^2-X_2)q(X_2), \\
 &\text{w iloczynie }X\text{ występuje tylko w potędze podzielnej przez }3 \\
\end{align*}
$w ∈ ℚ[X_1, X_2], w(X^3, X^2)$

$w ∈ ℚ[X_1][X_2]$
$w(X^2) ∈ ℚ[X_1][X^2] ≅ ℚ[X^2][X_1]$

%wielomian ∈ Q[X^3][X_2]
\fi

\section*{2/3D}
Załóżmy, że $f: K→K$ jest niezerowym endomorfizmem ciała $K$.
Niech $Fix(f) = \{ x ∈ K ∶ f(x) = x \}$.

Jeśli $x,y ∈ Fix(f)$, tzn. $f(x) = x$ i $f(y) = y$, to z homomorficzności:
\begin{itemize}
	\item $f(0) = 0$,
	\item $f(1) = 1$,
	\item $f(-x) = -f(x) = -x$,
	\item $f(x^{-1}) = (f(x))^{-1} = x^{-1}$,
	\item $f(x+y) = f(x)+f(y)=x+y$,
	\item $f(xy) = f(x)f(y) = xy$,
\end{itemize}
więc $Fix(f)$ zawiera $0$ i $1$ oraz jest zamknięte na
negację, odwracanie, dodawanie i mnożenie.
Zatem $Fix(f)$ jest podciałem $K$.



\section*{2/4}
Załóżmy, że $K$ jest ciałem skończonym charakterystyki $p$.

\subsection*{(a)D}
Niech $K=F(q)$, $q = p^k$, $f ∈ K[X]$ to wielomian nierozkładalny stopnia $m$.

Wtedy stopień rozszerzenia $K$ o pierwiastek $a$ wynosi
$[K(a) ∶ K] = m$, więc $a ∈ K(a) ≅ F(q^m)$.
% tu coś może zepsułem :)
Ciało rozkładu $f$ nad $K$ jest rozszerzeniem $K[X]/(f)$
(takie ciało mamy po pierwszym kroku konstrukcji ciała rozkładu).
Rozszerzenie z $K$ do $K[X]/(f)$ jest stopnia $m$.
Wiemy, %z wykładu?
że potęgi $a^k$ dla $0 ≤ k < m$ są liniowo niezależne,
więc $K(a)$ jest stopnia $m$ nad $K$
i z jedyności ciał skończonych o danej mocy $K(a) ≅ K[X]/(f)$.

% tu może coś zepsułem :)
W $F(q^m)$ dla każdego $x≠0$ zachodzi $x^{q^m-1} = 1$,
więc $X^{q^m-1} - 1 ∈ (f)$.
Zatem $f$ dzieli $X^{q^m-1} -1$, gdzie $q^m-1 ≡ p^{mk}-1 \not≡ 0 \pmod{p}$.

\iffalse
Niech $f ∈ K[X]$ to wielomian nierozkładalny i
$L$ to jego ciało rozkładu nad $K$.

Konstrukcja ciała rozkładu $L$ polega na rozszerzaniu ciała
$K(a_1, …, a_{i-1})$ o pierwiastek $a_i$ czynnika nierozkładalnego $f$,
więc wiemy w szczególności, że
$$[K(a_1, …, a_{i-1})(a_i) ∶ K(a_1, …, a_{i-1})] < ∞.$$
Zatem stopień rozszerzenia
$$[L∶K] = [K(a_1, …, a_{r-1})(a_r) ∶ K(a_1, …, a_{r-1})]…[K(a_1) ∶ K]$$
jest skończony i ciało $L$ jest skończone.

Skoro $f = c(X-a_1)…(X-a_r)$ jest nierozkładalny w $K[X]$, to pierwiastki $a_i$ są
niezerowe.
W skończonym ciele $L$ oznacza to, że są one pierwiastkami z jedynki.
Żaden z nich nie może mieć najmniejszej krotności $kp$ ($p \nmid k$), bo
jeśli $0 = x^{kp} - 1 = (x^k-1)^p$, to $x$ ma krotność $k<p$.

Niech $n$ to NWW pierwotnych krotności pierwiastków z jedynki $a_i$,
ta liczba jest niepodzielna przez $p$.
Każdy pierwiastek $f$ jest pierwiastkiem $W_n$, więc $f$ dzieli $W_n$.
\fi

\subsection*{(b)}
Mamy $n=p^0 n_1$, $p \nmid n_1$.
Uwaga (3.3) mówi, że każdy pierwiastek $W_n$ ma krotność $p^0 =1$.
Skoro $f$ dzieli $W_n$, to każdy jego pierwiastek jest też jednokrotny.

\section*{2/5aD}
Niech $K ⊆ L$ to ciała skończone, $\abs{K} = p^m$, $\abs{L} = p^n$.
$L$ jest przestrzenią wektorową nad $K$ i
$\abs{L} =\abs{K}^{[L ∶ K]} = p^{m[L ∶ K]}$,
więc $n = m[L ∶ K]$.

\end{document}
