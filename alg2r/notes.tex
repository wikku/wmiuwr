\documentclass[a4paper, 12pt]{article}
\usepackage[utf8]{inputenc}
\usepackage{silence}
\usepackage{polski}
\usepackage{parskip}
\usepackage{amsmath,amsfonts,amssymb,amsthm}
\usepackage{mathtools}
\usepackage{enumitem}
%\usepackage{pgfplots}
%\pgfplotsset{compat=1.16}
\usepackage{newunicodechar}
\usepackage{etoolbox}
\usepackage[margin=1.1in]{geometry}
\usepackage{algorithm}
\usepackage{algorithmicx}
\usepackage{algpseudocode}
\setcounter{secnumdepth}{0}

\title{Notatki (definicje, fakty) z Algebry 2R}
\author{}
\date{\vspace{-4ex}}

\DeclareMathOperator{\im}{Im}
\DeclareMathOperator{\rank}{rank}
\DeclareMathOperator{\Lin}{Lin}
\DeclareMathOperator{\sgn}{sgn}
\DeclareMathOperator{\Char}{char}
\DeclareMathOperator{\Hom}{Hom}
\DeclareMathOperator{\alg}{alg}
\DeclareMathOperator{\sep}{sep}
\DeclareMathOperator{\rad}{rad}
\DeclareMathOperator{\acl}{acl}
\DeclareMathOperator{\Aut}{Aut}
\DeclareMathOperator{\Tr}{Tr}
\DeclareMathOperator{\Norm}{N}
\DeclareMathOperator{\End}{End}
\newcommand{\N}{\mathbb{N}}
\newcommand{\Z}{\mathbb{Z}}
\newcommand{\Q}{\mathbb{Q}}
\newcommand{\R}{\mathbb{R}}
\newcommand{\C}{\mathbb{C}}
\newcommand{\modulus}[1]{\left| #1 \right|}
\newcommand{\abs}{\modulus}
\newtheorem{theorem}{Twierdzenie}
\newtheorem{lemat}{Lemat}
\newcommand{\ol}{\overline}
\DeclareMathOperator{\tr}{tr}
\DeclareMathOperator{\diag}{diag}
\newcommand{\+}{\enspace}

\newunicodechar{∅}{\emptyset} % Digr /0
\newunicodechar{∞}{\infty} % Digr 00
\newunicodechar{∂}{\partial} % Digr dP
\newunicodechar{α}{\alpha}
\newunicodechar{β}{\beta}
\newunicodechar{ξ}{\xi} % Digr c*
\newunicodechar{δ}{\delta} % Digr d*
\newunicodechar{ε}{\varepsilon}
\newunicodechar{φ}{\varphi}
\newunicodechar{θ}{\theta} % Digr h*
\newunicodechar{λ}{\lambda}
\newunicodechar{μ}{\mu}
\newunicodechar{π}{\pi}
\newunicodechar{ψ}{\psi}
\newunicodechar{σ}{\sigma}
\newunicodechar{τ}{\tau}
\newunicodechar{ω}{\omega}
\newunicodechar{η}{\eta} % Digr y*
\newunicodechar{ζ}{\zeta} % Digr z*
\newunicodechar{Δ}{\Delta}
\newunicodechar{Γ}{\Gamma}
\newunicodechar{Λ}{\Lambda}
\newunicodechar{Θ}{\Theta}
\newunicodechar{Φ}{\Phi} % Digr F*
\newunicodechar{Π}{\Pi}
\newunicodechar{Ψ}{\Psi} % digr Q*
\newunicodechar{ℕ}{\N} % Digr NN 8469 nonstandard
\newunicodechar{ℤ}{\Z} % Digr ZZ 8484 nonstandard
\newunicodechar{ℚ}{\Q} % Digr QQ 8474 nonstandard
\newunicodechar{ℝ}{\R} % Digr RR 8477 nonstandard
\newunicodechar{ℂ}{\C} % Digr CC 8450 nonstandard
\newunicodechar{∑}{\sum}
\newunicodechar{∏}{\prod}
\newunicodechar{∫}{\int}
\newunicodechar{∓}{\mp}
\newunicodechar{⌈}{\lceil} % Digr <7
\newunicodechar{⌉}{\rceil} % Digr >7
\newunicodechar{⌊}{\lfloor} % Digr 7<
\newunicodechar{⌋}{\rfloor} % Digr 7>
\newunicodechar{≅}{\cong} % Digr ?=
\newunicodechar{≡}{\equiv} % Digr 3=
\newunicodechar{◁}{\triangleleft} % Digr Tl
\newunicodechar{▷}{\triangleright} % Digr Tr
\newunicodechar{≤}{\le}
\newunicodechar{≥}{\ge}
\newunicodechar{≪}{\ll} % Digr <*
\newunicodechar{≫}{\gg} % Digr *>
\newunicodechar{≠}{\ne}
\newunicodechar{⊆}{\subseteq} % Digr (_
\newunicodechar{⊇}{\supseteq} % Digr _)
\newunicodechar{⊂}{\subset} % Digr (C
\newunicodechar{⊃}{\supset} % Digr C)
\newunicodechar{∩}{\cap} % Digr (U
\newunicodechar{∖}{\setminus} % Digr -\ nonstandard
\newunicodechar{∪}{\cup} % Digr )U
\newunicodechar{∼}{\sim} % Digr ?1
\newunicodechar{≈}{\approx} % Digr ?2
\newunicodechar{∈}{\in} % Digr (-
\newunicodechar{∋}{\ni} % Digr -)
\newunicodechar{∇}{\nabla} % Digr NB
\newunicodechar{∃}{\exists} % Digr TE
\newunicodechar{∀}{\forall} % Digr FA
\newunicodechar{∧}{\wedge} % Digr AN
\newunicodechar{∨}{\vee} % Digr OR
\newunicodechar{⊥}{\bot} % Digr -T
\newunicodechar{⊤}{\top} % Digr TO 8868 nonstandard
\newunicodechar{⇒}{\implies} % Digr =>
\newunicodechar{⇐}{\impliedby} % Digr <=
\newunicodechar{⇔}{\iff} % Digr ==
\newunicodechar{↔}{\leftrightarrow} % Digr <>
\newunicodechar{↦}{\mapsto} % Digr T> 8614 nonstandard
\newunicodechar{∘}{\circ} % Digr Ob
\newunicodechar{⊕}{\oplus} % Digr O+ 8853
\newunicodechar{⊗}{\otimes} % Digr OX 8855

% cursed
\WarningFilter{newunicodechar}{Redefining Unicode}
\newunicodechar{·}{\ifmmode\cdot\else\textperiodcentered\fi} % Digr .M
\newunicodechar{×}{\ifmmode\times\else\texttimes\fi} % Digr *X
\newunicodechar{→}{\ifmmode\rightarrow\else\textrightarrow\fi} % Digr ->
\newunicodechar{←}{\ifmmode\leftarrow\else\textleftarrow\fi} % Digr ->
\newunicodechar{⟨}{\ifmmode\langle\else\textlangle\fi} % Digr LA 10216 nonstandard
\newunicodechar{⟩}{\ifmmode\rangle\else\textrangle\fi} % Digr RA 10217 nonstandard
\newunicodechar{…}{\ifmmode\dots\else\textellipsis\fi} % Digr .,
\newunicodechar{±}{\ifmmode\pm\else\textpm\fi} % Digr +-

% https://tex.stackexchange.com/a/438184
% https://tex.stackexchange.com/q/528480
\newunicodechar{∶}{\mathbin{\text{:}}}
\def\newcolon{%
  \nobreak\mskip2mu\mathpunct{}\nonscript\mkern-\thinmuskip{\text{:}}%
  \mskip 6mu plus 1 mu \relax}
\mathcode`:="8000
{\catcode`:=\active \global\let:\newcolon}
% colon: for types; ratio∶ (digr :R) for relations (set builder)


\begin{document}

\maketitle

\section*{Wykład01.pdf}

% Def. 1.3
Niech $S ⊃ R$ to rozszerzenie pierścieni, $\bar{a} ⊆ S^n$.
Wtedy mówimy, że
$$I(\bar{a}/R) = \{ g∈R[\bar{X}] ∶ g(\bar{a}) = 0 \} ◁ R[\bar{X}]$$
to \textit{ideał $\bar{a}$ nad $R$}.
Jeśli $I(\bar{a}/R) = (f_1, …, f_m)$, to mówimy,
że $\bar{a}$ jest \textit{rozwiązaniem ogólnym} układu $f_1, …, f_m$.

Niech $K ⊂ L_1, K ⊂ L_2$ to rozszerzenia ciał.
Mówimy, że $L_1$ i $L_2$ są \textit{izomorficzne nad $K$},
gdy istnieje izomorfizm między nimi,
który jest identycznością na $K$.
Notacja: $L_1 ≅_K L_2$.

% Uwaga 1.5
Załóżmy, że $K ⊂ L_1$ i $K ⊂ L_2$ to rozszerzenia ciał,
$\bar{a}_1 ⊆ L_1$, $\bar{a}_2 ⊆ L_2$, $|\bar{a}_1| = |\bar{a}_2|$.
Wówczas $I(\bar{a}_1/K) = I(\bar{a}_2/K)$ wtedy i tylko wtedy,
gdy istnieje izomorfizm $f: K[\bar{a}_1] → K[\bar{a}_2]$
przekształcający $\bar{a}_1$ na $\bar{a}_2$ i ustalający $K$.

% Tw. 1.6
Niech $I ◁ K[\bar{X}]$.
Wtedy istnieje ciało $L ⊃ K$ oraz $\bar{a} = (a_1,…,a_n)⊂L$ takie,
że $f(\bar{a}) = 0$ dla każdego $f∈I$.

% Wn. 1.7
Niech $f∈K[X]$ stopnia dodatniego.
Wtedy istnieje rozszerzenie $K$, w którym $f$ ma pierwiastek.

% Fakt 1.8.(1)
Załóżmy, że $f∈K[X]$ nierozkładalny oraz dla $i=1,2$ mamy
$L_i = K(a_i)$ i $f(a_i) = 0$ (w $L_i$).
Wtedy $L_1 ≅_K L_2$.
% Fakt 1.8.(2)
Ogólniej: załóżmy, że $φ: K_1 \stackrel{≅}{→} K_2$, $f_i∈K_i[X]$,
$φ(f_1) = f_2$ i $f_i$ nierozkładalny nad $K_i$,
$L_1 = K_1(a_1)$, $L_2 = K_2(a_2)$,
gdzie $a_i$ jest pierwiastkiem $f_i$.
Wtedy istnieje $φ ⊆ ψ: L_1 \stackrel{≅}{→} L_2$ taki, że
$ψ(a_1) = a_2$.

Mówimy, że ciało $L ⊃ K$ jest \textit{ciałem rozkładu} wielomianu $f∈K[X]$ nad
$K$,
gdy $f$ rozkłada się w $L[X]$ na czynniki liniowe
i $L=K(a_1,…,a_n)$, gdzie $a_i$ to wszystkie pierwiastki $f$ w $L$.

% Wn. 2.1
Jeśli $f∈K[X]$ ma stopień dodatni,
to istnieje jedyne co do izomorfizmu nad $K$
ciało rozkładu $f$ nad $K$.

% Wn. 2.2 TODO

\section*{Wykład02.pdf}
Ciało $L$ jest \textit{algebraicznie domknięte},
gdy każdy $f∈L[X]$ stopnia $>0$ ma pierwiastek w $L$.

% (*) przed dowodem Tw. 2.3
Każde ciało jest algebraicznie domknięte w pewnym jego rozszerzeniu.

% Tw. 2.3
Każde ciało $K$ zawiera się w pewnym ciele algebraicznie domkniętym.


Mówimy, że ciało jest \textit{ciałem prostym}, gdy nie zawiera podciał właściwych.

% Uwaga 3.1.(1)
Każde ciało zawiera jedyne podciało proste.

% Uwaga 3.1.(2)
Z dokładnością do izomorfizmu,
$ℚ$ i $ℤ_p$ (dla $p$ pierwszych) to wszystkie ciała proste.


\begin{enumerate}
	\item $a∈R$ jest \textit{pierwiastkiem z $1$} (stopnia $n>0$), gdy $a^n=1$
	\item $μ_n(R) = \{ a∈R ∶ a^n=1 \} < R^*$
	\item $μ(R) = \{ a∈R ∶ ∃n>0\+ a^n=1 \} = \bigcup_{n>0} μ_n(R) < R^*$
	\item $a∈R$ jest \textit{pierwiastkiem pierwotnym (primitive) stopnia $n$ z jedynki},
		gdy $n$ jest najmniejsze takie, że $a^n=1$.
\end{enumerate}

Oznaczamy $W_n(X) = X^n-1$.
% Uwaga 3.3.(1)
W ciele o charakterystyce $0$ ten wielomian ma tylko pierwiastki jednokrotne.
W ciele o charakterystyce $p$ każdy pierwiastek tego wielomianu ma krotność
$p^l$, gdzie $p^l$ to najwyższa potęga $p$ dzieląca $n$.

% Tw. 3.4
Załóżmy, że $G < μ(K)$ to grupa skończona rzędu $n$.
Wtedy $G = μ_n(K)$, $G$ jest cykliczna i $p \nmid n$ (gdy $\Char K = p$).

% Wn. 3.5
Niech $a∈μ_n(K)$. Wtedy
Jeśli $a$ jest pierwiastkiem pierwotnym stopnia $n$ z $1$, to
$a$ generuje $μ_n(K)$.

% Tw. 3.6
Załóżmy, że $K$ jest ciałem skończonym i $p = \Char K$.
Wtedy $|K|=p^n$ dla pewnego $n$.
Dla każdego $n>0$ istnieje dokładnie jedno (co do izomorfizmu)
ciało mocy $p^n$.

\section*{Wykład03.pdf}
% Def 4.1
\begin{enumerate}
	\item \textit{$a$ jest algebraiczny nad $K$},
		gdy jest pierwiastkiem pewnego $f∈K[X]\setminus\{0\}$.
	\item \textit{$a$ jest przestępny nad $K$}, gdy nie jest algebraiczny nad $K$.
	\item Rozszerzenie $K⊂L$ jest \textit{algebraiczne},
		gdy każdy $l∈L$ jest algebraiczny nad $K$.
	\item Rozszerzenie $K⊂L$ jest \textit{przestępne}, gdy nie jest algebraiczne.
	\item Liczba zespolona $z∈ℂ$ jest \textit{algebraiczna / przestępna},
		gdy jest algebraiczna / przestępna nad $ℚ$.
\end{enumerate}

% Uwaga 4.2
$a$ jest algebraiczny nad $K$ wtedy i tylko wtedy, gdy
$I(a/K) ≠ \{0\}$.

Niech $K⊂L$ to rozszerzenie ciał.
\textit{Stopień rozszerzenia} $[L∶K]$ to wymiar $L$ jako przestrzeni liniowej nad $K$.

% Uwaga 4.3
Załóżmy, że $a ∈ L ⊃ K$.
Wtedy następujące warunki są równoważne:
\begin{enumerate}
	\item $a$ algebraiczny nad $K$
	\item $K[a] = K(a)$
	\item $[K(a)∶K] < ∞$
\end{enumerate}

% Def 4.4
Niech $K⊂L$ to rozszerzenie ciał, $a∈L$ jest algebraiczny nad $K$.
Wtedy \textit{wielomianem minimalnym $a$ nad $K$} nazywamy moniczny wielomian
generujący $I(a/K)$.
Stopień tego wielomianu minimalnego nazywamy \textit{stopniem $a$ nad $K$}.

% Uwaga 4.5
Wielomian minimalny $f$ elementu $a$ jest wielomianem unormowanym
minimalnego stopnia takim, że $f(a) = 0$.
$\deg f = [K(a) ∶ K]$.

% Fakt 4.6
Niech $K⊂L⊂M$ to rozszerzenia ciał.
Wtedy $[M∶K] = [M∶L][L∶K]$.

% Wn. 4.7, Def 4.8
$K_{alg}(L) = \{ a ∈ L ∶ a \text{ algebraiczny nad }K \}$
nazywamy \textit{algebraicznym domknięciem ciała $K$ w ciele $L$}.
Jest ono podciałem $L$ i nadciałem $K$.
$K$ jest algebraicznie domknięte w $L$, gdy $K_{alg}(L) = K$.

Algebraiczne domknięcie $K$ w ciele algebraicznie domkniętym
nazywamy \textit{algebraicznym domknięciem},
które oznaczamy $\hat{K}$ lub $K^{\alg}$.

% wut ? to samo w Wykład04.pdf Def. 5.7

% Wn. 5.1
Załóżmy, że $K⊂L⊂M$ to rozszerzenia ciał.
Wtedy $K⊂M$ jest algebraiczne wtedy i tylko wtedy, gdy
$K⊂L$ i $L⊂M$ są algebraiczne.

% Wn. 5.2
$K_{\alg}(L)$ jest algebraicznie domknięte w $L$.

\section*{Wykład04.pdf}
\textit{Wielomiany cyklotomiczne}
$$ F_m(x) =
	∏_{\substack{1≤k≤m \\ \gcd(k,m) = 1}}
		\left(x-e^{2πi\frac{k}{m}}\right)
$$

% Wn. 5.4
$F_m(X)$ jest nierozkładalny w $ℚ[X]$ (równoważnie w $ℤ[X]$ z lematu Gaussa).

% Wn 5.5
Załóżmy, że $ε∈ℂ$ jest pierwiastkiem pierwotnym z $1$ stopnia $m$.
Wtedy $[ℚ(ε)∶ℚ] = φ(m)$, bo $F_m$ jest wielomianem minimalnym $ε$ nad $ℚ$.

% Lemat 5.6 (Liouville)
(Lemat Liouville'a)
Jeśli $a∈ℝ$ algebraiczna stopnia $N>1$ nad $ℚ$, to
istnieje $C$ taka, że dla każdego $p/q∈ℚ$ mamy
$$\abs{a-\frac{p}{q}} ≥ \frac{C}{q^n}.$$

% Def. 5.7
$L ⊃ K$ jest algebraicznym domknięciem ciała $K$, gdy
$L$ jest algebraicznie domknięte i
rozszerzenie $L ⊂ L$ jest algebraiczne nad $K$.
Oznaczamy $L=\hat{K}=K^{alg}$.
% Wn. 5.8
% algebraiczne domknięcie ciała $K$
$\hat{K}$ zawsze istnieje
% Tw. 5.9
i jest jedyne co do izomorfizmu nad $K$.

% Wn. 5.10
Jeśli $f: K \stackrel{≅}{→} L$,
to istnieje $f⊆\hat{f}: \hat{K} \stackrel{≅}{→} \hat{L}$.

% Wn. 5.11
Jeśli rozszerzenie $K ⊂ L$ jest algebraiczne, to istnieje zanurzenie
$L$ w $\hat{K}$ stałe na $K$.

\textit{Grupa Galois rozszerzenia $K ⊂ L$} to
$$G(L/K) = \{ f∈\Aut(L) ∶ f\mid_K=id_K \} < \Aut(L).$$
$G(\hat{K}/K)$ jest \textit{absolutną grupą Galois ciała $K$}.

% Uwaga 6.1 (jednorodność \hat{K})
Jeśli $I(a/K) = I(b/K)$, to istnieje $f∈G(\hat{K}/K)$ taki,
że $f(a) = b$.

Rozszerzenie algebraiczne ciał $K ⊂ L$ jest \textit{normalne},
gdy każdy homomorfizm z $L$ do $\hat{K}$,
który jest identycznością na $K$, ma ten sam obraz. % FIXME?

% Uwaga 6.2
Rozszerzenie algebraiczne $K⊂L$ jest normalne wtedy i tylko wtedy,
gdy dla każdego $f∈G(\hat{K}/K)$ mamy $f[L] = L$.

% Wn.
Jeśli $K⊆L_1⊆L$ i $K⊆L$ normalne, to $L_1⊆L$ też.

% Tw. 6.3
Rozszerzenie algebraiczne $K⊂L$ jest normalne wtedy i tylko wtedy,
gdy wielomian minimalny każdego elementu $L$
rozkłada się nad $L$ na czynniki liniowe.

\section*{Wykład05.pdf}
Rozszerzenie ciał $K⊆L$ jest \textit{skończone}, gdy $[L∶K]<∞$.

%Tw.6.4
Rozszerzenie skończone $L⊇K$ jest normalne
$⇔$ $L$ jest ciałem rozkładu pewnego wielomianu $W∈K[X]$ nad $K$.

\textit{Normalne domknięcie ciała $L$ w $\hat{K}$ nad $K$} to
$$L_1 = \text{ciało generowane przez }\bigcup \{ f[L] ∶ f∈G(\hat{K}/K)\}.$$
Rozszerzenie $K ⊆ L$ jest normalne.

Gdy wielomian minimalny $a∈\hat{K}$ nad $K$,
$W_a(X) ∈ K[X]$, ma w $\hat{K}$ tylko pierwiastki
jednokrotne,
to mówimy, że element $a$ jest \textit{rozdzielczy} nad $K$.

Rozszerzenie algebraiczne $K⊂L$ jest \textit{rozdzielcze}, gdy
każdy element $L$ jest rozdzielczy nad $K$.

Wielomian $W(X)∈K[X]$ jest \textit{rozdzielczy}, gdy
ma tylko pierwiastki jednokrotne w $\hat{K}$.

% Uwaga 6.6
Wielomian $W$ nierozkładalny jest
nierozdzielczy wtedy i tylko wtedy, gdy
$W$ i $W'$ są względnie pierwsze.

W ciele o charakterystyce $0$ wszystkie wielomiany minimalne są rozdzielcze.
W ciele $K$ o charakterystyce $p$ wielomiany nierozdzielcze należą do $K[X^p]$.

Jeśli $K⊆L$ jest rozdzielcze i $K⊆L_1⊆L$, to $L_1 ⊆ L$ rozdzielcze.

Rozszerzenie $K⊆L$ ciał skończonych jest rozdzielcze.

Każde rozszerzenie algebraiczne ciała charakterystki $0$ jest rozdzielcze.

% Lemat 6.7
Zachodzi $\{f(a) ∶ f∈G(\hat{K}/K)\} ≤ \deg(a/K)$,
a jeśli $a$ jest rozdzielczy nad $K$, to zachodzi równość.

Element $a∈L$ nazywamy \textit{elementem pierwotnym rozszerzenia} $K⊆L$,
gdy $L=K(a)$.

% Tw. 6.8
(Twierdzenia Abela o elemencie pierwotnym)
Jeśli rozszerzenie $K⊂K(a_1,…,a_n)=L$ jest skończone i $a_i$ są
rozdzielcze nad $K$, to istnieje $a^*∈L$ rozdzielczy nad $K$ taki,
że $L=K(a^*)$.
Inaczej, rozszerzenie skończone rozdzielcze jest proste.

Element $a∈L$ nazywamy \textit{czysto nierozdzielczym (radykalnym)} nad $K$,
gdy $W_a(X)∈K[X]$ ma tylko jeden pierwiastek w $\hat{K}$.

Rozszerzenie $K⊆L$ nazywamy \textit{radykalnym (czysto nierozdzielczym)},
gdy każdy $a∈L$ jest radykalny nad $K$.

\section*{Wykład06.pdf}
\textit{Rozdzielcze domknięcie $K$ w $L$} to
$$\sep_L(K) = \{a∈L ∶ a \text{ rozdzielczy nad }K \}.$$

\textit{Czysto nierozdzielcze (radykalne) domknięcie $K$ w $L$} to
$$\rad_L(K) = \{a∈L ∶ a \text{ radykalny nad }K \}.$$

% Wniosek 7.2
Jeśli $K ⊆ L$ algebraiczne, to
$$K ⊆ \sep_L(K), \rad_L(K) ⊆ L ⊆ \hat{K},
\+ \sep_L(K) ∩ \rad_L(K) = K.$$

\textit{Rozdzielcze domknięcie} $K$ to $\hat{K}^s = \sep_{\hat{K}}(K)$.

\textit{Radykalne domknięcie} $K$ to $\hat{K}^r = \rad_{\hat{K}}(K)$.

% Uwaga 7.3 (1)
Gdy $K ⊆ L ⊆ \hat{K}$,
to $\sep_L(K) = \hat{K}^s ∩ L, \+ \rad_L(K) = \hat{K}^r ∩ L$.

% Uwaga 7.3 (2)
Załóżmy, że $K ⊆ L ⊆ M ⊆ \hat{K}$. Wtedy
$$K ⊆_{\rad} L ⊆_{\rad} M ⇔ K ⊆_{\rad} M.$$

% Uwaga 7.3 (3)
Gdy $\Char K = 0$,
to $\sep_L(K) = K^{\alg}(L)$ i $\rad_L(K) = K$
oraz $\hat{K}^s = \hat{K}$ i $\hat{K}^r = K$.

% Fakt 7.4 TODO

\textit{Stopień rozdzielczy ciała $L$ nad $K$} to $[L∶K]_s = [\sep_L(K)∶K]$.
\textit{Stopień radykalny ciała $L$ nad $K$} to $[L∶K]_r = [L∶\sep_L(K)]$.

% Uwaga 7.5 TODO

% Uwaga 7.6 (?)
Jeśli $\Char K = p > 0$ i $[L∶K]_r<∞$, to $[L∶K]_r$ jest potęgą $p$.

% ---
% Norma i ślad

Jeśli $K ⊆ L$ to rozszerzenie skończone $a ∈ L$, to
$f_a: L → L, f_a(x) = a·x$ jest przekształceniem $K$-liniowym.
\textit{Normą} nazywamy $\Norm_{L/K}(a) = \det f_a$,
a \textit{śladem} $\Tr_{L/K}(a) = \tr f_a$.

% Fakt 8.1
Niech $K ⊆ L$ to rozszerzenie skończone,
$\{ f_1, …, f_k \} = \{ f ∈ \Hom(L, \hat{K}) ∶ f|_K = id \}$,
$k=[L∶K]_s$, $a ∈ L$.
Wtedy
$$\Norm_{L/K}(a) = \left(∏_{i=1}^k f_i(a)\right)^{[L∶K]_r},
\+\Tr_{L/K}(a) = [L∶K]_r ∑_{i=1}^k f_i(a).$$


\section*{Wykład07.pdf}
Rozszerzenie algebraiczne ciał $K ⊂ L$ jest
\textit{rozszerzeniem Galois}, gdy
$∀a∈L\setminus K\, ∃f∈G(L/K),\, f(a)≠a$.

Niech $G<\Aut(L)$. Wtedy \textit{ciałem punktów stałych grupy $G$} nazywamy
$$L^G = \{ a∈L ∶ ∀f∈G f(a)=a \} = \bigcup_{f∈G} Fix(f).$$

Roszerzenie algebraiczne $K⊂L$
jest Galois wtedy i tylko wtedy, gdy
$K = L^{G(L/K)}.$

% Przykład (1), (2), (3) todo

% Tw. 8.2
Niech $K⊂L$ to rozszerzenie algebraiczne.
Jest ono Galois wtedy i tylko wtedy, gdy
jest rozdzielcze i normalne.

% Wniosek 8.3
Niech $K ⊂ L ⊂ M ⊂ \hat{K}$.
Jeśli $K ⊂ M$ Galois, to $L ⊂ M$ Galois.

% Tw. 8.4 (Artin)
Jeśli $G < \Aut(L)$ skończona, to $L^G ⊂ L$ Galois i $[L ∶ L^G] = |G|$.

% Wniosek 8.5
Jeśli $K ⊂ L$ to skończone rozszerzenie Galois,
to $[L ∶ K] = \abs{G(L/K)}.$

Niech $K⊂L$ to rozszerzenie algebraiczne,
\begin{align*}
	\mathcal{L} &= \{L' ∶ K ⊆ L' ⊆ L \}, & \mathcal{G} &= \{ H ∶ H < G(L/K) \}, \\
	Γ : \mathcal{L} &\rightarrow \mathcal{G}, & Λ : \mathcal{G} &\rightarrow \mathcal{L}, \\
	L' &\stackrel{Γ}{↦} G(L/L') < G(L/K), &  G &\stackrel{Λ}{↦} L^G ⊆ L.
\end{align*} %
% Podstawowe twierdzenie teorii Galois
Jeśli $K⊂L$ jest rozszerzeniem skończonym, to $Γ$ i $Λ$ są wzajemnie odwrotne.

% Wniosek 8.9
Jeśli $K⊂L$ jest skończonym rozszerzeniem Galois, to
dla $H<G(L/K)$
$$H◁G(L/K) ⇔ K ⊂ L^H \text{ normalne Galois}.$$
% FIXME nie wiem czy na pewno tak to rozumieć

\section*{Wykład08.pdf}
Załóżmy, że rozszerzenie $K⊂L$ jest skończone Galois.
Mówimy, że jest ono \textit{abelowe / cykliczne}, gdy $G(L/K)$ jest
\textit{abelowa / cykliczna}.

% Tw. 9.3
Załóżmy, że $K ⊂ L_1 ⊂ L$ to rozszerzenia ciał.
Jeśli $K⊂L$ abelowe/cykliczne, to $K⊂L_1$ i $L_1⊂L$ też.

% Tw. 9.4
Załóżmy, że rozszerzenie $K⊂L$ cykliczne,
$[L∶K]=n$, $ζ∈K$ to pierwiastek pierwotny z $1$ stopnia $n$.
Wtedy $∃a∈K\+ L=K(\sqrt[n]{a})$.

% Tw. 9.5
(Tw. Dedekinda o liniowej niezależności charakterów)
Załóżmy, że $α_1,…,α_n ∈ \Aut(L)$ i $(a_1,…,a_n)$
to niezerowa krotka w $L^n$.
Wtedy $∃c ∈ L\+ (∑_{i=1}^n a_i α_i)(c) ≠ 0$,
tzn. $α_i$ są liniowo niezależne w przestrzeni $L^L$ nad $L$.
% FIXME: L^L?

Załóżmy, że $K⊂L$ to skończone rozszerzenie ciał.
Mówimy, że jest ono \textit{rozwiązalne},
gdy jest Galois i grupa $G(L/K)$ jest rozwiązalna.
Mówimy, że jest ono \textit{przez pierwiastniki}, jeśli
istnieje ciąg zstępujący
$$L=L_0 ⊃ L_1 ⊃ … ⊃ L_k = K$$
taki, że $L_i$ jest ciałem rozkładu nad $L_{i+1}$
wielomianu
\begin{align*}
	X^{n_i}-b_i & \quad (\text{gdy }\Char K = p \nmid n_i)\\
	\text{ lub }X^p-X-b_i & \quad (\text{gdy }\Char K = p),
\end{align*}
gdzie $b_i ∈ L_{i+1}$.

% Tw. 9.6
Załóżmy, że $K⊂L$ to rozszerzenie skończone ciał.
Wtedy $K⊂L$ jest roszerzeniem przez pierwiastniki wtedy i tylko wtedy,
gdy istnieje $L'$ takie, że $K ⊂ L'$ rozwiązalne.

\section*{Wykład09.pdf}
Rozszerzenie $K⊂L$ nazywamy \textit{przestępnym}, gdy
istnieje $a∈L$ przestępny nad $K$ (tzn. $I(a/K) = \{ 0 \}$).

Rozszerzenie $K⊂L$ nazywamy \textit{czysto przestępnym}, gdy
każdy $a∈L∖K$ przestępny.

% Uwaga 10.1
Element $a$ jest przestępny nad $K$ wtedy i tylko wtedy, gdy
$K(a) ≅ K(X)$.

Niech $U = \hat{U}$ to ciało oraz $K ⊂ U$ to jego podciało,
a $F ⊂ K$ to podciało proste.
\textit{Operatorem domknięcia algebraicznego nad $K$} nazywamy
$\acl_K: \mathcal{P}(U) → \mathcal{P}(U), \acl_K(A) = K(A)^{\alg}$.

Zbiór $A ⊆ U$ jest \textit{algebraicznie domknięty nad $K$}, gdy $A = \acl_K(A)$.

\begin{enumerate}
	\item $\acl_K(∅) = \hat{K}$
	\item $\acl_K(\acl_K(A)) = \acl_K(A)$
	\item $\acl_K(A) = \bigcup_{A_0 ⊂_{\mathrm{fin}} A} \acl_K(A_0)$
		(skończony charakter)
	\item $a ∈ \acl_K(A∪\{b\})∖\acl_K(A) ⇒ b ∈ \acl_K(A∪\{a\})$
		(własność wymiany)
\end{enumerate}

Zbiór $A ⊂ U$ jest \textit{algebraicznie niezależny nad $K$},
gdy $∀a∈A\+a\not∈\acl_K(A∖\{a\})$.
Równoważnie, dla dowolnych różnych $a_1,…,a_n ∈ A$ i niezerowego
$W(X_1,…,X_n) ∈ K[\bar{X}]$ mamy $W(\bar{a}) ≠0$.

Zbiór $A$ jest \textit{bazą przestępną zbioru $B⊂U$ nad $K$},
gdy $A$ jest algebraicznie niezależny nad $K$ i $A⊆B⊆\acl_K(A)$.

Moc (jakiejkolwiek) bazy przestępnej zbioru $B$ nad $K$
nazywamy wymiarem przestępnym $B$ nad $K$ i oznaczamy
$\tr \deg_K(B)$.

% gdy ciało proste pomijamy je?? o co chodzi

% Tw. 10.2
Jeśli $A⊆B⊆U$ i $A$ jest algebraicznie niezależny nad $K$,
to istnieje $A'$ taki, że $A ⊆ A' ⊆ B$ i $A'$ jest bazą przestępną
$B$ nad $K$.

Każde dwie bazy przestępne zbioru $B$ nad $K$ są równoliczne.

% Przykład 1
Zbiór $\{ X_i ∶ i ∈ I \} ⊆ K(\bar{X}) = U$ jest niezależny nad $K$
i $\tr \deg_K(U) = |I|$.

% Przykład
Jeśli $K ⊂ L ⊂ U$ oraz $\{ a_i ∶ i ∈ I \}$ to baza przestępna $L/K$, to
$$K(a_i ∶ i ∈ I) ≅ K(X_i ∶ i ∈ I),$$
$$K \stackrel{\text{czysto przestępne}}{⊆}
K(a_i ∶ i ∈ I) \stackrel{\text{algebraiczne}}{⊆} L.$$

\section*{Wykład10.pdf}

% Def. 10.3
$(M, +, r)_{r ∈ R}$ to \textit{moduł} (domyślnie lewostronny) nad $R$, jeśli
\begin{enumerate}
	\item dla każdego $r$ mamy operację mnożenia elementu modułu przez skalar $r$ z lewej;
	\item $(M,+)$ to grupa abelowa, jej zero $0$ nazywamy zerem modułu $M$;
	\item $r(m_1+m_2) = rm_1 + rm_2$;
	\item $(r_1+r_2)m = r_1m + r_2m$;
	\item $r_1(r_2 m) = (r_1 r_2) m$ (zgodność); %?
	\item $1m = m$
\end{enumerate}
Analogicznie możemy zdefiniować moduł prawostronny, z odpowiednio
zmienionym aksjomatem zgodności.
Jeśli $R$ przemienny, to te pojęcia są równoważne.

% Przykład 1.
Przestrzeń liniowa nad $K$ to $K$-moduł.

% Przykład 2.
Grupy abelowe to dokładnie $ℤ$-moduły.

% Przykład 3.
Grupa abelowa $G$ jest $\End(G)$-modułem,
gdzie $\End(G)$ to jej pierścień endomorfizmów.

% Przykład 4.
Załóżmy, że $j: R → \End(G)$ to homomorfizm
pierścieni z jednością.
Wtedy $j$ wyznacza w $G$ strukturę $R$-modułu,
gdzie $r · g = j(r)(g)$.
Na odwrót, gdy $(G,+,r)$ to $R$-moduł, to
możemy wziąć za $j$ mnożenie skalarne.

% Przykład 5.
Jeśli $R_1 ⊂ R$, to $R$ jest modułem nad $R_1$.

% Przykład 6.
Niech $j: R_1 → R$ to homomorfizm pierścieni z jednością.
Wtedy $R$-moduł jest $R_1$-modułem z operacją mnożenia przez
wartość $j$.

% Przykład 7.
Jeśli $I ⊆ R$ to ideał lewostronny,
to $I$ jest $R$-modułem.

% Def. 10.4
Załóżmy, że $M$ to $R$-moduł.
Mówimy, że $N ⊆ M$ jest $R$-podmodułem $M$, gdy
jest podgrupą abelową z dodawaniem (więc $N$ jest niepusty)
i zamknięty na mnożenie przez skalary.

% Uwaga 10.5
Załóżmy, że $M$ to $R$-moduł.
Wtedy
\begin{enumerate}
	\item $0·m = 0$;
	\item $r·0 = 0$;
	\item $(-1)·m = -m$.
\end{enumerate}

% Uwaga 10.6
Niech $M$ to $R$-moduł.
Przekrój dowolnej niepustej rodziny podmodułów $M$
jest podmodułem $M$.

Mówimy, że $\{0\} ⊆ M$ to podmoduł zerowy.

% Wniosek 10.6
Jeśli $A ⊆ M$, to istnieje najmniejszy
podmoduł $N ⊆ M$ zawierający $A$.
Nazywamy go podmodułem generowanym przez $A$.

Jeśli $N_1, N_2$ to podmoduły $M$, to $N_1 + N_2$ też.

Produkt prosty $R$-modułów definiujemy podobnie jak dla przestrzeni liniowych
i oznaczamy $M × N$.

(Suma prosta wewnętrzna) % FIXME ???
Mówimy, że $M = N_1 ⊕ … ⊕ N_k$, gdy $N_i$ są podmodułami $M$
i każdy element $M$ się jednoznacznie zapisuje jako suma
elementów $N_1$ (po jednym z każdego).

Niech $h: M → N$ to homomorfizm $R$-modułów.
Jeśli $N' ⊂ N$ jest podmodułem, to $h^{-1}[N'] ⊂ M$ też.
Jeśli $M' ⊂ M$ jest podmodułem, to $h[M'] ⊂ N$ też.

Niech $M' ⊂ M$ to podmoduł.
Wtedy $M/M' = \{ x+M' ∶ x ∈ M \}$ nazywamy modułem ilorazowym
(ze standardowymi operacjami).

% Tw. 10.7
(zasadnicze tw. o homomorfizmie $R$-modułów)
% TODO

% Tw. 10.8
(tw. o faktoryzacji)
% TODO

Definiujemy $\Hom_R(M,N) = \{ h: M → N ∶ h \text{ to homomorfizm} \}$.

% Def. 10.9
$M$ jest $R$-modułem prostym, gdy $M ≠ \{ 0 \}$ i
każdy jego podmoduł jest zerowy lub całym $M$.

Pierścień endomorfizmów $R$-modułu $M$ zapisujemy $\End_R(M)$.

% Lemat 10.10
(Lemat Schura)
Jeśli $M$ to $R$-moduł prosty, to $\End_R(M)$ to pierścień z dzieleniem.

Załóżmy, że $M$ to $R$-moduł oraz $K = \End_R(M)$ to pierścień z dzieleniem
(ciało nieprzemienne).
Wtedy $M$ jest też $K$-modułem.
% TODO

\section*{Wykład11.pdf}
Niech $M$ to $R$-moduł.
Układ $(m_i ∶ i ∈ I) ⊆ M$ jest liniowo niezależny,
gdy jego (skończona) kombinacja liniowa (ze współczynnikami z $R$) się zeruje
dokładnie kiedy wszystkie współczynniki są zerami.

Zbiór $S ⊆ M$ jest liniowo niezależny, gdy
układ z niego utworzony (bez powtórzeń) jest liniowo niezależny.

Zbiór $\mathcal{B} ⊆ M$ jest bazą $R$-modułu $M$, gdy
$\mathcal{B}$ jest liniowo niezależny (nad $R$) i
generuje $M$ jako $R$-moduł.

% Przykład 1.
Zbiory $\{ 0 \}, \{ m_0, m_0 \}$ są liniowo zależne.

Rozpatrzmy $ℚ$ jako $ℤ$-moduł.
Wtedy dowolna para jego elementów jest liniowo zależna.

Moduł $ℚ$ nie ma bazy jako $ℤ$-moduł!

(Abstrakcyjna) suma prosta (koprodukt) rodziny modułów $\{ M_i ∶ i ∈ I \}$
to
$$\coprod_{i ∈ I} M_i ≅
\left\{ f ∈ ∏_{i∈I} M_i ∶ f(i) = 0 \text{ dla prawie wszystkich }i ∈ I \right\}.$$

% Własność uniwersalności TODO


% Uwaga 11.2 TODO

% Def 11.3
$M$ jest wolnym $R$-modułem, gdy $M$ ma bazę.

% Przykłady
$M$ jest wolnym $R$-modułem, z bazą $\{ 1 \}$.

Jeśli $M_i$ to wolne $R$-moduły, to $\coprod_{i∈I} M_i$ jest wolnym $R$-modułem.

% Uwaga 11.4
Niech $A = \{ a_i ∶ i ∈ I \} ⊆ M$.
Następujące warunki są równoważne:
\begin{enumerate}
	\item $A$ to baza $M$;
	\item każdy element $M$ się jednoznacznie przedstawia jako kombinacja
		$R$-liniowa $A$;
	\item Każda funkcja z $A$ w $R$-moduł się rozszerza do homomorfizmu z $M$.

\end{enumerate}

% Uwaga 11.5
Jeśli $A = \{ a_i ∶ i ∈ I \}$ to baza $M$,
to $Ra_i$ jest podmodułem $M$ i $M = \bigoplus_{i∈I} R a_i$.

Jeśli $A$ to zbiór, to istnieje $R$-moduł o bazie $A$
(koprodukt izomorficznych kopii $R$ dla każdego elementu $A$).

% Tw. 11.6
Jeśli $R$ jest pierścieniem przemiennym,
to każde dwie bazy $R$-modułu wolnego $M$ są równoliczne.

% Uwaga 11.7
Każdy $R$-moduł jest homomorficznym obrazem $R$-modułu wolnego.

% Fakt 11.8
Załóżmy, że $M,N$ to $R$-moduły, $N$ jest wolny i
$f: M → N$ to epimorfizm.
Wtedy $M ≅ \ker f ⊕ N$.
Więcej: istnieje podmoduł $N' ⊆ M$ izomorficzny z $N$
i $M = \ker f ⊕ N'$.

% Def. 11.9
Mówimy, że $R$-moduł $N$ jest \textit{projektywny}, gdy
Dla każdego epimorfizmu $f: M → N$ mamy
$M = \ker f ⊕ M'$ dla pewnego podmodułu $M' ⊂ M$.
Mówimy, że $f$ rozszczepia się (\textit{splits}).

Dualnie, mówimy, że $R$-moduł $M$ jest \textit{iniektywny}, gdy
dla każdego monomorfizmu $g: M → N$ mamy
$N = \im g ⊕ N'$ dla pewnego podmodułu $N' ⊂ N$.

Jeśli $R$ to ciało, to każdy $R$-moduł jest iniektywny i projektywny.

% Def. 11.10
Niech $R$ to pierścien przemienny z jednością.
Mówimy, że $R$-moduł jest \textit{cykliczny}, gdy jest generowany przez
jeden element $a$ (równy $Ra$).

% Uwaga 11.11
$R$-moduł jest cykliczny, jeśli jest izomorficzny z pewnym
ilorazem $R$.

Niech $M$ to $R$-moduł.
Wtedy
\begin{enumerate}
	\item dla $a ∈ M$ mówimy,
		że $I_a = \{ r∈R ∶ ra = 0 \} ◁ R$ jest \textit{torsją elementu $a$};
	\item mówimy, że $a∈M$ jest \textit{torsyjny}, gdy $I_a ≠ \{ 0 \}$
		(w przeciwnym razie \textit{beztorsyjny});
	\item mówimy, że $M$ jest \textit{torsyjny}, gdy każdy jego element jest torsyjny
		(\textit{beztorsyjny}, gdy każdy niezerowy beztorsyjny);
	\item zbiór $M_t = \{ a∈M ∶ a\text{ torsyjny} \}$
		nazywamy \textit{częścią torsyjną modułu $M$}.
\end{enumerate}

% Uwaga 11.13
Załóżmy, że $R$ jest dziedziną.
Wtedy $M_t$ jest podmodułem $M$ i $M/M_t$ jest beztorsyjny.

% Przykład
Grupy abelowe torsyjne / beztorsyjne
to dokładnie $ℤ$-moduły torsyjne / beztorsyjne.

Załóżmy, że $R$ jest przemienny, $M,N$ to $R$-moduły,
$f: M → N$ to epimorfizm,
$M' = \ker f$, $N ≅ M/M'$.
Wtedy jeśli $N, M'$ skończenie generowane, to $M$ skończnie generowany
i jeśli $M$ skończenie generowany, to $N$ skończenie generowany.

% Wniosek 11.15
Załóżmy, że $R$ to pierścień przemienny.
Wtedy noetherowskość $R$ jest równoważna temu,
że podmoduły skończenie generowanego $R$-modułu są skończenie generowane.

% Fakt 12.1 TODO

% Def. 12.2
Niech $X$ to $R$-moduł wolny o bazie $M_1 × M_2$ (jako zbiór).
Niech $L ⊆ X$ to podmoduł generowany przez elementy ,,dające dwuliniowość''.
% TODO
Wtedy $f: M_1×M_2 → X/L$ jest $R$-$2$-linowe.
Moduł $X/L$ nazywamy \textit{produktem tensorowym} $M_1$ i $M_2$ oraz
oznaczamy $M_1 ⊗ M_2$.

% TODO małe i duże \otimes ⊗ ?

\section*{Wykład12.pdf}

% Uwaga 12.3
Niech $f: M_1 × M_2 → M_1 ⊗ M_2$ i $f(m_1,m_2) = m_1 ⊗ m_2$.
Wtedy $f$ jest dwuliniowe (często oznaczane przez $⊗$)
oraz dla każdego dwuliniowego homomorfizmu $g: M_1 × M_2 → N$
istnieje jedyny homomorfizm $R$-liniowy $h: M_1 ⊗ M_2 → N$ taki,
że $g = h ∘ f$ (warunek uniwersalności).
Intuicyjnie, $f = ⊗$ to najogólniejsze odwzorowanie 2-liniowe z $M_1×M_2$
w jakikolwiek $R$-moduł.

% Uwaga 12.4
Powyższy warunek wyznacza iloczyn tensorowy z dokładnością do izomorfizmu.

\section*{Wykład13.pdf}
% Przykład
Mamy $R[X]⊗R[Y] ≅ R[X,Y]$ w tym sensie, że
$W(X)⊗W(Y) = W(X)W(Y)$.

% Wniosek.
Jeśli $M_n$ to wolny $R$-moduł wymiaru $n$ o bazie $\{b_1,…,b_n\}$
i analogicznie $M_n$ wymiaru $m$ o bazie $\{ c_1, …, c_m \}$,
to $M_n ⊗ M_M$ jest wolnym $R$-modułem o bazie
$\{ b_i ⊗ c_j ∶ 1 ≤ i ≤ n, 1 ≤ j ≤ m \}$.

Iloczyn tensorowy jest co do izomorfizmu przemienny, łączny i ma element neutralny
$R$ (jako $R$-moduł).

% Uwaga 13.1.(1)
Jeśli $A$ generuje $M$ i $B$ generuje $M$,
to $A ⊗ B = \{ a⊗b ∶ a ∈ A, b ∈ B \}$
generuje $M ⊗ N$.

% Uwaga 13.1.(2); Def. 13.2
Załóżmy, że $f: M → M'$, $g: N → N'$ są $R$-liniowe.
Wtedy istnieje jedyne $h: M ⊗ N → M' ⊗ N'$ takie,
że $h(m⊗n) = f(m)⊗g(n)$.
Funkcję $h$ nazywamy iloczynem tensorowym $f$ i $g$.

% Uwaga 13.3 TODO

% Uwaga 13.4
$$M ⊗ (\bigoplus_{i∈I} N_i) ≅ \bigoplus_{i∈I} (M⊗N_i)$$

% Iloczyny zewnętrzne
Niech $V$ to przestrzeń liniowa nad $K$.
Oznaczmy $V^{⊗n} = V ⊗ … ⊗ V$.
$σ∈S_n$ działa nad $V^{⊗n}$ (permutuje współrzędne w tensorach prostych).

% Def. 13.5
Niech $x ∈ V^{⊗n}$.
Mówimy, że $x$ jest \textit{symetryczny},
gdy dla każdego $σ∈S_n$ mamy $σ(x) = x$.
Mówimy, że $x$ jest \textit{antysymetryczny},
gdy dla każdego $σ∈S_n$ mamy $σ(x) = \sgn(σ) x$.

Niech $Λ^nV$ to zbiór elementów antysymetrycznych
a $S^nV$ to zbiór elementów symetrycznych $V^{⊗n}$.
Jeśli charakterystyka ciała to zero, to są to podprzestrzenie.

Mamy $V⊗V = Λ^2V ⊕ S^2 V$, bo
$x = \frac{1}{2}(x + σ(x)) + \frac{1}{2}(x-σ(x))$.

% Uwaga 13.6 TODO

\section*{Wykład14.pdf}
% Tw. 14.1
Podmoduł modułu wolnego nad PID jest wolny niewiększego wymiaru.

% Wn. 14.2
Podmoduł PID-modułu skończenie generowanego jest skończenie generowany.

% Tw. 14.3
Załóżmy, że $M$ jest PID-modułem skończenie generowanym.
Jeśli jest on beztorsyjny, to jest wolny.
Więcej, rozkłada się on na sumę prostą części torsyjnej i jego
pewnego podmodułu wolnego.

% Def.
Niech $R$ to PID,
$p∈R$ jest nierozkładalny (a więc pierwszy),
$M$ to $R$-moduł.
Mówimy, że
\begin{enumerate}
	\item
	$m ∈ M$ jest \textit{$p$-torsyjny},
	gdy torsja $I_m = \{ r ∈ R ∶ rm = 0 \} = (p^k)$ % ideał
	dla pewnego $k > 0$;
	\item
	zbiór elementów zerowych lub $p$-torsyjnych $M$
	to \textit{$p$-prymarna składowa} $M$; %podmoduł
	\item
	$M$ jest \textit{$p$-prymarny}, gdy $M = M_p$.
\end{enumerate}

% Tw. 14.4
Niech $R$ to PID i $M$ to $R$-moduł.
Wtedy $M_p$ jest podmodułem $M_t$.
Więcej, $M_t = ⊕_{i∈I} M_{p_i}$, gdzie $p_i$ to wszystkie elementy
pierwsze $R$ z dokładnością do stowarzyszenia.

% Uwaga.
Jeśli $R$-moduł jest cykliczny $p$-prymarny, to jest izomorficzny
z ilorazem $R/(p^k)$ dla pewnego $k$.

% Tw. 14.5
Skończenie generowany moduł $p$-prymarny
jest sumą prostą modułów cyklicznych.

% Wn. 14.6
Załóżmy, że $M$ to skończenie generowany $R$-moduł $p$-prymarny.
Wtedy $$M ≅ R/(p^{k_1}) ⊕ … ⊕ R/(p^{k_l})$$
dla pewnych $1 ≤ k_1 ≤ … ≤ k_l$.
Ponadto ciąg $(k_i)$ jest wyznaczony jednoznacznie.

% Tw. 14.6 błąd w numeracji
Niech $R$ to PID i $M$ to $R$-moduł
skończenie generowany.
Wtedy $M$ się rozkłada na sumę prostą podmodułów
nierozkładalnych cyklicznych, wyznaczoną jednoznacznie.

\section*{Wykład15.pdf}

% Uwaga 15.1
Załóżmy, że $V$ to przestrzeń liniowa nad $K$ skończonego wymiaru.
Wtedy jest to skończenie generowany i torsyjny $K[X]$-moduł.
Ponadto, $K[X]$ to PID, więc $V = ⊕_{p_i} V_{p_i}$ dla pewnych $p_i∈K[X]$
i $$V_{p_i} ≅ K[X]/(f_i^{k_1}) ⊕ … ⊕ K[X]/(f_i^{k_l}) \quad (1≤k_1≤…≤k_l).$$

% TODO ...

% Wn. 15.2
(Tw. Jordana)
Załóżmy, że $V$ to przestrzeń liniowa skończonego wymiaru
nad ciałem algebraicznie domkniętym $K$ i
$ψ$ to endomorfizm liniowy $V$.
Wtedy istnieje baza Jordana $B ⊆ V$ taka,
że $m_B(ψ)$ ma postać Jordana.
Rozmiary klatek macierzy są wyznaczone jednoznacznie.

% Def. 15.3
Załóżmy, że $R$ to pierścien przemienny z $1≠0$.
$R$-algebra (przemienna)
to $R$-moduł $S$ z dodatkowym mnożeniem $·: S × S → S$ takim,
że $S$ tworzy z nim i dodawaniem modułowym pierścień (przemienny).
Ponadto musi zachodzić zgodność
$$r(s s') = (rs)s' = s(rs').$$

% Przykłady
Załóżmy, że $R$ to pierścien przemienny z $1≠0$.
$R$ jest $ℤ$-algebrą.
$R[X], R[X,Y]$ to $R$-algebry.
Jeśli $R ⊂ S$ to podpierścień z jedynką,
to $S$ jest $R$-algebrą.

% Uwaga 15.4
Jeśli $S$ jest $R$-algebrą z jednością 1,
to $η: R → S$ dana przez $η(r) = r·1$
jest homomorfizmem $R$-algebr.

Gdy $R$ jest ciałem, to $η$ jest monomorfizmem
i $R$ jest podciałem pierścienia $S$.

Gdy $S$ to pierścień z jedynką i $R⊆S$ to podciało,
to $S$ jest $R$-algebrą.

Załóżmy, że $S$ to $R$-algebra z jedynką i $M$ to $R$-moduł.
Wtedy $S ⊗_R M$ to $R$-moduł, lecz także $S$-moduł.
Istnieje jedyna operacja mnożenia (na pierwszym argumencie tensora bazowego).

% Przykład 1.
Jeśli $G$ to $ℤ$-moduł, to $ℚ ⊗_Z G$ to $ℚ$-moduł.

Jeśli $V$ to przestrzeń liniowa nad $ℝ$,
to $ℂ ⊗_ℝ V$ to przestrzeń liniowa nad $ℂ$ (kompleksyfikacja $V$).

% ---
Jeśli $S_1, S_2$ to $R$-algebry z jedynką, to ich iloczyn tensorowy nad $R$ też.

% Tw. 15.5
(Nullstellensatz Hilberta)
Niech $I ◁ K[\bar{X}]$ i $f ∈ K[\bar{X}]$ takie, że
$Z_{\hat{K}}(I) ⊆ Z_{\hat{K}}(f)$, gdzie
$ℤ_L(I)$ to zbiór wspólnych pierwiastków $I$.

% Wn. 15.6
Załóżmy, że $K$ to ciało algebraicznie domknięte takie, że
układ równań wielomianowych $f_1(\bar{x}) = … = f_k(\bar{x}) = 0$,
gdzie $f_i ∈ K[\bar{X}]$,
nie ma rozwiązań w $K$.
Wtedy $1 ∈ (f_1, …, f_k)$.
$f_i ∈ K[\bar{X}]$.
\end{document}
