\documentclass[a4paper, 12pt]{article}
\usepackage[utf8]{inputenc}
\usepackage{silence}
\usepackage{polski}
%\usepackage{parskip}
\usepackage{amsmath,amsfonts,amssymb,amsthm}
\usepackage{mathtools}
\usepackage{enumitem}
%\usepackage{pgfplots}
%\pgfplotsset{compat=1.16}
\usepackage{newunicodechar}
\usepackage{etoolbox}
\usepackage[margin=1.2in]{geometry}
\setcounter{secnumdepth}{0}

\title{Algebra 2R na 05.03}
\author{Wiktor Kuchta}
\date{\vspace{-4ex}}

\DeclareMathOperator{\im}{Im}
\DeclareMathOperator{\rank}{rank}
\DeclareMathOperator{\Lin}{Lin}
\DeclareMathOperator{\sgn}{sgn}
\DeclareMathOperator{\Char}{char}
\newcommand{\N}{\mathbb{N}}
\newcommand{\Z}{\mathbb{Z}}
\newcommand{\Q}{\mathbb{Q}}
\newcommand{\R}{\mathbb{R}}
\newcommand{\C}{\mathbb{C}}
\newcommand{\inner}[2]{( #1 \, | \, #2)}
\newcommand{\norm}[1]{\left\lVert #1 \right\rVert}
\newcommand{\modulus}[1]{\left| #1 \right|}
\newcommand{\abs}{\modulus}
\newtheorem{theorem}{Twierdzenie}
\newtheorem{lemat}{Lemat}
\newcommand{\ol}{\overline}
\DeclareMathOperator{\tr}{tr}
\DeclareMathOperator{\diag}{diag}
\newcommand{\+}{\enspace}
\newcommand{\sump}{\sideset{}{'}{∑}} % sum prime
\newcommand{\sumb}{\sideset{}{"}{∑}} % sum bis

\newunicodechar{∅}{\emptyset} % Digr /0
\newunicodechar{∞}{\infty} % Digr 00
\newunicodechar{∂}{\partial} % Digr dP
\newunicodechar{α}{\alpha}
\newunicodechar{β}{\beta}
\newunicodechar{ξ}{\xi} % Digr c*
\newunicodechar{δ}{\delta} % Digr d*
\newunicodechar{ε}{\varepsilon}
\newunicodechar{φ}{\varphi}
\newunicodechar{θ}{\theta} % Digr h*
\newunicodechar{λ}{\lambda}
\newunicodechar{μ}{\mu}
\newunicodechar{π}{\pi}
\newunicodechar{σ}{\sigma}
\newunicodechar{τ}{\tau}
\newunicodechar{ω}{\omega}
\newunicodechar{η}{\eta} % Digr y*
\newunicodechar{Δ}{\Delta}
\newunicodechar{Φ}{\Phi} % Digr F*
\newunicodechar{Π}{\Pi}
\newunicodechar{Ψ}{\Psi} % digr Q*
\newunicodechar{ℕ}{\N} % Digr NN 8469 nonstandard
\newunicodechar{ℤ}{\Z} % Digr ZZ 8484 nonstandard
\newunicodechar{ℚ}{\Q} % Digr QQ 8474 nonstandard
\newunicodechar{ℝ}{\R} % Digr RR 8477 nonstandard
\newunicodechar{ℂ}{\C} % Digr CC 8450 nonstandard
\newunicodechar{∑}{\sum}
\newunicodechar{∏}{\prod}
\newunicodechar{∫}{\int}
\newunicodechar{∓}{\mp}
\newunicodechar{⌈}{\lceil} % Digr <7
\newunicodechar{⌉}{\rceil} % Digr >7
\newunicodechar{⌊}{\lfloor} % Digr 7<
\newunicodechar{⌋}{\rfloor} % Digr 7>
\newunicodechar{≅}{\cong} % Digr ?=
\newunicodechar{≡}{\equiv} % Digr 3=
\newunicodechar{◁}{\triangleleft} % Digr Tl
\newunicodechar{▷}{\triangleright} % Digr Tr
\newunicodechar{≤}{\le}
\newunicodechar{≥}{\ge}
\newunicodechar{≪}{\ll} % Digr <*
\newunicodechar{≫}{\gg} % Digr *>
\newunicodechar{≠}{\ne}
\newunicodechar{⊆}{\subseteq} % Digr (_
\newunicodechar{⊇}{\supseteq} % Digr _)
\newunicodechar{⊂}{\subset} % Digr (C
\newunicodechar{⊃}{\supset} % Digr C)
\newunicodechar{∩}{\cap} % Digr (U
\newunicodechar{∪}{\cup} % Digr )U
\newunicodechar{∼}{\sim} % Digr ?1
\newunicodechar{∈}{\in} % Digr (-
\newunicodechar{∋}{\ni} % Digr -)
\newunicodechar{∇}{\nabla} % Digr NB
\newunicodechar{∃}{\exists} % Digr TE
\newunicodechar{∀}{\forall} % Digr FA
\newunicodechar{∧}{\wedge} % Digr AN
\newunicodechar{∨}{\vee} % Digr OR
\newunicodechar{⊥}{\bot} % Digr -T
\newunicodechar{⊤}{\top} % Digr TO 8868 nonstandard
\newunicodechar{⇒}{\implies} % Digr =>
\newunicodechar{↦}{\mapsto} % Digr T> 8614 nonstandard
\newunicodechar{∘}{\circ} % Digr Ob

% cursed
\WarningFilter{newunicodechar}{Redefining Unicode}
\newunicodechar{·}{\ifmmode\cdot\else\textperiodcentered\fi} % Digr .M
\newunicodechar{×}{\ifmmode\times\else\texttimes\fi} % Digr *X
\newunicodechar{→}{\ifmmode\to\else\textrightarrow\fi} % Digr ->
\newunicodechar{⟨}{\ifmmode\langle\else\textlangle\fi} % Digr LA 10216 nonstandard
\newunicodechar{⟩}{\ifmmode\rangle\else\textrangle\fi} % Digr RA 10217 nonstandard
\newunicodechar{…}{\ifmmode\dots\else\textellipsis\fi} % Digr .,
\newunicodechar{±}{\ifmmode\pm\else\textpm\fi} % Digr +-

% https://tex.stackexchange.com/a/438184
% https://tex.stackexchange.com/q/528480
\newunicodechar{∶}{\mathbin{\text{:}}}
\def\newcolon{%
  \nobreak\mskip2mu\mathpunct{}\nonscript\mkern-\thinmuskip{\text{:}}%
  \mskip 6mu plus 1 mu \relax}
\mathcode`:="8000
{\catcode`:=\active \global\let:\newcolon}
% colon: for types; ratio∶ (digr :R) for relations (set builder)


\begin{document}

\maketitle

\section*{1/1D}
Weźmy $z ∈ ℂ \setminus ℝ$.
Skoro $z$ nie leży na prostej rzeczywistej, to $\arg z ≠ \arg z^2$.
To oznacza, że $z$ i $z^2$ traktowane w naturalny sposób jako wektory w
przestrzeni liniowej $ℝ^2$ nad $ℝ$ są liniowo niezależne i rozpinają całą
przestrzeń.
Ten naturalny izomorfizm między $ℂ$ i $ℝ^2$ zachowuje mnożenie przez skalary
rzeczywiste, więc wracając do $ℂ$ otrzymujemy, że każdą liczbę zespoloną
da się zapisać w postaci $xz + yz^2$ dla pewnych $x,y ∈ ℝ$.

Wszystkie liczby powyższej postaci $xz + yz^2$ są generowane przez $z$ nad $ℝ$,
tzn. należą do $ℝ[z]$.
Z zamkniętości ciała $ℂ$ na dodawanie i mnożenie mamy też $ℝ[z] ⊆ ℂ$,
więc otrzymujemy $ℝ[z] = ℂ$.

\section*{1/3}
Załóżmy, że $K ⊂ L$ jest rozszerzeniem ciał oraz $f_1, …, f_m ∈ K[X_1, …, X_n]$
są stopnia $1$, czyli mają postać
$$f_i = a_{i1}X_1 + … + a_{in}X_n - b_{i}, \quad \text{gdzie }a_{ij}, b_i ∈ K.$$

\subsection*{(a)D}
Problem rozwiązania układu $f_1 = … = f_m = 0$ możemy przedstawić jako
$$
\begin{bmatrix}
	a_{11} & \dots  & a_{1n} \\
	\vdots & \ddots & \vdots \\
	a_{m1} & \dots  & a_{mn}
\end{bmatrix}
\begin{bmatrix}
	x_{1} \\
	\vdots \\
	x_{n} \\
\end{bmatrix}
=
\begin{bmatrix}
	b_{1} \\
	\vdots \\
	b_{n} \\
\end{bmatrix}.
$$
Korzystając z algorytmu eliminacji Gaussa macierz rozszerzoną
$\begin{bmatrix}A \big\vert b\end{bmatrix}$
możemy sprowadzić do postaci schodkowej,
która opisuje układ równań równoważny początkowemu.
Z postaci schodkowej można bezpośrednio odczytać, czy układ jest niesprzeczny
(nie ma wiersza postaci $[\, 0 \+ … \+ 0 \, \vert \, b \,]$, gdzie $b ≠ 0$) i
jakie ma wtedy zmienne swobodne (być może ich nie ma).

Jeśli układ ma rozwiązanie w $L$, to ten układ jest niesprzeczny.
Możemy otrzymać inne rozwiązanie przypisując zmiennym swobodnym elementy z $K$.
Współczynniki macierzy także są w $K$,
a korzystając z nich i wartości zmiennych swobodnych się oblicza wartości
pozostałych zmiennych,
które więc też będą w $K$.

\subsection*{(b)}
Dla układu sprzecznego oczywiście nie ma rozwiązania ogólnego.

Dla układu niesprzecznego to też nie musi być prawda, np.
rozważmy równanie $X_1 - X_2 = 0$.
Weźmy pewne jego rozwiązanie $(x, x)$.
Wtedy $X_1 - x$ jest wielomianem, który się zeruje w $(x,x)$, ale
nie jest w ideale $(X_1 - X_2)$, więc to rozwiązanie nie jest ogólne.

\iffalse
\section*{1/6D}
Załóżmy, że $f ∈ K[X]$ jest nierozkładalny stopnia $n > 0$,
$\Char K = 0$, $L$ jest ciałem rozkładu wielomianu $f$ nad $K$.

Ciało $L$ możemy utożsamiać z $K[X] / (f)$.
Zatem jego elementy są postaci
$$∑_{k=0}^{n-1} c_k α^k,$$
gdzie $c_k ∈ K$, a $α = X + f$ to pierwiastek $f$.

Ciało $L$ jest nieskończone,
więc zawiera podciało izomorficzne z $ℚ$ zachowywane przez automorfizmy.
\fi

\section*{1/5aD}
Z definicji rozwiązania ogólnego wiemy, że $(1,1)$ jest rozwiązaniem
ogólnym $X_1^2 - X_2^3$ nad $ℚ$ wtedy i tylko wtedy, gdy
$$\left\{ g ∈ ℚ[X_1, X_2] ∶ g(1, 1) = 0 \right\} = ℚ[X_1, X_2](X_1^2 - X_2^3).$$
Ta równość nie zachodzi, bo $X_1 - X_2$ należy do lewego zbioru,
ale nie do prawego.








\end{document}
