\documentclass[a4paper, 12pt]{article}
\usepackage[utf8]{inputenc}
\usepackage{silence}
\usepackage{polski}
\usepackage{parskip}
\usepackage{amsmath,amsfonts,amssymb,amsthm}
\usepackage{mathtools}
\usepackage{enumitem}
%\usepackage{pgfplots}
%\pgfplotsset{compat=1.16}
\usepackage{newunicodechar}
\usepackage{etoolbox}
\usepackage[margin=1.2in]{geometry}
\usepackage{algorithm}
\usepackage{algorithmicx}
\usepackage{algpseudocode}
\setcounter{secnumdepth}{0}

\title{Algebra 2R lista 3}
\author{Wiktor Kuchta}
\date{\vspace{-4ex}}

\DeclareMathOperator{\im}{Im}
\DeclareMathOperator{\rank}{rank}
\DeclareMathOperator{\Lin}{Lin}
\DeclareMathOperator{\sgn}{sgn}
\DeclareMathOperator{\Char}{char}
\DeclareMathOperator{\ord}{ord}
\newcommand{\N}{\mathbb{N}}
\newcommand{\Z}{\mathbb{Z}}
\newcommand{\Q}{\mathbb{Q}}
\newcommand{\R}{\mathbb{R}}
\newcommand{\C}{\mathbb{C}}
\newcommand{\inner}[2]{( #1 \, | \, #2)}
\newcommand{\norm}[1]{\left\lVert #1 \right\rVert}
\newcommand{\modulus}[1]{\left| #1 \right|}
\newcommand{\abs}{\modulus}
\newtheorem{theorem}{Twierdzenie}
\newtheorem{lemat}{Lemat}
\newcommand{\ol}{\overline}
\DeclareMathOperator{\tr}{tr}
\DeclareMathOperator{\diag}{diag}
\newcommand{\+}{\enspace}
\newcommand{\sump}{\sideset{}{'}{∑}} % sum prime
\newcommand{\sumb}{\sideset{}{"}{∑}} % sum bis

\newunicodechar{∅}{\emptyset} % Digr /0
\newunicodechar{∞}{\infty} % Digr 00
\newunicodechar{∂}{\partial} % Digr dP
\newunicodechar{α}{\alpha}
\newunicodechar{β}{\beta}
\newunicodechar{ξ}{\xi} % Digr c*
\newunicodechar{δ}{\delta} % Digr d*
\newunicodechar{ε}{\varepsilon}
\newunicodechar{φ}{\varphi}
\newunicodechar{θ}{\theta} % Digr h*
\newunicodechar{λ}{\lambda}
\newunicodechar{μ}{\mu}
\newunicodechar{π}{\pi}
\newunicodechar{σ}{\sigma}
\newunicodechar{τ}{\tau}
\newunicodechar{ω}{\omega}
\newunicodechar{η}{\eta} % Digr y*
\newunicodechar{Δ}{\Delta}
\newunicodechar{Θ}{\Theta}
\newunicodechar{Φ}{\Phi} % Digr F*
\newunicodechar{Π}{\Pi}
\newunicodechar{Ψ}{\Psi} % digr Q*
\newunicodechar{ℕ}{\N} % Digr NN 8469 nonstandard
\newunicodechar{ℤ}{\Z} % Digr ZZ 8484 nonstandard
\newunicodechar{ℚ}{\Q} % Digr QQ 8474 nonstandard
\newunicodechar{ℝ}{\R} % Digr RR 8477 nonstandard
\newunicodechar{ℂ}{\C} % Digr CC 8450 nonstandard
\newunicodechar{∑}{\sum}
\newunicodechar{∏}{\prod}
\newunicodechar{∫}{\int}
\newunicodechar{∓}{\mp}
\newunicodechar{⌈}{\lceil} % Digr <7
\newunicodechar{⌉}{\rceil} % Digr >7
\newunicodechar{⌊}{\lfloor} % Digr 7<
\newunicodechar{⌋}{\rfloor} % Digr 7>
\newunicodechar{≅}{\cong} % Digr ?=
\newunicodechar{≡}{\equiv} % Digr 3=
\newunicodechar{◁}{\triangleleft} % Digr Tl
\newunicodechar{▷}{\triangleright} % Digr Tr
\newunicodechar{≤}{\le}
\newunicodechar{≥}{\ge}
\newunicodechar{≪}{\ll} % Digr <*
\newunicodechar{≫}{\gg} % Digr *>
\newunicodechar{≠}{\ne}
\newunicodechar{⊆}{\subseteq} % Digr (_
\newunicodechar{⊇}{\supseteq} % Digr _)
\newunicodechar{⊂}{\subset} % Digr (C
\newunicodechar{⊃}{\supset} % Digr C)
\newunicodechar{∩}{\cap} % Digr (U
\newunicodechar{∪}{\cup} % Digr )U
\newunicodechar{∼}{\sim} % Digr ?1
\newunicodechar{≈}{\approx} % Digr ?2
\newunicodechar{∈}{\in} % Digr (-
\newunicodechar{∋}{\ni} % Digr -)
\newunicodechar{∇}{\nabla} % Digr NB
\newunicodechar{∃}{\exists} % Digr TE
\newunicodechar{∀}{\forall} % Digr FA
\newunicodechar{∧}{\wedge} % Digr AN
\newunicodechar{∨}{\vee} % Digr OR
\newunicodechar{⊥}{\bot} % Digr -T
\newunicodechar{⊤}{\top} % Digr TO 8868 nonstandard
\newunicodechar{⇒}{\implies} % Digr =>
\newunicodechar{⇐}{\impliedby} % Digr <=
\newunicodechar{⇔}{\iff} % Digr ==
\newunicodechar{↔}{\leftrightarrow} % Digr <>
\newunicodechar{↦}{\mapsto} % Digr T> 8614 nonstandard
\newunicodechar{∘}{\circ} % Digr Ob

% cursed
\WarningFilter{newunicodechar}{Redefining Unicode}
\newunicodechar{·}{\ifmmode\cdot\else\textperiodcentered\fi} % Digr .M
\newunicodechar{×}{\ifmmode\times\else\texttimes\fi} % Digr *X
\newunicodechar{→}{\ifmmode\rightarrow\else\textrightarrow\fi} % Digr ->
\newunicodechar{←}{\ifmmode\leftarrow\else\textleftarrow\fi} % Digr ->
\newunicodechar{⟨}{\ifmmode\langle\else\textlangle\fi} % Digr LA 10216 nonstandard
\newunicodechar{⟩}{\ifmmode\rangle\else\textrangle\fi} % Digr RA 10217 nonstandard
\newunicodechar{…}{\ifmmode\dots\else\textellipsis\fi} % Digr .,
\newunicodechar{±}{\ifmmode\pm\else\textpm\fi} % Digr +-

% https://tex.stackexchange.com/a/438184
% https://tex.stackexchange.com/q/528480
\newunicodechar{∶}{\mathbin{\text{:}}}
\def\newcolon{%
  \nobreak\mskip2mu\mathpunct{}\nonscript\mkern-\thinmuskip{\text{:}}%
  \mskip 6mu plus 1 mu \relax}
\mathcode`:="8000
{\catcode`:=\active \global\let:\newcolon}
% colon: for types; ratio∶ (digr :R) for relations (set builder)


\begin{document}

\maketitle

\section*{1/5e}
Niech $Φ: ℂ(X) → ℂ(X)$, $Φ(f) = f\left(\frac{X}{X-1}\right)$.
Aby to była poprawna definicja, musimy pokazać, że $Φ$ jest określone na całej dziedzinie,
tzn. $\frac{X}{X-1}$ nie jest pierwiastkiem żadnego niezerowego wielomianu $w∈ℂ[X]$.
Funkcja $z ↦ \frac{z}{z-1}$ jest inwolucją, a zatem bijekcją $ℂ \setminus \{ 1 \}$.
Zatem $w\left( \frac{X}{X-1} \right)$ ma dokładnie tyle samo różnych pierwiastków co $w$, być może poza $1$.
To oznacza, że $w\left( \frac{X}{X-1} \right)$ jest zerem dokładnie wtedy, co $w$.

Pokazaliśmy, że homomorfizm ewaluacji $φ_{\frac{X}{X-1}}: ℂ[X] → ℂ(X)$ jest różnowartościowy,
zatem rozszerza się on do jedynego homomorfizmu z ciała ułamków $ℂ[X]$.
Tym homomorfizmem jest dokładnie $Φ$.
Powyższy argument z inwolucją pokazuje też, że $Φ(Φ(f)) = f$, zatem $Φ$ jest
automorfizmem.

Wiemy z 1/5d, że każdy $W ∈ ℚ[X_1, X_2]$ zerujący się w $(X^3, X^2)$
jest podzielny przez $X_1^2 - X_2^3$.

Weźmy wielomian $W ∈ ℚ[X_1, X_2]$ taki, że
$W\left( \frac{X^3}{(X-1)^3}, \frac{X^2}{(X-1)^2} \right) = 0$.
Wtedy skoro $Φ$ jest identycznością na $ℚ$, to mamy
\begin{align*}
Φ\left(W\left( \frac{X^3}{(X-1)^3}, \frac{X^2}{(X-1)^2} \right)\right)
&= W\left( Φ\left(\frac{X^3}{(X-1)^3}\right), Φ\left(\frac{X^2}{(X-1)^2}\right) \right) \\
&= W(X^3, X^2) = 0.
\end{align*}
Zatem $W$ jest podzielny przez $X_1^2 - X_2^3$.


\section*{3/3aD}
Załóżmy, że $K ⊃ F(p)$ jest skończonym rozszerzeniem ciała $F(p)$,
charakterystyki $p$.
Załóżmy, że $a ∈ K$ jest pierwiastkiem pierwotnym stopnia $m$ z jedynki.

Pierwotnym pierwiastkom stopnia $m$ w $F(p^n)$ odpowiadają elementy rzędu $m$ w
$F(p^n)^*$.
Rząd elementu dzieli rząd grupy,
zatem takie pierwiastki istnieją tylko jeśli $m$ dzieli $p^n-1$.

Ustalmy najmniejsze $n$ takie, że $m \mid p^n-1$.
Ciało $F(p)(a)$ musi być mocy co najmniej $p^n$,
inaczej nie mogłoby ono zawierać pierwiastka pierwotnego stopnia $m$ z jedynki.

Niech $g$ to generator $F(p^n)^*$.
Niech $r = g^{\frac{p^n-1}{m}}$, wtedy
$$
\ord(r)
=\ord\left(g^{\frac{p^n-1}{m}}\right)
= \frac{\ord(g)}{\gcd\left(\frac{p^n-1}{m}, \ord(g)\right)}
= \frac{p^n-1}{\gcd\left(\frac{p^n-1}{m}, p^n-1\right)}
= \frac{p^n-1}{\frac{p^n-1}{m}}
= m.
$$
Potęgi $r^0, r^1, …, r^{m-1}$ to wszystkie pierwiastki stopnia $m$ z jedynki,
więc wśród ich jest $a$.
Zatem $F(p^n) = F(p)(a)$.
Stopień $a$ nad $F(p)$ to
$[F(p^n) ∶ F(p)] = n$.

\section*{3/4aD}
Niech $x = \sqrt{2}+\sqrt{3}$.
$$x^2 = 5 + 2\sqrt{6}$$
$$x^2-5= 2\sqrt{6}$$
$$x^4-10x^2+25=24$$
$$x^4-10x^2+1=0$$
Więc znaleźliśmy wielomian zerujący się w $\sqrt{2}+\sqrt{3}$.

$ℚ(\sqrt{2}+\sqrt{3})$ jest podciałem $ℚ(\sqrt{2},\sqrt{3})$.
Zauważmy, że
$$\frac{1}{\sqrt{2}+\sqrt{3}} = \frac{\sqrt{2}-\sqrt{3}}{2-3}=\sqrt{3}-\sqrt{2},$$
więc w $ℚ(\sqrt{2}+\sqrt{3})$ są $2\sqrt{3}$ i dalej $\sqrt{2}$.
Zatem mamy też zawieranie w drugą stronę $ℚ(\sqrt{2},\sqrt{3})⊆ℚ(\sqrt{2}+\sqrt{3})$.

Wiemy z dowodu 1/8, że $[ℚ(\sqrt{2}, \sqrt{3}) ∶ ℚ] = 4$.
Wielomian $x^4-10x^2+1$ jest unormowany stopnia $4$, zatem jest minimalny dla $x$ nad $ℚ$.

\section*{3/5D}
%Jeśli $a∈ℝ$ algebraiczna stopnia $n>1$ (nad $ℚ$),
%to $∃C>0 ∀r=p/q∈ℚ$
%\abs{a-p/q}≥C/qⁿ$

Liczba
$$a = ∑_{k=1}^∞ 2^{-k!}$$
ma w zapis dwójkowy taki, że $n$-ta cyfra po przecinku
jest jedynką dokładnie kiedy $n$ jest silnią pewnej liczby.

Załóżmy, że $a$ algebraiczna stopnia $d$.
Weźmy $C>0$.
Niech $q_n = 2^{n!}$.
Dla pewnego $m$ mamy $q_m^{-m} < C q_m^{-d}$.
Niech $p = q_m ∑_{k=1}^m 2^{-k!}$.
Wtedy
$$
a - \frac{p}{q_m}
= a - ∑_{k=1}^m \frac{1}{2^{k!}}
= ∑_{k=m+1}^∞ \frac{1}{2^{k!}}
< \frac{1}{q_m^m},$$
bo $q_m^{-m}=2^{-m!m}$ ma w zapisie binarnym jedynkę na $(m!m)$-tym miejscu
po przecinku,
a ostatnia suma ma pierwszą jedynkę dopiero na $(m+1)!$-tym miejscu po
przecinku.
Otrzymujemy sprzeczność z lematem Liouville'a, zatem $a$ nie jest algebraiczna.

\end{document}
