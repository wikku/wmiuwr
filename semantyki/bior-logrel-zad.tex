\documentclass[a4paper, 12pt]{article}
\usepackage[utf8]{inputenc}
\usepackage{silence}
\usepackage{polski}
\usepackage{parskip}
\usepackage{amsmath,amsfonts,amssymb,amsthm}
\usepackage{newunicodechar}
\usepackage[margin=1.2in]{geometry}
\usepackage{mathpartir}
\setcounter{secnumdepth}{0}
\pagenumbering{gobble}

\title{Biortogonalne relacje logiczne — zadania}
\author{Wiktor Kuchta}
\date{\vspace{-2ex}}

\newcommand{\subst}[2]{\{#1/#2\}}
\newcommand{\cont}{\textsf{cont}}
\newcommand{\Cont}{\textsf{Cont}}
\newcommand{\Letcc}{\textsf{letcc}}
\newcommand{\Throw}{\textsf{throw}}
\newcommand{\Obs}{\mathrm{Obs}}

\newunicodechar{∷}{::} % Digr ::
\newunicodechar{□}{\square} % Digr OS
\newunicodechar{│}{\mid} % Digr vv
\newunicodechar{λ}{\lambda}
\newunicodechar{Γ}{\Gamma}
\newunicodechar{⊆}{\subseteq} % Digr (_
\newunicodechar{∈}{\in} % Digr (-
\newunicodechar{∀}{\forall} % Digr FA
\newunicodechar{⊥}{\bot} % Digr -T
\newunicodechar{⊢}{\vdash} % Digr \- 8866 nonstandard

% cursed
\WarningFilter{newunicodechar}{Redefining Unicode}
\newunicodechar{→}{\ifmmode\rightarrow\else\textrightarrow\fi} % Digr ->

\begin{document}

\maketitle

\hrulefill

{\bf Zadanie 1.}
Dla ustalonego języka (który ma termy $t$, kotermy $E$ i pojęcie złączania ich
w pełną konfigurację $E[t]$) i zbioru jego konfiguracji $\Obs$, dopełnienia
ortogonalne definiujemy następująco:

Jeśli $S$ to zbiór termów, to $S^{⊥} = \{E │ ∀t∈S.\,E[t] ∈ \Obs\}$.

Jeśli $S$ to zbiór kotermów, to $S^{⊥} = \{t │ ∀E∈S.\, E[t] ∈ \Obs\}$.

Załóżmy, że $S$ to zbiór termów albo kotermów.
Pokaż, że $S ⊆ S^{⊥⊥}$ i $S^{⊥⊥⊥} ⊆ S^{⊥}$.

{\bf Zadanie 2.}
Rozważamy rachunek lambda call-by-value z typami prostymi rozszerzony o
kontynuacje niedelimitowane.
\begin{align*}
	v &∷= x │ λx.\,t │ \cont\;E \\
	t &∷= v │ t\;t │ \Letcc\;x.\,t │ \Throw\;t\;t \\
	E &∷= □ │ E\;t │ v\;E │ \Throw\;E\;t │ \Throw\;v\;E
\end{align*}
\begin{mathpar}
	E[(λx.\,t)\;v] → E[t\subst{v}{x}]

	E[\Letcc\;x.\,t] → E[t\subst{\cont\;E}{x}]

	E[\Throw\;(\cont\;E')\;v] → E'[v]
\end{mathpar}
Intencja jest oczywiście taka, że kontekst ewaluacyjny zapakowany w wartość postaci $\cont\;E$ jest ,,zamrożony''
tzn. występujące w nim dziury $□$ nie podlegają podstawieniom i nie są miejscami redukcji.

Interesuje nas terminacja ewaluacji.

Zdefiniuj odpowiednie relacje logiczne i udowodnij lematy kompatybilności dla
reguł wprowadzających $\Letcc$ i $\Throw$:
\begin{mathpar}
	T ∷= B │ T → T │ \Cont\;T

	\inferrule
		{Γ, x : \Cont\;T ⊢ t : T}
		{Γ ⊢ \Letcc\;x.\,t:T}

	\inferrule
		{Γ ⊢ t_1 : \Cont\;T \\ Γ ⊢ t_2 : T}
		{Γ ⊢ \Throw\;t_1\;t_2 : T'}
\end{mathpar}


\end{document}
