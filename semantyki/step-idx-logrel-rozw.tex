\documentclass[a4paper, 12pt]{article}
\usepackage[utf8]{inputenc}
\usepackage{silence}
\usepackage{polski}
\usepackage{parskip}
\usepackage{amsmath,amsfonts,amssymb,amsthm}
\usepackage{newunicodechar}
\usepackage[margin=1.2in]{geometry}
\setcounter{secnumdepth}{0}
\usepackage{stmaryrd}
\usepackage{mathpartir}

\title{}
\author{Wiktor Kuchta}
\date{\vspace{-4ex}}

\newcommand{\E}{\mathcal{E}}
\newcommand{\V}{\mathcal{V}}
\newcommand{\bool}{\mathtt{bool}}
\newcommand{\true}{\mathtt{true}}
\newcommand{\false}{\mathtt{false}}
\newcommand{\pack}{\mathtt{pack}}
\newcommand{\fold}{\mathtt{fold}}
\newcommand{\CVal}{\mathrm{CVal}}
\newcommand{\SemType}{\textit{SemType}}

\newcommand{\N}{\mathbb{N}}
\newtheorem{theorem}{Twierdzenie}
\newtheorem{lemat}{Lemat}

\newunicodechar{α}{\alpha}
\newunicodechar{δ}{\delta} % Digr d*
\newunicodechar{γ}{\gamma} % Digr g*
\newunicodechar{λ}{\lambda}
\newunicodechar{μ}{\mu}
\newunicodechar{τ}{\tau}
\newunicodechar{Δ}{\Delta}
\newunicodechar{Γ}{\Gamma}
\newunicodechar{Λ}{\Lambda}
\newunicodechar{ℕ}{\N} % Digr NN 8469 nonstandard
\newunicodechar{≤}{\le}
\newunicodechar{∈}{\in} % Digr (-
\newunicodechar{∇}{\nabla} % Digr NB
\newunicodechar{∃}{\exists} % Digr TE
\newunicodechar{∀}{\forall} % Digr FA
\newunicodechar{∧}{\wedge} % Digr AN
\newunicodechar{⊨}{\models} % Digr \= 8872 nonstandard
\newunicodechar{⇒}{\implies} % Digr =>
\newunicodechar{↦}{\mapsto} % Digr T> 8614 nonstandard
\newunicodechar{⟦}{\llbracket} % Digr [[ 10214 nonstandard (needs pkg stmaryrd)
\newunicodechar{⟧}{\rrbracket} % Digr ]] 10215 nonstandard
\newunicodechar{│}{\mid} % Digr vv

% cursed
\WarningFilter{newunicodechar}{Redefining Unicode}
\newunicodechar{×}{\ifmmode\times\else\texttimes\fi} % Digr *X
\newunicodechar{→}{\ifmmode\rightarrow\else\textrightarrow\fi} % Digr ->

\begin{document}

\maketitle

\section*{Zadanie 1.}

\begin{lemat}[Monotoniczność]
	Jeśli $(k,v) ∈ \V⟦τ⟧ δ$,
	to $∀j≤k.\,(j,v) ∈ \V⟦τ⟧δ$.
	Jeśli $(k,e) ∈ \E⟦τ⟧δ$, to $∀j≤k.\,(j,e) ∈ \E⟦τ⟧δ$.
\end{lemat}

\begin{proof}

Indukcja względem struktury typu $τ$.

Pierwsza część: załóżmy, że $(k,v) ∈ \V⟦τ⟧δ$.

Przypadek $τ=α$. Mamy $\V⟦α⟧δ = δ(α) ∈ \SemType$.
Teza wynika z definicji: $$\SemType =
\{S ∈ \mathcal{P}(ℕ × \CVal) │ ∀(k, v) ∈ S.\,∀j < k.\,(j,v) ∈ S\}.$$

Przypadek $τ=\bool$. Mamy $\V⟦\bool⟧δ = \{(k,b) │ k ∈ ℕ ∧ b ∈ \{\true, \false\}\}$.
Zatem teza też wynika z definicji.

Przypadek $τ=∀α.\,τ$. Mamy
$$\V⟦∀α.\,τ⟧δ = \{(k, Λ.\,e) │ (Λ.\,e) ∈ \CVal ∧ ∀S ∈ \SemType.\,(k,e) ∈ \E⟦τ⟧(δ, α↦S)\}.$$
Zatem $v$ ma postać $Λ.\,e$. Weźmy $j≤k$. Chcemy pokazać, że
$(j, Λ.\,e) ∈ \V⟦∀α.\,τ⟧δ$.
Z założenia indukcyjnego lemat zachodzi dla $(k,e) ∈ \E⟦τ⟧(δ, α↦S)$ dla dowolnego $S ∈ \SemType$.
Zatem $(j,e) ∈ \E⟦τ⟧(δ, α↦S)$ dla dowolnego $S ∈ \SemType$,
a więc tezę mamy z definicji.

Przypadek $τ=∃α.\,τ$. Mamy
$$\V⟦∃α.\,τ⟧δ = \{(k, \pack\;v) │ ∃S ∈ \SemType.\,(k,v) ∈ \V⟦τ⟧(δ, α↦S)\}.$$
Zatem $v$ ma postać $\pack\;v'$.
Weźmy $S∈\SemType$ świadczące o $(k, \pack\;v') ∈ \V⟦∃α.\,τ⟧δ$
i $j≤k$.
Chcemy pokazać, że $(j,\pack\;v') ∈ \V⟦∃α.\,τ⟧δ$.
Z założenia indukcyjnego lemat zachodzi dla $(k,v') ∈ \V⟦τ⟧(δ, α↦S)$,
zatem $(j,v') ∈ \V⟦τ⟧(δ, α↦S)$ i teza zachodzi z definicji.

Przypadek $τ=μα.\,τ$. Mamy
$$\V⟦μα.\,τ⟧δ = \{(k, \fold\;v) │ v ∈ \CVal ∧ ∀j<k.\,(j,v) ∈ \V⟦τ[μα.\,τ/α]⟧δ\}.$$
Weźmy $j≤k$, wtedy $(j, v) ∈ \V⟦μα.\,τ⟧δ$ wynika łatwo z definicji.

Przypadek $τ=τ_1 → τ_2$. Mamy
\begin{align*}
	\V⟦τ_1 → τ_2⟧δ = \{(k, λx.\,e) │\ &(λx.\,e) ∈ \CVal \ ∧ \\ &∀j ≤ k.\,∀v.\,(j,v) ∈ \V⟦τ_1⟧δ ⇒ (j, e[v/x]) ∈ \E⟦τ_2⟧δ\}.
\end{align*}
Weźmy $j≤k$, wtedy $(j, v) ∈ \V⟦τ_1 → τ_2⟧δ$ wynika łatwo z definicji.

Druga część: teraz załóżmy, że $(k, e) ∈ \E⟦τ⟧δ$.
Mamy
$$\E⟦τ⟧δ = \{(k, e) │ ∀j<k, e'.\,e \searrow^j e' ⇒ (k-j, e') ∈ \V⟦τ⟧δ\}.$$
Weźmy $j≤k$.
Chcemy pokazać, że $(j,e) ∈ \E⟦τ⟧δ$.
Weźmy $j' < j$.
Wiemy, że $e \searrow^{j'} e'$ implikuje $(k-j', e') ∈ \V⟦τ⟧δ$,
a chcemy pokazać, że implikuje $(j-j', e') ∈ \V⟦τ⟧δ$.
Ale to wynika z monotoniczności dla $\V⟦τ⟧δ$, którą przed chwilą udowodniliśmy.

\end{proof}

\section*{Zadanie 2.}
\begin{lemat}[Kompatybilność aplikacji]
	$$
	\inferrule
		{Δ; Γ ⊨ e_1 : τ_1 → τ_2 \\ Δ; Γ ⊨ e_2 : τ_1}
		{Δ; Γ ⊨ e_1\,e_2 : τ_2 }
	$$
\end{lemat}
\begin{proof}
	Weźmy $δ ∈ \mathcal{D}⟦Δ⟧$ i $(k,γ)∈\mathcal{G}⟦Γ⟧δ$.
	Chcemy pokazać $(k, γ(e_1\,e_2)) ∈ \E⟦τ_2⟧δ$.

	Wiemy, że $(k, γ(e_1)) ∈ \E⟦τ_1 → τ_2⟧δ$,
	więc pokażemy $(k, γ(e_1\, e_2)) ∈ \E⟦τ_2⟧$
	korzystając z lematu Bind.
	Musimy pokazać, że $∀j≤k.\,∀v.\,(j,v) ∈ \V⟦τ_1→τ_2⟧δ ⇒ (j,v\,γ(e_2)) ∈ \E⟦τ_2⟧δ$.

	Zatem weźmy takie $j$ i $v$.
	Wiemy (z definicji $⊨$ i monotoniczności), że $(j, γ(e_2)) ∈ \E⟦τ_1⟧δ$,
	więc pokażemy $(j, v\,γ(e_2)) ∈ \E⟦τ_2⟧δ$
	korzystając z lematu Bind.
	Musimy pokazać, że $∀l≤j.\,∀u.\,(l,u) ∈ \V⟦τ_1⟧δ ⇒ (l, v\,u) ∈ \E⟦τ_2⟧δ$.

	Zatem weźmy takie $l$ i $u$.
	Z definicji relacji dla typu funkcyjnego wiemy, że $v$ ma postać $λx.\,e$
	i $(l, e[u/x]) ∈ \E⟦τ_2⟧δ$.
	Z Closure under expansion mamy $(l+1, v\, u) ∈ \E⟦τ_2⟧δ$,
	więc z monotoniczności $(l, v\,u) ∈ \E⟦τ_2⟧δ$ tak, jak chcieliśmy.

\end{proof}



\end{document}
