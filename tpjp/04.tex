\documentclass[a4paper, 12pt]{article}
\usepackage[utf8]{inputenc}
\usepackage{silence}
\usepackage{polski}
\usepackage{parskip}
\usepackage{amsmath,amsfonts,amssymb,amsthm}
\usepackage{mathtools}
\usepackage{enumitem}
\usepackage{newunicodechar}
\usepackage{etoolbox}
\usepackage[margin=1.2in]{geometry}
\usepackage{bussproofs}
\usepackage{listings}
\setcounter{secnumdepth}{0}

\title{Lista 4}
\author{Wiktor Kuchta (nr indeksu 315599)}

\DeclareMathOperator{\im}{Im}
\DeclareMathOperator{\rank}{rank}
\DeclareMathOperator{\Lin}{Lin}
\DeclareMathOperator{\sgn}{sgn}
\DeclareMathOperator{\Char}{char}
\newcommand{\N}{\mathbb{N}}
\newcommand{\Z}{\mathbb{Z}}
\newcommand{\Q}{\mathbb{Q}}
\newcommand{\R}{\mathbb{R}}
\newcommand{\C}{\mathbb{C}}
\newcommand{\inner}[2]{( #1 \, | \, #2)}
\newcommand{\norm}[1]{\left\lVert #1 \right\rVert}
\newcommand{\modulus}[1]{\left| #1 \right|}
\newcommand{\abs}{\modulus}
\newtheorem{theorem}{Twierdzenie}
\newtheorem{lemat}{Lemat}
\newcommand{\ol}{\overline}
\newcommand{\ul}{\underline}
\DeclareMathOperator{\tr}{tr}
\DeclareMathOperator{\diag}{diag}
\newcommand{\+}{\enspace}
\newcommand{\sump}{\sideset{}{'}{∑}} % sum prime
\newcommand{\sumb}{\sideset{}{"}{∑}} % sum bis

\newunicodechar{∅}{\emptyset} % Digr /0
\newunicodechar{∞}{\infty} % Digr 00
\newunicodechar{∂}{\partial} % Digr dP
\newunicodechar{α}{\alpha}
\newunicodechar{β}{\beta}
\newunicodechar{ξ}{\xi} % Digr c*
\newunicodechar{δ}{\delta} % Digr d*
\newunicodechar{ε}{\varepsilon}
\newunicodechar{φ}{\varphi}
\newunicodechar{γ}{\gamma} % Digr g*
\newunicodechar{θ}{\theta} % Digr h*
\newunicodechar{ι}{\iota} % Digr i*
\newunicodechar{κ}{\kappa}
\newunicodechar{λ}{\lambda}
\newunicodechar{μ}{\mu}
\newunicodechar{π}{\pi}
\newunicodechar{ψ}{\psi}
\newunicodechar{ρ}{\rho}
\newunicodechar{σ}{\sigma}
\newunicodechar{τ}{\tau}
\newunicodechar{ω}{\omega}
\newunicodechar{η}{\eta} % Digr y*
\newunicodechar{ζ}{\zeta} % Digr z*
\newunicodechar{Δ}{\Delta}
\newunicodechar{Γ}{\Gamma}
\newunicodechar{Λ}{\Lambda}
\newunicodechar{Θ}{\Theta}
\newunicodechar{Φ}{\Phi} % Digr F*
\newunicodechar{Π}{\Pi}
\newunicodechar{Ψ}{\Psi} % digr Q*
\newunicodechar{Σ}{\Sigma} % digr S*
\newunicodechar{Ω}{\Omega} % digr W*
\newunicodechar{ℕ}{\N} % Digr NN 8469 nonstandard
\newunicodechar{ℤ}{\Z} % Digr ZZ 8484 nonstandard
\newunicodechar{ℚ}{\Q} % Digr QQ 8474 nonstandard
\newunicodechar{ℝ}{\R} % Digr RR 8477 nonstandard
\newunicodechar{ℂ}{\C} % Digr CC 8450 nonstandard
\newunicodechar{∑}{\sum}
\newunicodechar{∏}{\prod}
\newunicodechar{∫}{\int}
\newunicodechar{∓}{\mp}
\newunicodechar{⌈}{\lceil} % Digr <7
\newunicodechar{⌉}{\rceil} % Digr >7
\newunicodechar{⌊}{\lfloor} % Digr 7<
\newunicodechar{⌋}{\rfloor} % Digr 7>
\newunicodechar{≅}{\cong} % Digr ?=
\newunicodechar{≡}{\equiv} % Digr 3=
\newunicodechar{◁}{\triangleleft} % Digr Tl
\newunicodechar{▷}{\triangleright} % Digr Tr
\newunicodechar{≤}{\le}
\newunicodechar{≥}{\ge}
\newunicodechar{≪}{\ll} % Digr <*
\newunicodechar{≫}{\gg} % Digr *>
\newunicodechar{≠}{\ne}
\newunicodechar{⊆}{\subseteq} % Digr (_
\newunicodechar{⊇}{\supseteq} % Digr _)
\newunicodechar{⊂}{\subset} % Digr (C
\newunicodechar{⊃}{\supset} % Digr C)
\newunicodechar{∩}{\cap} % Digr (U
\newunicodechar{∖}{\setminus} % Digr -\ 8726 nonstandard
\newunicodechar{∪}{\cup} % Digr )U
\newunicodechar{∼}{\sim} % Digr ?1
\newunicodechar{≈}{\approx} % Digr ?2
\newunicodechar{∈}{\in} % Digr (-
\newunicodechar{∋}{\ni} % Digr -)
\newunicodechar{∇}{\nabla} % Digr NB
\newunicodechar{∃}{\exists} % Digr TE
\newunicodechar{∀}{\forall} % Digr FA
\newunicodechar{∧}{\wedge} % Digr AN
\newunicodechar{∨}{\vee} % Digr OR
\newunicodechar{⊥}{\bot} % Digr -T
\newunicodechar{⊢}{\vdash} % Digr \- 8866 nonstandard
\newunicodechar{⊨}{\models} % Digr \= 8872 nonstandard
\newunicodechar{⊤}{\top} % Digr TO 8868 nonstandard
\newunicodechar{⇒}{\implies} % Digr =>
\newunicodechar{⊸}{\multimap} % Digr #> nonstandard
\newunicodechar{⇐}{\impliedby} % Digr <=
\newunicodechar{⇔}{\iff} % Digr ==
\newunicodechar{↔}{\leftrightarrow} % Digr <>
\newunicodechar{↦}{\mapsto} % Digr T> 8614 nonstandard
\newunicodechar{∘}{\circ} % Digr Ob
\newunicodechar{⊕}{\oplus} % Digr O+ 8853
\newunicodechar{⊗}{\otimes} % Digr OX 8855
\newunicodechar{⟦}{\llbracket} % Digr [[ 10214 nonstandard (needs pkg stmaryrd)
\newunicodechar{⟧}{\rrbracket} % Digr ]] 10215 nonstandard
\newunicodechar{✓}{\checkmark}

% cursed
\WarningFilter{newunicodechar}{Redefining Unicode}
\newunicodechar{·}{\ifmmode\cdot\else\textperiodcentered\fi} % Digr .M
\newunicodechar{×}{\ifmmode\times\else\texttimes\fi} % Digr *X
\newunicodechar{→}{\ifmmode\rightarrow\else\textrightarrow\fi} % Digr ->
\newunicodechar{←}{\ifmmode\leftarrow\else\textleftarrow\fi} % Digr ->
\newunicodechar{⟨}{\ifmmode\langle\else\textlangle\fi} % Digr LA 10216 nonstandard
\newunicodechar{⟩}{\ifmmode\rangle\else\textrangle\fi} % Digr RA 10217 nonstandard
\newunicodechar{…}{\ifmmode\dots\else\textellipsis\fi} % Digr .,
\newunicodechar{±}{\ifmmode\pm\else\textpm\fi} % Digr +-
\newunicodechar{¬}{\ifmmode\lnot\else\textlnot\fi} % Digr NO


\begin{document}

\maketitle

\begin{center}
	\begin{tabular}{ |*{8}{c|} }
	\hline
	1 & 2 & 3 & 4 & 5 & 6 & 7 & 8 \\
	\hline
	+ & + & + & + & + & + & + & \\
	\hline
\end{tabular}
\end{center}

\section*{1.}

\begin{align*}
	\lnot &= λb. \mathsf{if}\;b\;\mathsf{false}\;\mathsf{true} \\
	\wedge &= λbc. \mathsf{if}\;b\;c\;\mathsf{false} \\
	\vee &= λbc. \mathsf{if}\;b\;\mathsf{true}\;c \\
	\rightarrow &= λbc. \mathsf{if}\;b\;c\;\mathsf{true}
\end{align*}


\section*{2.}

\begin{align*}
	\mathsf{0} &= λfx.x \\
	\mathsf{suc} &= λnfx. f (nfx) \\
	\mathsf{Iter} &= λnfa. nfa
\end{align*}

\begin{align*}
	\mathsf{Iter}\;\mathsf{0}\;M\;N
	&≡ (λnfa. nfa) (λfx.x)\;M\;N \\
	&\rightarrow (λfa. (λfx.x)\;f\;a)\;M\;N \\
	&\rightarrow (λa. (λfx.x)\;M\;a)\;N \\
	&\rightarrow (λfx.x)\;M\;N \\
	&\rightarrow (λx.x)\;N \\
	&\rightarrow N
\end{align*}

\begin{align*}
	\mathsf{Iter}\;(\mathsf{suc}\;n)\;M\;N
	&≡ (λnfa. nfa)\;((λnfx.f(nfx))\;n)\;M\;N \\
	&\rightarrow (λnfa. nfa)\;(λfx.f(nfx))\;M\;N \\
	&\rightarrow^3 (λfx.f(nfx))\;M\;N \\
	&\rightarrow (λx.M(nMx))\;N \\
	&\rightarrow M\;(n\;M\;N)\\
	&\leftarrow^3 M\;((λnfa. nfa)\;n\;M\;N) \\
	&≡ M\;(\mathsf{Iter}\;n\;M\;N)
\end{align*}

\section*{3.}
$$G=λyf.f(yf)$$

$$∀F.\,MF=F(MF) \iff ∀F.\,MF=GMF \iff M=GM$$
W pierwszym przejściu korzystamy z $GMF=F(MF)$ oraz przechodniości,
a w drugim $η$-redukcji (za $F$ bierzemy zmienną) w prawo i kompatybilności z aplikacją w lewo.

\section*{4.}
$$M(SI)F=SI(M(SI))F=IF(M(SI)F)=F(M(SI)F)$$

\section*{5.}
\begin{align*}
	Y^1 = Y(SI) = (λx.SI(xx))(λx.SI(xx)) &= (λx.λz.Iz(xxz))(λx.λz.Iz(xxz)) \\ &= (λxz.z(xxz))(λxz.z(xxz) = Θ
\end{align*}

\section*{6.}
\begin{lstlisting}
newtype Self a = Fold { unfold :: Self a -> a }

a :: Self ((a -> b) -> a) -> (a -> b) -> b
a x y = y (unfold x x y)

theta :: (a -> a) -> a
theta = a (Fold a)
\end{lstlisting}

\section*{7.}
\begin{align*}
	h(\bar{x}) &= add(U^2_3(\bar x), U^3_3(\bar x)) \\
	mult(0,n) &= U^1_1(n) \\
	mult(S(m), n) &= h(m, mult(m, n), n)
\end{align*}

\begin{align*}
	h(\bar{x}) &= mult(U^2_3(\bar x), U^3_3(\bar x)) \\
	flippow(0,m) &= S(Z(m)) \\
	flippow(S(n), m) &= h(n, flippow(n, m), m) \\
	pow(\bar{x}) &= flippow(U^2_2(\bar{x}), U^1_2(\bar{x}))
\end{align*}

\begin{align*}
	sucfst(\bar x) &= S(U^1_2(\bar x)) \\
	h(\bar{x}) &= mult(sucfst(\bar x), U^2_2(x)) \\
	fact(0) &= 1 \\
	fact(S(n)) &= h(n, fact(n))
\end{align*}

\iffalse
\section*{8.}

\begin{align*}
	le(0,n) &= 1 \\
	le(S(m),n) &=  \\
	lt(m,n) &= le(S(m), n) \\
	min(0,n) &= 0 \\
	min(S(m), n) &= min(m,n) + lt(m, n)
\end{align*}
\fi

\end{document}
