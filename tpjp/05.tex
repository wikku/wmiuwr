\documentclass[a4paper, 12pt]{article}
\usepackage[utf8]{inputenc}
\usepackage{silence}
\usepackage{polski}
\usepackage{parskip}
\usepackage{amsmath,amsfonts,amssymb,amsthm}
\usepackage{mathtools}
\usepackage{enumitem}
\usepackage{newunicodechar}
\usepackage{etoolbox}
\usepackage[margin=1.2in]{geometry}
\usepackage{bussproofs}
\usepackage{listings}
\setcounter{secnumdepth}{0}

\title{Lista 4}
\author{Wiktor Kuchta (nr indeksu 315599)}

\DeclareMathOperator{\im}{Im}
\newcommand{\N}{\mathbb{N}}
\newcommand{\Z}{\mathbb{Z}}
\newcommand{\ol}{\overline}
\newcommand{\ul}{\underline}
\newcommand{\+}{\enspace}

\newunicodechar{∅}{\emptyset} % Digr /0
\newunicodechar{∞}{\infty} % Digr 00
\newunicodechar{∂}{\partial} % Digr dP
\newunicodechar{α}{\alpha}
\newunicodechar{β}{\beta}
\newunicodechar{ξ}{\xi} % Digr c*
\newunicodechar{δ}{\delta} % Digr d*
\newunicodechar{ε}{\varepsilon}
\newunicodechar{φ}{\varphi}
\newunicodechar{γ}{\gamma} % Digr g*
\newunicodechar{θ}{\theta} % Digr h*
\newunicodechar{ι}{\iota} % Digr i*
\newunicodechar{κ}{\kappa}
\newunicodechar{λ}{\lambda}
\newunicodechar{μ}{\mu}
\newunicodechar{π}{\pi}
\newunicodechar{ψ}{\psi}
\newunicodechar{ρ}{\rho}
\newunicodechar{σ}{\sigma}
\newunicodechar{τ}{\tau}
\newunicodechar{ω}{\omega}
\newunicodechar{η}{\eta} % Digr y*
\newunicodechar{ζ}{\zeta} % Digr z*
\newunicodechar{Δ}{\Delta}
\newunicodechar{Γ}{\Gamma}
\newunicodechar{Λ}{\Lambda}
\newunicodechar{Θ}{\Theta}
\newunicodechar{Φ}{\Phi} % Digr F*
\newunicodechar{Π}{\Pi}
\newunicodechar{Ψ}{\Psi} % digr Q*
\newunicodechar{Σ}{\Sigma} % digr S*
\newunicodechar{Ω}{\Omega} % digr W*
\newunicodechar{ℕ}{\N} % Digr NN 8469 nonstandard
\newunicodechar{ℤ}{\Z} % Digr ZZ 8484 nonstandard
\newunicodechar{ℚ}{\Q} % Digr QQ 8474 nonstandard
\newunicodechar{ℝ}{\R} % Digr RR 8477 nonstandard
\newunicodechar{ℂ}{\C} % Digr CC 8450 nonstandard
\newunicodechar{∑}{\sum}
\newunicodechar{∏}{\prod}
\newunicodechar{∫}{\int}
\newunicodechar{∓}{\mp}
\newunicodechar{⌈}{\lceil} % Digr <7
\newunicodechar{⌉}{\rceil} % Digr >7
\newunicodechar{⌊}{\lfloor} % Digr 7<
\newunicodechar{⌋}{\rfloor} % Digr 7>
\newunicodechar{≅}{\cong} % Digr ?=
\newunicodechar{≡}{\equiv} % Digr 3=
\newunicodechar{◁}{\triangleleft} % Digr Tl
\newunicodechar{▷}{\triangleright} % Digr Tr
\newunicodechar{≤}{\le}
\newunicodechar{≥}{\ge}
\newunicodechar{≪}{\ll} % Digr <*
\newunicodechar{≫}{\gg} % Digr *>
\newunicodechar{≠}{\ne}
\newunicodechar{⊆}{\subseteq} % Digr (_
\newunicodechar{⊇}{\supseteq} % Digr _)
\newunicodechar{⊂}{\subset} % Digr (C
\newunicodechar{⊃}{\supset} % Digr C)
\newunicodechar{∩}{\cap} % Digr (U
\newunicodechar{∖}{\setminus} % Digr -\ 8726 nonstandard
\newunicodechar{∪}{\cup} % Digr )U
\newunicodechar{∼}{\sim} % Digr ?1
\newunicodechar{≈}{\approx} % Digr ?2
\newunicodechar{∈}{\in} % Digr (-
\newunicodechar{∋}{\ni} % Digr -)
\newunicodechar{∇}{\nabla} % Digr NB
\newunicodechar{∃}{\exists} % Digr TE
\newunicodechar{∀}{\forall} % Digr FA
\newunicodechar{∧}{\wedge} % Digr AN
\newunicodechar{∨}{\vee} % Digr OR
\newunicodechar{⊥}{\bot} % Digr -T
\newunicodechar{⊢}{\vdash} % Digr \- 8866 nonstandard
\newunicodechar{⊨}{\models} % Digr \= 8872 nonstandard
\newunicodechar{⊤}{\top} % Digr TO 8868 nonstandard
\newunicodechar{⇒}{\implies} % Digr =>
\newunicodechar{⊸}{\multimap} % Digr #> nonstandard
\newunicodechar{⇐}{\impliedby} % Digr <=
\newunicodechar{⇔}{\iff} % Digr ==
\newunicodechar{↔}{\leftrightarrow} % Digr <>
\newunicodechar{↦}{\mapsto} % Digr T> 8614 nonstandard
\newunicodechar{∘}{\circ} % Digr Ob
\newunicodechar{⊕}{\oplus} % Digr O+ 8853
\newunicodechar{⊗}{\otimes} % Digr OX 8855
\newunicodechar{⟦}{\llbracket} % Digr [[ 10214 nonstandard (needs pkg stmaryrd)
\newunicodechar{⟧}{\rrbracket} % Digr ]] 10215 nonstandard
\newunicodechar{✓}{\checkmark}

% cursed
\WarningFilter{newunicodechar}{Redefining Unicode}
\newunicodechar{·}{\ifmmode\cdot\else\textperiodcentered\fi} % Digr .M
\newunicodechar{×}{\ifmmode\times\else\texttimes\fi} % Digr *X
\newunicodechar{→}{\ifmmode\rightarrow\else\textrightarrow\fi} % Digr ->
\newunicodechar{←}{\ifmmode\leftarrow\else\textleftarrow\fi} % Digr ->
\newunicodechar{⟨}{\ifmmode\langle\else\textlangle\fi} % Digr LA 10216 nonstandard
\newunicodechar{⟩}{\ifmmode\rangle\else\textrangle\fi} % Digr RA 10217 nonstandard
\newunicodechar{…}{\ifmmode\dots\else\textellipsis\fi} % Digr .,
\newunicodechar{±}{\ifmmode\pm\else\textpm\fi} % Digr +-
\newunicodechar{¬}{\ifmmode\lnot\else\textlnot\fi} % Digr NO


\begin{document}

\maketitle

\begin{center}
	\begin{tabular}{ |*{6}{c|} }
	\hline
	1 & 2 & 3 & 4 & 5 & 6 \\
	\hline
	 & + & + & + & + & + \\
	\hline
\end{tabular}
\end{center}

\section*{3.}

Niech $\mathsf{Iter} = λnfz.\,\mathsf{Rec}\;n\;(λx.f)\;z.$


\begin{align*}
	\mathsf{Iter}\;0\;M\;N &\rightarrow^3 \mathsf{Rec}\;0\;(λx.M)\;N = N \\
	\mathsf{Iter}\;(\mathsf{suc}\;n)\;M\;N &\rightarrow^3 \mathsf{Rec}\;(\mathsf{suc}\;n)\;(λx.M)\;N
	= \ul{(λx.M)\;n}\;(\mathsf{Rec}\;n\;(λx.M)\;N) \\
	&\rightarrow M\;(\mathsf{Rec}\;n\;(λx.M)\;N) \leftarrow^3 M\;(\mathsf{Iter}\;n\;M\;N)
\end{align*}

($x \not\in FV(M)$)

\section*{4.}

\begin{align*}
	\mathsf{Rec} &= λnfz.\,nfz \\
	\mathsf{0} &= λfz.\,z \\
	\mathsf{suc} &= λnfz.\,f\;n\;(\mathsf{Rec}\;n\;f\;z)\\ \\
	\mathsf{Rec}\;\mathsf{0}\;M\;N &\rightarrow^3 \mathsf{0}\;M\;N →^2 N \\
	\mathsf{Rec}\;(\mathsf{suc}\;n)\;M\;N &\rightarrow^3 \mathsf{suc}\;n\;M\;N →^3 M\;n\;(\mathsf{Rec}\;n\;M\;N)
\end{align*}

\section*{5.}



\begin{align*}
	\mathsf{0} &= λfz.\,z \\
	\mathsf{suc} &= λnfz.\,f\;n\\ \\
	\mathsf{Rec} &= λnfz.\,n\;(λn.\,f\;n\;(\mathsf{Rec}\;n\;f\;z)\;z \\
	\mathsf{Rec} &= \mathsf{Y}\;(λrnfz.\,n\;(λn.\,f\;n\;(r\;n\;f\;z)\;z) \\
	\mathsf{Rec}\;\mathsf{0}\;M\;N &\rightarrow^3 \mathsf{0}\;(λn.\,M\;n\;(\mathsf{Rec}\;n\;M\;N)\;N →^2 N \\
	\mathsf{Rec}\;(\mathsf{suc}\;n)\;M\;N &\rightarrow^3
	\mathsf{suc}\;n\;(λn.\;M\;n\;(\mathsf{Rec}\;n\;M\;N))\;N \\
	&\rightarrow^4 M\;n\;(\mathsf{Rec}\;n\;M\;N) \\ \\
	\mathsf{isZero} &= λn.\,\mathsf{Rec}\;n\;(λxy.\,\mathsf{false})\;\mathsf{true} \\
	\mathsf{pred} &= λn.\,\mathsf{Rec}\;n\;(λxy.\,x)\;\mathsf{0}
\end{align*}

\section*{6.}

Drzewo typowania musiałoby mieć taką postać:
\begin{prooftree}
	\AxiomC{}
	\RightLabel{Ass}
	\UnaryInfC{$Γ, x:σ ⊢ x : ρ → τ$}
	\AxiomC{}
	\RightLabel{Ass}
	\UnaryInfC{$Γ, x:σ ⊢ x : ρ$}
	\RightLabel{$→E$}
	\BinaryInfC{$Γ, x:σ ⊢ x\;x : τ$}
	\RightLabel{$→I$}
	\UnaryInfC{$Γ ⊢ λx.\,x\;x : σ → τ$}
\end{prooftree}
Skąd mamy sprzeczność $ρ=ρ→τ$.

\end{document}
