\documentclass[a4paper, 12pt]{article}
\usepackage[utf8]{inputenc}
\usepackage{silence}
\usepackage{polski}
\usepackage{parskip}
\usepackage{amsmath,amsfonts,amssymb,amsthm}
\usepackage{mathtools}
\usepackage{enumitem}
\usepackage{newunicodechar}
\usepackage{etoolbox}
\usepackage[margin=1.2in]{geometry}
\usepackage{bussproofs}
\usepackage{listings}
\usepackage{lplfitch}
\renewcommand{\formula}[1]{\ensuremath{#1}}
\renewcommand{\subproof}[2]{&\hfuzz\maxdimen\fitchprf{#1}{#2}\\} %https://tex.stackexchange.com/questions/494503/package-lplfitch-and-overfull-hbox
\setcounter{secnumdepth}{0}

\title{Lista 11}
\author{Wiktor Kuchta (nr indeksu 315599)}

\newtheorem*{lemat}{Lemat}

\DeclareMathOperator{\im}{Im}
\newcommand{\N}{\mathbb{N}}
\newcommand{\Z}{\mathbb{Z}}
\newcommand{\ol}{\overline}
\newcommand{\ul}{\underline}
\newcommand{\+}{\enspace}
\newcommand{\If}{\mathsf{if}}
\newcommand{\True}{\mathsf{true}}
\newcommand{\False}{\mathsf{false}}
\newcommand{\Bool}{\mathsf{Bool}}
\newcommand{\Fst}{\mathsf{fst}}
\newcommand{\Snd}{\mathsf{snd}}
\newcommand{\Pair}{\mathsf{pair}}

\newunicodechar{∅}{\emptyset} % Digr /0
\newunicodechar{∞}{\infty} % Digr 00
\newunicodechar{∂}{\partial} % Digr dP
\newunicodechar{α}{\alpha}
\newunicodechar{β}{\beta}
\newunicodechar{ξ}{\xi} % Digr c*
\newunicodechar{δ}{\delta} % Digr d*
\newunicodechar{ε}{\varepsilon}
\newunicodechar{φ}{\varphi}
\newunicodechar{γ}{\gamma} % Digr g*
\newunicodechar{θ}{\theta} % Digr h*
\newunicodechar{ι}{\iota} % Digr i*
\newunicodechar{κ}{\kappa}
\newunicodechar{λ}{\lambda}
\newunicodechar{μ}{\mu}
\newunicodechar{π}{\pi}
\newunicodechar{ψ}{\psi}
\newunicodechar{ρ}{\rho}
\newunicodechar{σ}{\sigma}
\newunicodechar{τ}{\tau}
\newunicodechar{ω}{\omega}
\newunicodechar{η}{\eta} % Digr y*
\newunicodechar{ζ}{\zeta} % Digr z*
\newunicodechar{Δ}{\Delta}
\newunicodechar{Γ}{\Gamma}
\newunicodechar{Λ}{\Lambda}
\newunicodechar{Θ}{\Theta}
\newunicodechar{Φ}{\Phi} % Digr F*
\newunicodechar{Π}{\Pi}
\newunicodechar{Ψ}{\Psi} % digr Q*
\newunicodechar{Σ}{\Sigma} % digr S*
\newunicodechar{Ω}{\Omega} % digr W*
\newunicodechar{ℕ}{\N} % Digr NN 8469 nonstandard
\newunicodechar{ℤ}{\Z} % Digr ZZ 8484 nonstandard
\newunicodechar{ℚ}{\Q} % Digr QQ 8474 nonstandard
\newunicodechar{ℝ}{\R} % Digr RR 8477 nonstandard
\newunicodechar{ℂ}{\C} % Digr CC 8450 nonstandard
\newunicodechar{∑}{\sum}
\newunicodechar{∏}{\prod}
\newunicodechar{∫}{\int}
\newunicodechar{∓}{\mp}
\newunicodechar{⌈}{\lceil} % Digr <7
\newunicodechar{⌉}{\rceil} % Digr >7
\newunicodechar{⌊}{\lfloor} % Digr 7<
\newunicodechar{⌋}{\rfloor} % Digr 7>
\newunicodechar{≅}{\cong} % Digr ?=
\newunicodechar{≡}{\equiv} % Digr 3=
\newunicodechar{◁}{\triangleleft} % Digr Tl
\newunicodechar{▷}{\triangleright} % Digr Tr
\newunicodechar{≤}{\le}
\newunicodechar{≥}{\ge}
\newunicodechar{≪}{\ll} % Digr <*
\newunicodechar{≫}{\gg} % Digr *>
\newunicodechar{≠}{\ne}
\newunicodechar{⊆}{\subseteq} % Digr (_
\newunicodechar{⊇}{\supseteq} % Digr _)
\newunicodechar{⊂}{\subset} % Digr (C
\newunicodechar{⊃}{\supset} % Digr C)
\newunicodechar{∩}{\cap} % Digr (U
\newunicodechar{∖}{\setminus} % Digr -\ 8726 nonstandard
\newunicodechar{∪}{\cup} % Digr )U
\newunicodechar{∼}{\sim} % Digr ?1
\newunicodechar{≈}{\approx} % Digr ?2
\newunicodechar{∈}{\in} % Digr (-
\newunicodechar{∋}{\ni} % Digr -)
\newunicodechar{∇}{\nabla} % Digr NB
\newunicodechar{∃}{\exists} % Digr TE
\newunicodechar{∀}{\forall} % Digr FA
\newunicodechar{∧}{\wedge} % Digr AN
\newunicodechar{∨}{\vee} % Digr OR
\newunicodechar{⊥}{\bot} % Digr -T
\newunicodechar{⊢}{\vdash} % Digr \- 8866 nonstandard
\newunicodechar{⊨}{\models} % Digr \= 8872 nonstandard
\newunicodechar{⊤}{\top} % Digr TO 8868 nonstandard
\newunicodechar{⇒}{\implies} % Digr =>
\newunicodechar{⊸}{\multimap} % Digr #> nonstandard
\newunicodechar{⇐}{\impliedby} % Digr <=
\newunicodechar{⇔}{\iff} % Digr ==
\newunicodechar{↔}{\leftrightarrow} % Digr <>
\newunicodechar{↦}{\mapsto} % Digr T> 8614 nonstandard
\newunicodechar{∘}{\circ} % Digr Ob
\newunicodechar{⊕}{\oplus} % Digr O+ 8853
\newunicodechar{⊗}{\otimes} % Digr OX 8855
\newunicodechar{⟦}{\llbracket} % Digr [[ 10214 nonstandard (needs pkg stmaryrd)
\newunicodechar{⟧}{\rrbracket} % Digr ]] 10215 nonstandard
\newunicodechar{✓}{\checkmark}
\newunicodechar{□}{\square}

% cursed
\WarningFilter{newunicodechar}{Redefining Unicode}
\newunicodechar{·}{\ifmmode\cdot\else\textperiodcentered\fi} % Digr .M
\newunicodechar{×}{\ifmmode\times\else\texttimes\fi} % Digr *X
\newunicodechar{→}{\ifmmode\rightarrow\else\textrightarrow\fi} % Digr ->
\newunicodechar{←}{\ifmmode\leftarrow\else\textleftarrow\fi} % Digr ->
\newunicodechar{⟨}{\ifmmode\langle\else\textlangle\fi} % Digr LA 10216 nonstandard
\newunicodechar{⟩}{\ifmmode\rangle\else\textrangle\fi} % Digr RA 10217 nonstandard
\newunicodechar{…}{\ifmmode\dots\else\textellipsis\fi} % Digr .,
\newunicodechar{±}{\ifmmode\pm\else\textpm\fi} % Digr +-
\newunicodechar{¬}{\ifmmode\lnot\else\textlnot\fi} % Digr NO


\begin{document}

\maketitle

\begin{center}
	\begin{tabular}{ |*{7}{c|} }
	\hline
	1 & 2 & 3 & 4 & 5 & 6 \\
	\hline
	+ & + & + & + & + & + \\
	\hline
\end{tabular}
\end{center}

\section*{1.}

\begin{prooftree}
	\AxiomC{}
	\LeftLabel{(Sort)}
	\UnaryInfC{$∅ ⊢ *:□$}
	\AxiomC{}
	\LeftLabel{(Sort)}
	\UnaryInfC{$∅ ⊢*:□$}
	\LeftLabel{(Form)}
	\BinaryInfC{$∅ ⊢ * → * : □$}
	\AxiomC{}
	\LeftLabel{(Sort)}
	\UnaryInfC{$∅ ⊢*:□$}
	\LeftLabel{(Form)}
	\BinaryInfC{$∅ ⊢ (* → *) → * : □$}
\end{prooftree}

\begin{center}
\fitchprf{}{
	\pline[1.]{* : □}[Sort] \\
	\pline[2.]{* → * : □}[Form, 1, 1] \\
	\pline[3.]{(* → *) → * : □}[Form, 2, 1]
}
\end{center}

\section*{2.}

\begin{center}
\fitchprf{
	\pline[1.]{α : *} \\
	\pline[2.]{x : α}
}{
	\subproof{
		\pline[3.]{y:α}
	}{
		\pline[4.]{x : α}[Weak, 2]
	}
	\pline[5.]{λ{y:α}.\,x : α → α}[Abst, 3–4] \\
	\pline[6.]{λ{y:α}.\,x : (λ{β:*}.\,β→β)\,α}[Conv, 5]
}
\end{center}

\section*{3.}

\setlength{\fitchprfwidth}{4.5in}
\begin{center}
\fitchprf{}{
	\subproof{\pline[1.]{∀x∈S.\,P(x) → Q(x)}}{
		\subproof{\pline[2.]{∀y∈S.\,P(y)}}{
			\boxedsubproof[3.]{z}{z∈S}{
				\pline[4.]{P(z)}[$∀$E, 2] \\
				\pline[5.]{P(z) → Q(z)}[$∀$E, 1] \\
				\pline[6.]{Q(z)}[$→$E, 5, 4]
			}
			\pline[7.]{∀z∈S.\,Q(z)}[$∀$I, 3–6]
		}
		\pline[8.]{(∀y∈S.\,P(y)) → ∀z∈S.\,Q(z)}[$→$I, 2–7]
	}
	\pline[9.]{
		(∀x∈S.\,P(x) → Q(x)) → (∀y∈S.\,P(y)) → ∀z∈S.\,Q(z)
	}[$→$I, 1–8]
}
\end{center}

\setlength{\fitchprfwidth}{4.5in}
\begin{center}
\fitchprf{
	\pline[1.]{S : *} \\
	\pline[2.]{P : S → *} \\
	\pline[3.]{Q : S → *}
}{
	\subproof{\pline[4.]{t : Πx:S.\,P(x) → Q(x)}}{
		\subproof{\pline[5.]{a : Πy:S.\,P(y)}}{
			\subproof{\pline[6.]{z:S}}{
				\pline[7.]{az : P(z)}[Appl, 5, 6] \\
				\pline[8.]{tz : P(z) → Q(z)}[Appl, 4, 6] \\
				\pline[9.]{t z (a z) : Q(z)}[Appl, 8, 7]
			}
			\pline[10.]{λz.\, tz(az) : Πz:S.\,Q(z)}[Abst, 6–9]
		}
		\pline[11.]{λaz.tz(az) : (Πy:S.\,P(y)) → Πz:S.\,Q(z)}[Abst, 5–10]
	}
	\pline[12.]{
		\begin{aligned}
			&λtaz.tz(az) \\
			&: (Πx:S.\,P(x) → Q(x)) → (Πy:S.\,P(y)) → Πz:S.\,Q(z)
		\end{aligned}
	}[Abst, 4–11]
}
\end{center}

\section*{4.}

\begin{prooftree}
	\AxiomC{}
	\LeftLabel{(Sort)}
	\UnaryInfC{$∅ ⊢ * : □$}
	\AxiomC{}
	\LeftLabel{(Sort)}
	\UnaryInfC{$∅ ⊢ * : □$}
	\LeftLabel{(Var)}
	\UnaryInfC{$α : * ⊢ α : *$}
	\LeftLabel{(Form)}
	\BinaryInfC{$∅ ⊢ (Πα:*.\,α) : *$}
\end{prooftree}

\setlength{\fitchprfwidth}{2in}
\begin{center}
\fitchprf{}{
	\pline[1.]{* : □}[Sort] \\
	\subproof{\pline[2.]{α:*}}{
		\pline[3.]{α : *}[Var, 1]
	}
	\pline[4.]{(Πα:*.\,α) : *}[Form, 1, 2–3]
}
\end{center}

\section*{5.}

\subsection*{(a)}
Przypomnienie:
\begin{prooftree}
	\AxiomC{$Γ⊢A:s_1$}
	\AxiomC{$Γ,x:A⊢B:s_2$}
	\LeftLabel{(Form)}
	\BinaryInfC{$Γ⊢(Πx:A.\,B):s_2$}
\end{prooftree}
$$A → B = Π\_:A.\,B$$
Aby móc uformować $A → B$ bądź $Πx:A.\,B$ potrzebujemy reguły (Form)
dla pary $(s_1, s_2)$, gdzie $s_1$ jest rodzajem $A$, a $s_2$ rodzajem $B$.

Aby uformować $⊥ = Πα : *.\, α$ potrzebujemy reguły dla $(□, *)$.
$$M = λS : *.λP : \ul{S → *}.λx : S.\ul{P x → ⊥}.$$
Aby uformować pierwszy i drugi podkreślony typ, potrzebujemy odpowiednio $(*,□)$ i $(*, *)$.
Innymi słowy, w wyrażeniu korzystamy z polimorfizmu ($⊥$), typów zależnych ($P$) i przestrzeni funkcyjnej
($P x → ⊥$), co wymaga $λP2$.

$M$ jest jednak funkcją zwracającą typ (konstruktorem typu), zatem otypowanie $M$ wymaga całego $λC$.

\subsection*{(b)}

\setlength{\fitchprfwidth}{4in}
\begin{center}
\fitchprf{

}{
	\subproof{\pline[1.]{S:*}}{
		\subproof{\pline[2.]{P:S→*}}{
			\subproof{\pline[3.]{x:S}}{
				\pline[4.]{P x : *}[Appl, 2, 3] \\
				\pline[5.]{⊥ : *}[Form] \\
				\pline[6.]{P x → ⊥ : *}[Form, 4, 5]
			}
			\pline[7.]{λx : S.\,P x → ⊥ : S → *}[Abst, 3–6]
		}
		\pline[8.]{λP : S → *.\, λx : S.\,P x → ⊥ : (S→*) → S→*}[Abst, 2–7]
	}
	\pline[9.]{
		\begin{aligned}
		&λS:*.\,λP : S → *.\, λx : S.\,P x → ⊥ \\
		&: ΠS:*.\,(S→*) → S → *
		\end{aligned}
	}[Abst, 1–8]
}
\end{center}

\subsection*{(c)}
$S → *$ interpretuje się zwykle jako predykaty na $S$ czy typ własności $S$
(w językach terminalnych $S → \mathtt{Bool}$ może wyrazić tylko własności rozstrzygalne).
Jeśli $P : S → *$, to w teorii konstruktywnej $P s$ możemy rozumieć jako typ dowodów własności $P$ dla $a$.
Jeśli dowody nas nie obchodzą, to $P$ rozumiemy jako zbiór elementów z $S$.

Zatem term $M$ zamienia predykat (na dowolnym $S$) na jego negację.
Teoriomnogościowo jest to dopełnienie zbioru (w uniwersum $S$).

\section*{6.}


\subsection*{(a)}
\setlength{\fitchprfwidth}{4.7in}
\begin{center}
\fitchprf{
	\pline[1.]{S:*}\\
	\pline[2.]{P:S→*}
}{
	\subproof{\pline[3.]{c : Πα:*.\,(Πx:S.\,Px→α)→α}}{
		\subproof{\pline[4.]{n : Πx:S.\,Px→⊥}}{
			\pline[5.]{⊥ : *}[Form] \\
			\pline[6.]{c\,⊥ : (Πx:S.\,Px→⊥)→⊥}[Appl, 3, 5]\\
			\pline[7.]{c\,⊥\,n : ⊥}[Appl]
		}
		\pline[8.]{λn.\,c\,⊥\,n : (Πx:S.\,Px→⊥) → ⊥}[Abst, 4–7]
	}
	\pline[9.]{
		\begin{aligned}
		&λcn.\,c\,⊥\,n \\
		&: (Πα:*.\,(Πx:S.\,Px→α) → α) → (Πx:S.\,Px→⊥) → ⊥
		\end{aligned}
	}[Abst, 3–8]
}
\end{center}

\subsection*{(b)}
Korzystamy z typów zależnych i polimorfizmu, ale nie z konstruktorów typów, więc wystarczy $λP2$.

\subsection*{(c)}

Term $N$ to jedna strona prawa De Morgana:
jeśli istnieje element $S$ spełniający $P$,
to nieprawdą jest, że dla każdego elementu $S$ zachodzi $¬P$.

\end{document}
