\documentclass[a4paper, 12pt]{article}
\usepackage[utf8]{inputenc}
\usepackage{silence}
\usepackage{polski}
\usepackage{parskip}
\usepackage{amsmath,amsfonts,amssymb,amsthm}
\usepackage{mathtools}
\usepackage{enumitem}
\usepackage{newunicodechar}
\usepackage{etoolbox}
\usepackage[margin=1.2in]{geometry}
\usepackage{bussproofs}
\setcounter{secnumdepth}{0}

\title{Lista 3}
\author{Wiktor Kuchta (nr indeksu 315599)}

\DeclareMathOperator{\im}{Im}
\DeclareMathOperator{\rank}{rank}
\DeclareMathOperator{\Lin}{Lin}
\DeclareMathOperator{\sgn}{sgn}
\DeclareMathOperator{\Char}{char}
\newcommand{\N}{\mathbb{N}}
\newcommand{\Z}{\mathbb{Z}}
\newcommand{\Q}{\mathbb{Q}}
\newcommand{\R}{\mathbb{R}}
\newcommand{\C}{\mathbb{C}}
\newcommand{\inner}[2]{( #1 \, | \, #2)}
\newcommand{\norm}[1]{\left\lVert #1 \right\rVert}
\newcommand{\modulus}[1]{\left| #1 \right|}
\newcommand{\abs}{\modulus}
\newtheorem{theorem}{Twierdzenie}
\newtheorem{lemat}{Lemat}
\newcommand{\ol}{\overline}
\newcommand{\ul}{\underline}
\DeclareMathOperator{\tr}{tr}
\DeclareMathOperator{\diag}{diag}
\newcommand{\+}{\enspace}
\newcommand{\sump}{\sideset{}{'}{∑}} % sum prime
\newcommand{\sumb}{\sideset{}{"}{∑}} % sum bis

\newunicodechar{∅}{\emptyset} % Digr /0
\newunicodechar{∞}{\infty} % Digr 00
\newunicodechar{∂}{\partial} % Digr dP
\newunicodechar{α}{\alpha}
\newunicodechar{β}{\beta}
\newunicodechar{ξ}{\xi} % Digr c*
\newunicodechar{δ}{\delta} % Digr d*
\newunicodechar{ε}{\varepsilon}
\newunicodechar{φ}{\varphi}
\newunicodechar{γ}{\gamma} % Digr g*
\newunicodechar{θ}{\theta} % Digr h*
\newunicodechar{ι}{\iota} % Digr i*
\newunicodechar{κ}{\kappa}
\newunicodechar{λ}{\lambda}
\newunicodechar{μ}{\mu}
\newunicodechar{π}{\pi}
\newunicodechar{ψ}{\psi}
\newunicodechar{ρ}{\rho}
\newunicodechar{σ}{\sigma}
\newunicodechar{τ}{\tau}
\newunicodechar{ω}{\omega}
\newunicodechar{η}{\eta} % Digr y*
\newunicodechar{ζ}{\zeta} % Digr z*
\newunicodechar{Δ}{\Delta}
\newunicodechar{Γ}{\Gamma}
\newunicodechar{Λ}{\Lambda}
\newunicodechar{Θ}{\Theta}
\newunicodechar{Φ}{\Phi} % Digr F*
\newunicodechar{Π}{\Pi}
\newunicodechar{Ψ}{\Psi} % digr Q*
\newunicodechar{Σ}{\Sigma} % digr S*
\newunicodechar{Ω}{\Omega} % digr W*
\newunicodechar{ℕ}{\N} % Digr NN 8469 nonstandard
\newunicodechar{ℤ}{\Z} % Digr ZZ 8484 nonstandard
\newunicodechar{ℚ}{\Q} % Digr QQ 8474 nonstandard
\newunicodechar{ℝ}{\R} % Digr RR 8477 nonstandard
\newunicodechar{ℂ}{\C} % Digr CC 8450 nonstandard
\newunicodechar{∑}{\sum}
\newunicodechar{∏}{\prod}
\newunicodechar{∫}{\int}
\newunicodechar{∓}{\mp}
\newunicodechar{⌈}{\lceil} % Digr <7
\newunicodechar{⌉}{\rceil} % Digr >7
\newunicodechar{⌊}{\lfloor} % Digr 7<
\newunicodechar{⌋}{\rfloor} % Digr 7>
\newunicodechar{≅}{\cong} % Digr ?=
\newunicodechar{≡}{\equiv} % Digr 3=
\newunicodechar{◁}{\triangleleft} % Digr Tl
\newunicodechar{▷}{\triangleright} % Digr Tr
\newunicodechar{≤}{\le}
\newunicodechar{≥}{\ge}
\newunicodechar{≪}{\ll} % Digr <*
\newunicodechar{≫}{\gg} % Digr *>
\newunicodechar{≠}{\ne}
\newunicodechar{⊆}{\subseteq} % Digr (_
\newunicodechar{⊇}{\supseteq} % Digr _)
\newunicodechar{⊂}{\subset} % Digr (C
\newunicodechar{⊃}{\supset} % Digr C)
\newunicodechar{∩}{\cap} % Digr (U
\newunicodechar{∖}{\setminus} % Digr -\ 8726 nonstandard
\newunicodechar{∪}{\cup} % Digr )U
\newunicodechar{∼}{\sim} % Digr ?1
\newunicodechar{≈}{\approx} % Digr ?2
\newunicodechar{∈}{\in} % Digr (-
\newunicodechar{∋}{\ni} % Digr -)
\newunicodechar{∇}{\nabla} % Digr NB
\newunicodechar{∃}{\exists} % Digr TE
\newunicodechar{∀}{\forall} % Digr FA
\newunicodechar{∧}{\wedge} % Digr AN
\newunicodechar{∨}{\vee} % Digr OR
\newunicodechar{⊥}{\bot} % Digr -T
\newunicodechar{⊢}{\vdash} % Digr \- 8866 nonstandard
\newunicodechar{⊨}{\models} % Digr \= 8872 nonstandard
\newunicodechar{⊤}{\top} % Digr TO 8868 nonstandard
\newunicodechar{⇒}{\implies} % Digr =>
\newunicodechar{⊸}{\multimap} % Digr #> nonstandard
\newunicodechar{⇐}{\impliedby} % Digr <=
\newunicodechar{⇔}{\iff} % Digr ==
\newunicodechar{↔}{\leftrightarrow} % Digr <>
\newunicodechar{↦}{\mapsto} % Digr T> 8614 nonstandard
\newunicodechar{∘}{\circ} % Digr Ob
\newunicodechar{⊕}{\oplus} % Digr O+ 8853
\newunicodechar{⊗}{\otimes} % Digr OX 8855
\newunicodechar{⟦}{\llbracket} % Digr [[ 10214 nonstandard (needs pkg stmaryrd)
\newunicodechar{⟧}{\rrbracket} % Digr ]] 10215 nonstandard
\newunicodechar{✓}{\checkmark}

% cursed
\WarningFilter{newunicodechar}{Redefining Unicode}
\newunicodechar{·}{\ifmmode\cdot\else\textperiodcentered\fi} % Digr .M
\newunicodechar{×}{\ifmmode\times\else\texttimes\fi} % Digr *X
\newunicodechar{→}{\ifmmode\rightarrow\else\textrightarrow\fi} % Digr ->
\newunicodechar{←}{\ifmmode\leftarrow\else\textleftarrow\fi} % Digr ->
\newunicodechar{⟨}{\ifmmode\langle\else\textlangle\fi} % Digr LA 10216 nonstandard
\newunicodechar{⟩}{\ifmmode\rangle\else\textrangle\fi} % Digr RA 10217 nonstandard
\newunicodechar{…}{\ifmmode\dots\else\textellipsis\fi} % Digr .,
\newunicodechar{±}{\ifmmode\pm\else\textpm\fi} % Digr +-
\newunicodechar{¬}{\ifmmode\lnot\else\textlnot\fi} % Digr NO


\begin{document}

\maketitle

\begin{center}
	\begin{tabular}{ |*{6}{c|} }
	\hline
	1 & 2 & 3 & 4 & 5 & 6 \\
	\hline
	+ & + & + & + & & + \\
	\hline
\end{tabular}
\end{center}

\section*{1.}

Podkreślone $β$-redeksy:
$$\ul{(λx.x)(\ul{(λx.x)(λz.\ul{(λx.x)z})})}$$

\section*{2.}

\begin{align*}
	(λx.xx)(λyz.yz)
		&\rightarrow xx[x:=(λyz.yz)] ≡ (λyz.yz)(λyz.yz) \\
		&\rightarrow (λz.yz)[y:=(λyz.yz)] ≡ λz.(λyz.yz)z \\
		&\rightarrow λz.(λz.yz)[y:=z] ≡ λz.(λt.yt)[y:=z] ≡ λz.(λt.zt) ≡ λzt.zt \\
\end{align*}

\section*{3.}

Pokażemy równoważność poniższych reguł:

\begin{center}
	\AxiomC{}
	\RightLabel{$η$}
	\UnaryInfC{$λx.Mx=M$}
	\DisplayProof
	\, gdzie $x\not\in FV(M)$

	\AxiomC{$Mx=Nx$}
	\RightLabel{Ext}
	\UnaryInfC{$M=N$}
	\DisplayProof
	\, gdzie $x\not\in FV(M)∪FV(N)$

\end{center}

Wyprowadzamy (Ext) z $(η)$. Zakładamy $x \not\in FV(M) ∪ FV(N)$.
\begin{prooftree}
	\AxiomC{}
	\RightLabel{$η$}
	\UnaryInfC{$λx.Mx=M$}
	\RightLabel{Sym}
	\UnaryInfC{$M=λx.Mx$}
	\AxiomC{$Mx=Nx$}
	\RightLabel{MonAbs}
	\UnaryInfC{$λx.Mx=λx.Nx$}
	\AxiomC{}
	\RightLabel{$η$}
	\UnaryInfC{$λx.Nx=N$}
	\RightLabel{Trans}
	\BinaryInfC{$λx.Mx=N$}
	\RightLabel{Trans}
	\BinaryInfC{$M=N$}
\end{prooftree}

Wyprowadzamy $(η)$ z (Ext). Zakładamy $x \not\in FV(M)$.
\begin{prooftree}
	\AxiomC{}
	\RightLabel{$β$}
	\UnaryInfC{$(λx.Mx)x=Mx$}
	\RightLabel{Ext}
	\UnaryInfC{$λx.Mx=M$}
\end{prooftree}

\section*{4.}
Ustalmy $λ$-termy $M$ i $N$ takie, że $M=N$.
Indukcją względem struktury kontekstów pokażemy,
że dla każdego kontekstu $C$ zachodzi $C[M] = C[N]$.

\begin{enumerate}
	\item
		Jeśli $C$ ma postać $x$ (zmienna),
		to $C[M] ≡ C[N] ≡ x$, a więc $C[M] = C[N]$ wnioskujemy regułą Refl.
	\item
		Jeśli $C≡[]$, to $C[M] ≡ M$ i $C[N] ≡ N$,
		więc korzystamy bezpośrednio z założenia $M=N$.
	\item
		Jeśli $C$ ma postać $C_1[]C_2[]$,
		to z założenia indukcyjnego mamy $C_1[M] = C_1[N]$ oraz $C_2[M]=C_2[N]$.
		Korzystając z MonApp otrzymujemy
		$$C[M] ≡ C_1[M]C_2[M] = C_1[N]C_2[N] ≡ C[N].$$
	\item
		Jeśli $C$ ma postać $λx. C_1[]$,
		to z założenia indukcyjnego mamy $C_1[M] = C_1[N]$.
		Korzystając z MonAbs otrzymujemy
		$$C[M] ≡ λx. C_1[M] = λx. C_1[N] ≡ C[N].$$
\end{enumerate}


\section*{6.}

Niech $ω_3 ≡ λx. (λt_1. (λt_2. (λt_3. x t_3) t_2) t_1) x$,
tzn. standardowa $ω ≡ λx. x x$, tylko że $x$ którego aplikujemy został trzykrotnie $η$-ekspandowany.

\subsection*{a)}

$ω_3 ω_3$

\subsection*{b)}

$ω ω (ω_3 ω_3)$

\end{document}
